\documentclass[12pt]{article}

\usepackage[]{amsmath}
\usepackage[]{amsthm}
\usepackage[]{amsfonts}
\usepackage[]{amssymb}
\usepackage{blindtext}
\usepackage[a4paper, total={7.5in, 11in}]{geometry}
\usepackage{graphicx}
\usepackage{listings}
\usepackage{color}
\usepackage{array}
\usepackage{makecell}

\definecolor{dkgreen}{rgb}{0,0.6,0}
\definecolor{gray}{rgb}{0.5,0.5,0.5}
\definecolor{mauve}{rgb}{0.58,0,0.82}

\pagenumbering{arabic}
\let\cleardoublepage\clearpage

\lstset{frame=tb,
  language=Java,
  aboveskip=3mm,
  belowskip=3mm,
  showstringspaces=false,
  columns=flexible,
  basicstyle={\small\ttfamily},
  numbers=left,
  numberstyle=\small\color{black},
  keywordstyle=\color{blue},
  commentstyle=\color{dkgreen},
  stringstyle=\color{mauve},
  breaklines=true,
  breakatwhitespace=true,
  tabsize=4
}

\title{EECS4461 Project Proposal}
\author{
    Jerry Wu (217545898, jerrywu0@my.yorku.ca)\\
    Joseph Isedowo  (216390825, gbemmy12@my.yorku.ca)
}
\date{Due Feb. 25 2024}

\begin{document}
\maketitle

\section*{Topic and Method}
\subsection*{Inquiry prompt }    
 What are some Templates and resources for end of life planning that can be added to Google inactive account manager?

\subsection*{Description and method}
In this modern era, our lives are increasingly intertwined with digital platforms and assets, from digital photos and social media accounts to online banking and cryptocurrency. As we navigate the complexities of the digital world, the question of what happens to these digital assets upon death becomes increasingly relevant. The inheritability of digital assets is a complex issue that involves technological considerations, ethical and legal. People leave behind vast amounts of personal information stored on computers, cloud storage, and various networked systems. This causes the rise of questions about privacy, access rights, and the transferability of digital ownership after death. 

The method to which we will be answering this question is M1. We plan to create a Figma prototype of a website that is for the inheritance of digital assets. This prototype will be extensively tested by various individuals, many of which will be students at York University. The students' feedback will be useful in the design description report, as it will provide real world insight as to how our prototype concept is useful as well as how it can be improved in the long run should we choose to develop it into a fully functional web application.

\section*{Motivation}

The topic of extractive digital media technology tools for end-of-life planning is fascinating for several reasons. Firstly, it addresses a critical aspect of modern life that often goes overlooked. Furthermore, exploring how these digital tools operate and how they facilitate the end-of-life planning process provides insight into Human computer interaction (HCI). Understanding the features and functionalities of these tools could shed more light on how individuals navigate sensitive topics and make preparations for their own mortality in the digital age. 

The method we choose is interesting to us because it allows us to utilize our imagination. Through research through design methodology, we will be able to push boundaries, challenge assumptions and uncover new possibilities. This method combined with getting user feedback, ensures that our prototype will not only be imaginative but also grounded in real-world needs and contexts.

\newpage
\section*{Project Plan}

\begin{tabular}{|c|c|c|c|}
    \hline
     \textbf{Milestone name}&  \textbf{Description}&  \textbf{Contributions}& \makecell{Connections to \\final/interim deliverable}\\
     \hline
     Interim Deliverable
     
     &  \makecell{
         \textbullet \textbf{$I_1$ - Lofi figma prototype}:\\ Contains relevant boilerplate \\website outlines with\\ all relevant page\\transitions and\\ minimal images\\ \textbf{Success criteria}:\\ Prototype will represent\\ all components of the\\ website with one of\\ the pages developed\\ to medium fidelity\\
         \textbullet \textbf{$I_2$ - Interim report draft}:\\Contains all relevant\\ topics covered in the proposal\\like design, method,\\ related works, etc.\\ \textbf{Success criteria}:\\ Report contains\\ comprehensive\\ list of sources,\\ a design section\\ detailing experiment\\ influences on prototype\\ design as well\\ as design concepts\\ which were applied.\\ Relevance to the\\ 12 papers discussed\\ will be justified
     } 
     
     &  \makecell{
        \textbf{Jerry}\\ focused on creating\\ lofi prototype\\ on figma, as\\ well as gathered\\ 6 related articles\\ on Google Scholar\\ and ACM\\ library \\ \\ \textbf{Joseph}\\ focused on gathering 6  \\ related articles from \\google Scholar and \\ACM library. \\ Will work on  \\method for report \\
     }
     
     & \makecell{
        $I_1$ and $I_2$ will be\\ improved upon in\\ the final deliverable
        } \\
     \hline
     
     Final deliverable 
     
     &\makecell{
         \textbullet \textbf{$F_1$ - Med fi prototype}:  \\  Same as the interim prototype,\\ but every page on the\\ figma will be medium fidelity \\
         \textbf{Success criteria}:\\
         The prototype will be\\ improved from the\\ interim prototype by \\adding webpage\\ designs closer to a finished\\ and polished website\\
         \textbullet \textbf{$F_2$ - Final report}:\\ A report containing\\all relevant sections found\\in the draft along with\\feedback from users \\
         \textbf{Success criteria}:\\
         Same as interim\\ report, but there\\ will be user \\feedback included \\in a discussion section 
     } 
     
     & \makecell{
        \textbf{Jerry}\\ 
        Continue improving\\ the prototype until\\ it meets the\\ standards of\\ a medium fidelity \\prototype\\
        \\
        \textbf{Joseph}\\
        finish all relevant\\ sections in the report, \\
         as well as work\\ on the annotated\\ 
         bibliography\\ and infographic\\ to
         effectively \\summarize related works
     } 
     
     &\makecell{
       $F_1$ and $F_2$ are\\ improvements of\\ $I_1$ and $I_2$.
    } \\
     \hline
\end{tabular}

\end{document}
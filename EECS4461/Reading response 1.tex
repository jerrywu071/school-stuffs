\documentclass[12pt]{article}

\usepackage[]{amsmath}
\usepackage[]{amsthm}
\usepackage[]{amsfonts}
\usepackage[]{amssymb}
\usepackage{blindtext}
\usepackage[a4paper, total={7.5in, 11in}]{geometry}
\usepackage{graphicx}
\usepackage{listings}
\usepackage{color}
\usepackage{array}

\definecolor{dkgreen}{rgb}{0,0.6,0}
\definecolor{gray}{rgb}{0.5,0.5,0.5}
\definecolor{mauve}{rgb}{0.58,0,0.82}

\pagenumbering{arabic}
\let\cleardoublepage\clearpage

\lstset{frame=tb,
  language=Java,
  aboveskip=3mm,
  belowskip=3mm,
  showstringspaces=false,
  columns=flexible,
  basicstyle={\small\ttfamily},
  numbers=left,
  numberstyle=\small\color{black},
  keywordstyle=\color{blue},
  commentstyle=\color{dkgreen},
  stringstyle=\color{mauve},
  breaklines=true,
  breakatwhitespace=true,
  tabsize=4
}

\title{EECS4461 Reading response 1}
\author{Jerry Wu (217545898)}
\date{Due Mar. 21 2024}

\begin{document}
\maketitle

\section*{Summary}
\textit{P. Denning and T. Bell, "The Information Paradox," American Scientist, vol. 100, no. 2, pp. 92-95, Mar.-Apr. 2012.}\\

This article discusses the challenges and opportunities presented by the abundance of digital data in today's society. The authors go over the idea of having too much and too little information and they emphasize the importance of effective data management and analysis techniques. The authors also mention the significance of addressing ethical and societal implications related to data privacy and information overload. All in all, the article provides a thought provoking examination of the complexities surrounding the management and utilization of digital information in the modern age.

\section*{Future research directions}

\textit{What future directions could be inspired by research described in the readings?}\\

Going through the text, it is evident that many different research directions can stem from the research described. Some examples include, but are not limited to:


\begin{itemize}
    \item \textbf{Data management}: Continued research into more efficient data management methods is essential. This could involve developing algorithms for better data compression, finding duplicates in storage, and different mediums of data storage that can handle the demand for more storage due to the increasing volume of digital information.
    \item \textbf{Information Retrieval and Search Technologies}: Improving information retrieval systems and search technologies could help tackle the challenge of accessing information involving large data sets. This may involve advancements in natural language processing, machine learning, and semantic search algorithms
    \item \textbf{Information Literacy}: Promoting information literacy and digital skills in an age where technology is abundant is essential for navigating the complexities of said age. Future research could focus on developing educational programs and platforms to enhance information literacy and critical thinking skills.
\end{itemize}

\section*{Lessons learned}

\textit{What were the key learning points for you?}

Overall, there were quite a few things that I took away from reading this article. One such thing is my understanding of Murphy's law, i.e. anything that can go wrong will go wrong. In essence, despite the numerous advancements data processing capabilities in this day and age, unforeseen issues and complications can arise, leading to suboptimal outcomes or failures in information management. All in all, there are many uncertainties when it comes to making advancements in the field of technology, so one who delves into this field needs to expected the unexpected so as to not lose their sanity.

\end{document}
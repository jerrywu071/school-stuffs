\documentclass[12pt]{book}

\usepackage[]{amsmath}
\usepackage[]{amsthm}
\usepackage[]{amsfonts}
\usepackage[]{amssymb}
\usepackage{blindtext}
\usepackage[a4paper, total={6in, 8in}]{geometry}

\usepackage{listings}
\usepackage{color}

\definecolor{dkgreen}{rgb}{0,0.6,0}
\definecolor{gray}{rgb}{0.5,0.5,0.5}
\definecolor{mauve}{rgb}{0.58,0,0.82}

\lstset{frame=tb,
  language=Python,
  aboveskip=3mm,
  belowskip=3mm,
  showstringspaces=false,
  columns=flexible,
  basicstyle={\small\ttfamily},
  numbers=left,
  numberstyle=\small\color{black},
  keywordstyle=\color{blue},
  commentstyle=\color{dkgreen},
  stringstyle=\color{mauve},
  breaklines=true,
  breakatwhitespace=true,
  tabsize=4
}

\title{EECS4443 Notes}
\author{Jerry Wu}
\date{2023-09-06}

\begin{document}

\maketitle

\chapter*{Android app anatomy}

\section*{Review of basic software design concepts}

\subsection*{What is a framework?}

\subsection*{Fitt's law}

\begin{align*}
    ID=\log_2\left(\frac{D}{W} + 1\right)
\end{align*}

\subsection*{UI layout (MKB vs touch screen) + "dark patterns"}

\begin{itemize}
  \item For a mouse and desktop system, we want to have submit and cancel buttons beside each other, but when we are designing a mobile UI, we want to keep them far away so the end user doesn't touch the wrong button.
  \item Keep destructive actions like deletion and undo to avoid mistakes from the end user
  \item People are distracted, so when turning attention away from the mobile device, we want to retain a state where data is restored once the user refocuses on their device (suspend).
  \item Do not place any important functions on the edges of the screen, as the edges are usually used for system functions like back, menus, etc.
  \item Popups are bad; especially for mobile UI because they dissociate from the context of the application. Usually associated with something malicious.
  \item Always provide information and context to the user so that they can make informed decisions about actions they will perform within an application.
\end{itemize}


\end{document}
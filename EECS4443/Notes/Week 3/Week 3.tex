\documentclass[12pt]{book}

\usepackage[]{amsmath}
\usepackage[]{amsthm}
\usepackage[]{amsfonts}
\usepackage[]{amssymb}
\usepackage{blindtext}
\usepackage[a4paper, total={6in, 8in}]{geometry}

\usepackage{listings}
\usepackage{color}

\usepackage{graphicx}

\definecolor{dkgreen}{rgb}{0,0.6,0}
\definecolor{gray}{rgb}{0.5,0.5,0.5}
\definecolor{mauve}{rgb}{0.58,0,0.82}

\lstset{frame=tb,
  language=Python,
  aboveskip=3mm,
  belowskip=3mm,
  showstringspaces=false,
  columns=flexible,
  basicstyle={\small\ttfamily},
  numbers=left,
  numberstyle=\small\color{black},
  keywordstyle=\color{blue},
  commentstyle=\color{dkgreen},
  stringstyle=\color{mauve},
  breaklines=true,
  breakatwhitespace=true,
  tabsize=4
}

\graphicspath{{img}}

\title{EECS4443 Notes}
\author{Jerry Wu}
\date{2023-09-18}

\begin{document}

\maketitle

\section*{Layouts}

\subsection*{XML advantages}


\begin{itemize}
    \item XML is used exclusively for the view. We can separate concerns between function and form by using MVC.
    \item The ID attribute: \texttt{android:id="@+id/decbutton"} signifies that we are pulling a value, in this case, a button from the \texttt{R} (resources) class. 
    \item \texttt{R.java} is automatically generated based on what is in the XML view. This allows us to change a single value in the XML to change \textbf{all} occurences of that value in code.
\end{itemize}

\subsection*{Layout parameters}
An app activity can have multiple layouts.
\begin{itemize}
  \item Layouts take on the form of \texttt{layout\_foo}
  \item \texttt{LinearLayout}: every element is stacked sequentially on the screen depending on the orientation of the screen.
\end{itemize}

\subsection*{What is a layout?}
Simply put, a layout is what defines the UI elements in a given activity (screen). It is the \textbf{parent view} to all other elements within it. A \texttt{Layout} is a \texttt{ViewGroup}, and a \texttt{ViewGroup} is a \texttt{View}

\subsection*{Dynamic layout}
In a dynamic layout, the elements on the screen are not pre determined. In other words, \texttt{AdapterView} is used to populate the layout with views \textbf{at runtime} like \texttt{ListView}, \texttt{GridView}, etc. These are very common views for organizing data in many applications, since they display items in either a list or a grid respectively.
\begin{itemize}
  \item When scrolling through a list of elements, we don't want to have them all rendered at once, as that will cause performance issues (low memory). We can deal with these issues by using dynamic rendering. To achieve this, we only render what is on the screen at the moment (text, images, etc.) when they are scrolled to.
\end{itemize}

\end{document}
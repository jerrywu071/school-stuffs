\documentclass[12pt]{book}

\usepackage[]{amsmath}
\usepackage[]{amsthm}
\usepackage[]{amsfonts}
\usepackage[]{amssymb}
\usepackage{blindtext}
\usepackage[a4paper, total={6in, 8in}]{geometry}

\usepackage{listings}
\usepackage{color}

\usepackage{graphicx}

\definecolor{dkgreen}{rgb}{0,0.6,0}
\definecolor{gray}{rgb}{0.5,0.5,0.5}
\definecolor{mauve}{rgb}{0.58,0,0.82}

\lstset{frame=tb,
  language=Python,
  aboveskip=3mm,
  belowskip=3mm,
  showstringspaces=false,
  columns=flexible,
  basicstyle={\small\ttfamily},
  numbers=left,
  numberstyle=\small\color{black},
  keywordstyle=\color{blue},
  commentstyle=\color{dkgreen},
  stringstyle=\color{mauve},
  breaklines=true,
  breakatwhitespace=true,
  tabsize=4
}

\graphicspath{{img}}

\title{EECS4443 Notes}
\author{Jerry Wu}
\date{2023-09-18}

\begin{document}

\maketitle

\section*{XML advantages}

\begin{itemize}
    \item XML is used exclusively for the view. We can separate concerns between function and form by using MVC.
    \item The ID attribute: \texttt{android:id="@+id/decbutton"} signifies that we are pulling a value, in this case, a button from the \texttt{R} (resources) class. 
    \item \texttt{R.java} is automatically generated based on what is in the XML view. This allows us to change a single value in the XML to change \textbf{all} occurences of that value in code.
\end{itemize}

\subsection*{Layout parameters}
An app activity can have multiple layouts.
\begin{itemize}
  \item Layouts take on the form of \texttt{layout\_foo}
  \item \texttt{LinearLayout}: every element is stacked sequentially on the screen depending on the orientation of the screen.
\end{itemize}

\end{document}
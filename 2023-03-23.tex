\documentclass[12pt]{book}

\usepackage[]{amsmath}
\usepackage[]{amsthm}
\usepackage[]{amsfonts}
\usepackage[]{amssymb}
\usepackage{blindtext}
\usepackage[a4paper, total={6in, 8in}]{geometry}

\usepackage{listings}
\usepackage{color}

\definecolor{dkgreen}{rgb}{0,0.6,0}
\definecolor{gray}{rgb}{0.5,0.5,0.5}
\definecolor{mauve}{rgb}{0.58,0,0.82}

\lstset{frame=tb,
  language=Python,
  aboveskip=3mm,
  belowskip=3mm,
  showstringspaces=false,
  columns=flexible,
  basicstyle={\small\ttfamily},
  numbers=left,
  numberstyle=\small\color{black},
  keywordstyle=\color{blue},
  commentstyle=\color{dkgreen},
  stringstyle=\color{mauve},
  breaklines=true,
  breakatwhitespace=true,
  tabsize=4
}

\title{EECS3101 notes:: Graph algorithms continued}
\author{Jerry Wu}
\date{2023-03-23}

\begin{document}
\maketitle
\chapter*{Minimum spanning trees (MST)}

\subsection*{MST in an undirected and weighted graph}
This will be our main subject of interest today. Say we have a graph $G=(V,E)$ where we have a weight for each edge $w(\epsilon)\forall\epsilon\in E$. We want to find a sub tree (connected and acyclic subgraph) in the graph that covers every vertex $v\in V$ and has the minimum possible total weight. \textbf{A good application of this idea} would be to build a road network connecting all cities in a group of $n$ cities while minimizing the cost, or connecting all components of a circuit with the least amount of wiring. More examples would include::

\begin{itemize}
    \item Cluster analysis
    \item Approximation algorithms for the travelling salesman problem (can never be exact because it is NP complete).
    \item The list goes on!
\end{itemize}

\subsection*{What is a tree?}

A tree is an \textbf{undirected}, \textbf{connected}, and \textbf{acyclic} graph that has the following properties::

\begin{itemize}
    \item A tree with $n$ vertices has exactly $n-1$ edges.
    \item Removing any edge from the tree will cause it to become \textbf{disconnected} $\implies$ not tree.
    \item Adding an edge anywhere in the tree will create a cycle $\implies$ not tree.
    \item $MST(G)$ yields exactly $|V|$ vertices and $|V|-1$ edges by way of the tree property.
    \item Not necessarily unique. There can be multiple $MST(G)$, but we only need to find one of them.
\end{itemize}

Our goal with MST is to find a \textbf{subset} of the edge set $E$ in the original graph such that we get a tree.

\subsection*{MST algorithms}

Start with a $G=(V,E)$, continue to delete edges until we end up with $T=(V,E)$ such that $T\subset G$. It is also tangible to start with no tree and construct a tree by adding edges between vertices. Which one is more efficient? Well, it would depend on the size of the graph.

\begin{quote}
    \textit{"Don't just say 'it depends'. Keep talking!"-Larry YL Zhang 2023}
\end{quote}

\begin{itemize}
    \item The subtraction based one would take at most $|E|-|V|$ iterations ($|V|^2$ in the worst case, i.e. complete graph), whereas the addition based one would take at most $|V|$ iterations.
    \item So our best bet is to "grow" the tree from bottom up (NOT DYNAMIC PROGRAMMING!) as a means to save time instead of breaking the graph to form a tree. So this solution would also be more environmentally friendy.
\end{itemize}

\begin{quote}
    \textit{"Resist the temptation to solve all problems with dynamic programming"-Larry YL Zhang 2023}
\end{quote}
We can now create some pseudocode.
\begin{lstlisting}
GENERIC_MST(G=(V,E,w)):
    T=Null
    while T is not a spanning tree: #(|T|<|V|-1)
        find safe edge e
        T=Union(T,{e})
    return T

\end{lstlisting}

\subsection*{Safe edge}
\textit{"Always have the hope to become an MST. When you're ready for it, you will become it!"-Larry YL Zhang 2023}

A safe edge is a loop invariant $T\subset some MST$. If we always make sure that $T$ is always a subset of an MST while growing it, it will eventually become an MST. A safe edge keeps the hope of a tree becoming an MST. How can we find it? We can utilize the following algorithms::

\begin{itemize}
    \item Kruskal's algorithm
    \item Prim's algorithm
\end{itemize}

Both are based on the following::

\subsection*{Theorem}
\begin{itemize}
    \item Let $G=(V,E)$ be connected, undirected, and weighted.
    \item Let $T\subset E$ such that it is included in some $MST(G)$.
    \item Let $C$ be a connected component (tree) in the forest $G=(V,E$).
    \item Let $(u,v)$ be a minimum weighted edge crossing $C$ and some other component in $G_T$.
    \item Then the edge $(u,v)$ is safe for T.
\end{itemize}

This is known as the greedy choice property of MST (always choose the lowest weighted edge/safest edge). It will work no matter which vertex you start to grow the tree at! (example on slide 24).

\subsection*{Things to keep in mind when implementing}

\begin{itemize}
    \item We need to keep track of all the components we have so far.
    \item How can we find the safe/minimum weighted edge efficiently?
\end{itemize}

This is where the two algorithms discussed earlier will come into play (animated example on slide 27).

\begin{itemize}
    \item Prim's algorithm uses a single tree along with isolated vertices to keep track of connected components, and a priority queue (max/min heap) to find the minimum weighted edge. Centered on the root of the tree.
    \item Kruskal's uses a disjoint set to track connected components and a sorted list of all edges according to their weights to find min edge. More decentralized.
\end{itemize}

\subsection*{Prim's algorithm (pseudocode on slide 30 with comments)}
\begin{itemize}
    \item Start with some arbitrary $v\in V$ as the root.
    \item Focus on growing one tree, adding one edge at a time. The current tree is one component, and the isolated vertices are all their own individual components.
    \item Which edge should we add to the tree? Among all edges that are \textbf{incident} to the current tree, pick the \textbf{minimum} weighted edge.
    \item To get the minimum weighted edge, store each weight corresponding to each edge in a min heap, whos key is the weight of said weighted crossing edge.
\end{itemize}

\subsection*{Runtime analysis}
Assuming we use a binary min heap, the worst case runtime would be $V\log(|V|)$. $ExtractMin$ takes $\log(|V|)$ time, and we check all $|E|$ edges.


\subsection*{Kruskal's algorithm (pseudocode on slide 52)}

\begin{itemize}
    \item Sort all weights in ascending order, then start adding to MST from the lowest weight.
    \item Constraint:: Added edge cannot create a cycle! (must cross 2 components. Cannot be in the same component)
    \item This process is similar to taking the union of many smaller trees to form a bigger tree.
\end{itemize}

We need the disjoint set ADT to use Kruskal's, but this is a topic for EECS4101.

\end{document}
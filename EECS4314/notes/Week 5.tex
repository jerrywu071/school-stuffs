\documentclass[12pt]{book}

\usepackage[]{amsmath}
\usepackage[]{amsthm}
\usepackage[]{amsfonts}
\usepackage[]{amssymb}
\usepackage{blindtext}
\usepackage[a4paper, total={6in, 8in}]{geometry}
\usepackage{graphicx}
\usepackage{listings}
\usepackage{color}
\usepackage{array}

\definecolor{dkgreen}{rgb}{0,0.6,0}
\definecolor{gray}{rgb}{0.5,0.5,0.5}
\definecolor{mauve}{rgb}{0.58,0,0.82}

\pagenumbering{arabic}
\renewcommand{\chaptername}{Section}
\let\cleardoublepage\clearpage

\lstset{frame=tb,
  language=Java,
  aboveskip=3mm,
  belowskip=3mm,
  showstringspaces=false,
  columns=flexible,
  basicstyle={\small\ttfamily},
  numbers=left,
  numberstyle=\small\color{black},
  keywordstyle=\color{blue},
  commentstyle=\color{dkgreen},
  stringstyle=\color{mauve},
  breaklines=true,
  breakatwhitespace=true,
  tabsize=4
}

\title{EECS4314 week 5}
\author{Jerry Wu}
\date{2024-02-07}

\begin{document}
\maketitle
\tableofcontents

\chapter{Design patterns review}

\section{Evolution of programming abstractions}

\begin{itemize}
  \item First modern programmable computers in the 1950s were largely hardwired (first software was written in machine code)
  \item Assembly introduced through symbolic assemblers, macro processors, etc
  \item In the 1960s high level languages were created (FORTRAN, COBOL, C, etc.)
  \item HL languages were independent of machine and problem domain
\end{itemize}

\subsection{Abstractions from developers perspective}

\begin{itemize}
  \item Typed variables and user defined types created in late 1960s
  \item Modules created in early 1970s
  \item ADTs and OOP created in mid 1970s
  \item OOP design patterns, refactoring in 1990s
\end{itemize}

\section{What are design patterns? (OODP)}

\begin{itemize}
  \item Design patterns are reusable solutions to common problems
  \begin{itemize}
    \item An OODP involves a small set of classes cooperating to achieve a desired end
    \item Done via adding a level of indirection in some clever way
    \item New solutions provide the small functionality as an existing approach but in some more desirable way in terms of elegance, efficiency, and adaptability
  \end{itemize} 

  \item OODPs make heavy use of interfaces, information hiding, polymorphism, and intermediary objects
  \item Typical presentation of an OODP
  \begin{itemize}
    \item A motivating problem and its context
    \item Discussion of the possible solutions 
    \item Common variations and tradeoffs
  \end{itemize} 
\end{itemize}

\subsection{Learning design patterns}

\begin{itemize}
  \item Think of OODP as high level programming abstractions
  \begin{itemize}
    \item First, learn the basics (data structures, algorithms, tools and language details)
    \item Then learn modules, interfaces, information hiding, classes/OOP
    \item Design patterns are the next level of abstractions
    \item Architecture
  \end{itemize} 
\end{itemize}

\section{Why design patterns?}
\begin{quote}
  \textit{"It gives you a lot of ability, you become superman" - H.V. Pham 2024}
\end{quote}

\subsection{Design patterns help with}
\begin{itemize}
  \item Creating a system with
  \begin{itemize}
    \item Portability
    \item Extensibility
    \item Maintainability
    \item Reusability
    \item Scalability
  \end{itemize} 

  \item Use higher-level abstractions than variables, procedures and classes
  \item Understanding tradeoffs, appropriateness, (dis)advantages of patterns 
  \item Understanding the nature of both the system you are constructing and OOP in general
  \item Communicating about systems with other developers
  \item Giving guidance in resolving
  \begin{itemize}
    \item Non functional requirements
    \item Possible tradeoffs
  \end{itemize} 

  \item Avoiding known traps, pitfalls and temptations
  \item With easier restructuring, refactoring
  \item Faster coherent directed system evolution and maintenance based on greater understanding of OO
\end{itemize}

\subsection{Gang of four patterns (23 in total, but not all!)}

\begin{itemize}
  \item Creational (instantiation of new objects)
  \begin{itemize}
    \item Abstract factory, singleton, factory, etc.
  \end{itemize} 
  \item Structural (assembling objects and classes)
  \begin{itemize}
    \item Adapter, facade, composite, decorator, etc
  \end{itemize} 
  \item Behavioral (interaction between classes and objects)
  \begin{itemize}
    \item Iterator, observer, strategy, etc.
  \end{itemize} 
\end{itemize}



\end{document}
\documentclass[12pt]{book}

\usepackage[]{amsmath}
\usepackage[]{amsthm}
\usepackage[]{amsfonts}
\usepackage[]{amssymb}
\usepackage{blindtext}
\usepackage[a4paper, total={6in, 8in}]{geometry}
\usepackage{graphicx}
\usepackage{listings}
\usepackage{color}
\usepackage{array}

\definecolor{dkgreen}{rgb}{0,0.6,0}
\definecolor{gray}{rgb}{0.5,0.5,0.5}
\definecolor{mauve}{rgb}{0.58,0,0.82}

\pagenumbering{arabic}
\renewcommand{\chaptername}{Lecture}
\let\cleardoublepage\clearpage

\lstset{frame=tb,
  language=Java,
  aboveskip=3mm,
  belowskip=3mm,
  showstringspaces=false,
  columns=flexible,
  basicstyle={\small\ttfamily},
  numbers=left,
  numberstyle=\small\color{black},
  keywordstyle=\color{blue},
  commentstyle=\color{dkgreen},
  stringstyle=\color{mauve},
  breaklines=true,
  breakatwhitespace=true,
  tabsize=4
}

\title{EECS4314 week 2}
\author{Jerry Wu}
\date{2024-01-17}

\begin{document}
\maketitle
\tableofcontents

\chapter{UML overview}

\section*{Short history of analysis and design notations}

\begin{quote}
    \textit{"STD is state transition diagram, not the other STD" - H.V. Pham 2024}
\end{quote}

\begin{itemize}
    \item 
\end{itemize}

\begin{quote}
    \textit{"How many graphics cards do you want? Yes." - H.V. Pham 2024}
\end{quote}




\section*{Class diagrams review}
A class is a description of a set of objects that share the same attributes, operations, relationships, and semantics. Graphically, a class is drawn as a rectangle including its name, attributes, and operations in their own separate (3) compartments:

\begin{itemize}
    \item Class name (eg. Person, Car, etc.)
    
    \item Attributes (instance variables with their respective types)
    \begin{itemize}
        \item[] $+\equiv$ public
        \item[] $\#\equiv$ protected
        \item[] $-\equiv$ private
        \item[] $/\equiv$ derived
    \end{itemize} 

    \item Operations (getters, setters, misc. methods) and their respective signatures
    \begin{itemize}
        \item eg. \texttt{newEntry(n:Name, a:Address, p:PhoneNumber, d:Description)}
    \end{itemize}

\end{itemize}

\subsection*{Class relationships}

In UML, pbject interconnections (logical or physical), are modelled as relationships

\begin{itemize}
    \item Dependencies - a semantic relationship between two or more elements
    
    \item Generalization - connection from subclass to superclass. Denotes inheritance of attributes and behaviors from the superclass to the subclass and indicates a specialization in the subclass of the more general superclass
    
    \item Association - if two classes in a model need to commmunicate with each other, but are not dependent on each other, then there will be a link denoted between the two classes. Dual associations are possible
    \begin{itemize}
        \item Aggregation - specifies a whole part relationship between an aggregate (whole) and a constituent part where the part can exist independently from the aggregate. Denoted by a hollow diamond. For example (car has an owner and a manufacturer, but owner and manufacturer and owner can exist without car)
        \item Composition - Same as aggregation, but is a strong ownership relationship, so the constituent parts have to exist for the composite to exist. Denoted by a filled diamond.
    \end{itemize} 

    \item Interfaces - a named set of operations (no instance variables) that speifies the behavior of objects without showing their inner structure. It can be rendered in the model by a one or two compartment rectangle with the \textbf{stereotype} \texttt{<<interface>>} above the interface name
    \begin{itemize}
        \item Interface relationships are called \textbf{realization}, which means a concrete class \texttt{implements} an interface
    \end{itemize} 

    \item Enumeration - a user defined data type that consists of a name and ordered list of enumeration literals. Denoted using \texttt{<<enumeration>>}
    
    \item Exceptions - can be customized like any other class, denoted using \texttt{<<exception>>}
\end{itemize}

\subsection*{Package}
A package is a container like element for organizing other elements into groups. Packages can contain classes, other packages, and diagrams. They can provide controlled access between classes in different packages. Denoted using a folder like shape.

\begin{itemize}
    \item Packages can depend/relate on/to other packages just like classes
\end{itemize}

\subsection*{Component diagrams}

\begin{itemize}
    \item 
\end{itemize}

\section*{Dynamic modelling using UML}

\subsection*{Use cases}
\begin{itemize}
    \item A use case specifies the behavior of a system or part of a system
    \item A description of a set of sequences of actions, including its variants. Always  produces an observable result of the actor(user)
    \item An \textbf{actor} is either a user of the system, or another system, subclass, or class
\end{itemize}

%%finish UML notes from slide 56

\end{document}
\documentclass[12pt]{book}

\usepackage[]{amsmath}
\usepackage[]{amsthm}
\usepackage[]{amsfonts}
\usepackage[]{amssymb}
\usepackage{blindtext}
\usepackage[a4paper, total={6in, 8in}]{geometry}

\usepackage{listings}
\usepackage{color}

\definecolor{dkgreen}{rgb}{0,0.6,0}
\definecolor{gray}{rgb}{0.5,0.5,0.5}
\definecolor{mauve}{rgb}{0.58,0,0.82}

\lstset{frame=tb,
  language=C,
  aboveskip=3mm,
  belowskip=3mm,
  showstringspaces=false,
  columns=flexible,
  basicstyle={\small\ttfamily},
  numbers=left,
  numberstyle=\small\color{black},
  keywordstyle=\color{blue},
  commentstyle=\color{dkgreen},
  stringstyle=\color{mauve},
  breaklines=true,
  breakatwhitespace=true,
  tabsize=4
}

\title{EECS3221 notes}
\author{Jerry Wu}
\date{Week III 2023/05/23}

\begin{document}
\maketitle

\chapter*{Process management}

\section*{Complete tuesday stuff later!}
to complete
\section*{Memory layout of a C program}

\begin{lstlisting}
#include<stdio.h>

\end{lstlisting}

\section*{Multitasking}
We want to be able to do multiple things at the same time on a computer system. This is where multitasking comes into play. In case one process crashes, we can use \textbf{multithreading} so that the whole system doesn't crash.
\subsection*{Process creation}

\begin{itemize}
    \item \textbf{Parent} processes create \textbf{child} processes which create other processes, forming a \textbf{process tree}.
    \item A process is identified and managed via a \textbf{process identifier (pid)}.
    \item Resource sharing options::
    \begin{itemize}
        \item Parent and children share all resources
        \item children share a subset of parents' resources
        \item parent and child share no resources
    \end{itemize} 
    \item Execution options::
    \begin{itemize}
        \item Parent and child processes execute concurrently
        \item Parent process waits until child process terminates before terminating
        \item Usually, killing a parent will also kill their children (wow, that's dark)
    \end{itemize} 
    \item Addresses space
    \begin{itemize}
        \item Child can be a duplicate of parent
        \item Child has a program loaded into it
    \end{itemize} 

    \item UNIX examples::
    \begin{itemize}
        \item \texttt{fork()} is a system call that creates a new process
        \item \texttt{exec()} is a system call used after fork to replace the process' memory space with a new program (there are multiple variations of \texttt{exec()}, read sytstem call API for more info)
        \item Parent process calls \texttt{wait()} for the child to terminate
        \item Always call \texttt{exit()} at the end of the program. This is done automatically by the OS, but it is good practice to put it at the end of your code.
    \end{itemize}
\end{itemize}

\newpage
\subsection*{Creating a process by forking}
\begin{lstlisting}
#include <sys/types.h>
#include <stdio.h>
#include <unistd.h>

int main()
{
    pid_t pid;
    //fork a child process
    pid = fork();
    if(pid < 0) //error occurred because fork returned n<0
    {
        fprintf(stderr, "Fork failed");
        return 1;
    }

    else if(pid == 0) //child process
    {
        execlp("/bin/ls","ls",NULL);
    }
    else  //parent process
    {
    //parent will wait for the child to complete
        wait(NULL);
        printf("Child complete");
    }

    return 0;
}
\end{lstlisting}

\subsection*{Some things to note}

\begin{itemize}
    \item We use \texttt{fprintf()} in the error handling because we can \textbf{treat memory blocks as files in C}. Network output streams also behave in a similar manner and can be treated as files among other things as well.
    \item When the return value of \texttt{fork()} $<0$, we have an error.
    \item When $return(\texttt{fork()})=0$, we have the pid of the child process.
    \item \texttt{sys/types.h} is the system call library.
\end{itemize}

\subsection*{Process termination (good for dark humour)}

\begin{itemize}
    \item When calling \texttt{exit()}, the following things happen::
    \begin{itemize}
        \item Returns status data from child to parent via \texttt{wait()}
        \item Process' resources are deallocated by the OS
    \end{itemize}

    \item Parent may terminate the execution of child processes using the \texttt{abort()} system call. Some reasons for doing so::
    \begin{itemize}
        \item Child has exceeded allocated resources
        \item Task assigned to child is no longer required
        \item The parent is exiting and the operating systems does not allow a child to continue to run
        \item Don't try the above as a person!
    \end{itemize}
    \item Some OS do not allow child to exist if its parent has been terminated. If a process terminates, then all its children must also be terminated.
    \begin{itemize}
        \item \textbf{Cascading termination} is what this is known as.
        \item This is initiated by the OS
    \end{itemize} 
    \item The parent process may wait for termination of a child process by using \texttt{wait()} system call. The call returns status information and the pid of the terminated process. (\texttt{pid = wait(\&status);})
    \item If no parent is waiting (did not invoke \texttt{wait()}), the child is a \textbf{zombie}
    \item If the parent is terminated without invoking \texttt{wait()}, the child process is an \textbf{orphan}
    \item Moral of the story:: Always \texttt{wait} for your child(ren)!
\end{itemize}
\newpage
\section*{Android process importance hierarchy}

\begin{itemize}
    \item Mobile OSs often have to terminate processes to reclaim system resources such as memory. They are terminated from least to most important:
    \begin{itemize}
        \item Foreground (UI, MainActivity)
        \item Visible processes (static text, images, etc.)
        \item Services
        \item Background processes
        \item Empty processes
    \end{itemize} 
    \item Start down here
\end{itemize}

\section*{Interprocess communication}

\begin{itemize}
    \item Processes within a system may be \textbf{independent} or \textbf{cooperative}
    \item Cooperating processes can affect or be affected by other processes
    \item Reasons for cooperation::
    \begin{itemize}
        \item Information sharing
        \item Computation speedup
        \item Modularity
        \item Convenience
    \end{itemize} 

    \item Cooperating processes need \textbf{interprocess communication (IPC)}
    \item Two models of IPC
    \begin{itemize}
        \item \textbf{Shared memory}
        \item \textbf{Message passing}
    \end{itemize} 
\end{itemize}

\subsection*{Producer consumer problem}

\begin{itemize}
    \item Paradigm for cooperating processes, producer process produces information that is consumed by a consumer process
\end{itemize}

\end{document}
\documentclass[12pt]{article}

\usepackage[]{amsmath}
\usepackage[]{amsthm}
\usepackage[]{amsfonts}
\usepackage[]{amssymb}
\usepackage{blindtext}
\usepackage[a4paper, total={7.5in, 10.5in}]{geometry}
\usepackage{graphicx}
\usepackage{listings}
\usepackage{color}
\usepackage{array}
\usepackage{hyperref}


\definecolor{dkgreen}{rgb}{0,0.6,0}
\definecolor{gray}{rgb}{0.5,0.5,0.5}
\definecolor{mauve}{rgb}{0.58,0,0.82}

\pagenumbering{arabic}
\let\cleardoublepage\clearpage

\lstset{frame=tb,
  language=Java,
  aboveskip=3mm,
  belowskip=3mm,
  showstringspaces=false,
  columns=flexible,
  basicstyle={\small\ttfamily},
  numbers=left,
  numberstyle=\small\color{black},
  keywordstyle=\color{blue},
  commentstyle=\color{dkgreen},
  stringstyle=\color{mauve},
  breaklines=true,
  breakatwhitespace=true,
  tabsize=4
}

\title{HIST3617 Final Exam 2024 Review}

\author{Jerry Wu}

\begin{document}
\maketitle

\section*{Course Themes}
\begin{itemize}
    \item{\textbf{Capital and Labor}} - The relationship between employers (capital) and workers (labor), often focused on issues of industrialization, working conditions, wage disparities, and labor movements like unions and strikes.
    \item{\textbf{Race relations}} - The interactions and power dynamics between different racial groups, shaped by systemic discrimination, segregation, and struggles for equality and civil rights.
    \item{\textbf{Gender relations}} - The roles, rights, and expectations associated with men and women in society, as well as the power dynamics and struggles for gender equality and feminism.
    \item{\textbf{Sexuality}} - The ways societies understand and regulate sexual behavior, identities, and norms, including struggles for LGBTQ+ rights and challenges to traditional sexual mores.
    \item{\textbf{Class}} - The divisions in society based on economic status, wealth, and access to resources, often examining inequality, social mobility, and class-based conflicts.
    \item{\textbf{Political Radicalism and Reaction}} - The tension between progressive or revolutionary movements seeking major societal changes (e.g., socialism, communism) and conservative or reactionary forces opposing them.
    \item{\textbf{Empire}} - The establishment and maintenance of colonies or territories by dominant nations, often exploring themes of exploitation, cultural exchange, and resistance by colonized peoples.
    \item{\textbf{War}} - Armed conflict between nations or groups, examining causes, consequences, and the impact on societies, economies, and cultures.
    \item{\textbf{Industrialism}} - The transformation of economies and societies through industrial production, technological innovation, and factory-based systems, shaping urbanization and labor practices.
    \item{\textbf{Immigration}} - The movement of people across borders, driven by economic, political, or social factors, and its effects on identity, labor markets, and cultural diversity.
    \item{\textbf{Urbanization}} - The growth and development of cities as populations shift from rural to urban areas, often tied to industrialization and creating new social dynamics and challenges.
    \item{\textbf{Mass culture}} - Cultural forms and practices widely disseminated and consumed by large populations, often driven by mass media, commercialization, and shared societal norms.
    \item{\textbf{Technology}} - The tools, systems, and innovations that shape human societies, influencing communication, work, and daily life, while also transforming economies and cultures over time.
\end{itemize}

\section*{Key Terms}

\subsection*{Week 1}

\begin{itemize}
    \item \textit{Old v. New Immigration} - Refers to the distinction between two waves of immigration to the United States. "Old immigration" (pre-1880s) consisted mainly of Northern and Western Europeans (e.g., Irish, Germans). "New immigration" (1880s-1920s) brought Southern and Eastern Europeans (e.g., Italians, Poles, Jews), often seen as culturally distinct and facing greater discrimination. This is further emphasized when other ethnic groups like African and East/South Asian immigrants come to the United States.
\end{itemize}

\subsection*{Week 2}

\begin{itemize}
    \item \textit{Urban Biography} - A narrative or historical account that explores the development, culture, and experiences of a particular city or urban area, often focusing on the lives of its inhabitants and social dynamics.
    
    \item \textit{Georg Simmel} - A German sociologist and philosopher (1858–1918) known for his work on social structures, urban life, and the sociology of individuals, particularly his essay "The Metropolis and Mental Life," which examines how urban environments affect human behavior and social interactions.
    
    \item \textit{How the Other Half Lives} - A groundbreaking 1890 book by Jacob Riis that used photography and journalism to expose the living conditions of the poor in New York City's tenements, sparking public awareness and reform efforts.
    
    \item \textit{Suburbs} - Residential areas located on the outskirts of cities, typically characterized by lower population density, single-family homes, and a focus on family-oriented living. Suburbs expanded significantly during the post-World War II era due to urban sprawl and the availability of automobiles.
\end{itemize}

\subsection*{Week 3}

\begin{itemize}
    \item \textit{Frontier Thesis’} - A concept introduced by historian Frederick Jackson Turner in 1893, asserting that the American frontier played a key role in shaping the nation's democracy, character, and innovation. Turner argued that the closing of the frontier marked the end of a significant era in U.S. history.
    
    \item \textit{Columbian Exposition} - Also known as the Chicago World's Fair of 1893, this event celebrated the 400th anniversary of Christopher Columbus's arrival in the Americas. It showcased technological innovations, cultural exhibits, and the grandeur of the "White City," symbolizing industrial progress and American exceptionalism.
    
    \item \textit{White Man’s Burden} - A concept derived from Rudyard Kipling’s 1899 poem, which justified imperialism by suggesting it was the moral duty of white Europeans and Americans to civilize and uplift non-European peoples, often masking exploitation and colonial domination.
    
    \item \textit{Theodore Roosevelt} - The 26th President of the United States (1901–1909), known for his progressive policies, conservation efforts, and assertive foreign policy, encapsulated in his "Big Stick Diplomacy." He was a key figure in the Progressive Era and a driving force behind the construction of the Panama Canal.
\end{itemize}

\subsection*{Week 4}

\begin{itemize}
    \item \textit{Sharecropping} - A post-Civil War agricultural system in the Southern United States where landowners allowed tenant farmers, often formerly enslaved individuals, to work their land in exchange for a share of the crops produced. This system often trapped sharecroppers in cycles of debt and poverty.
    
    \item \textit{Plessy v. Ferguson (1896)} - A landmark Supreme Court decision that upheld racial segregation under the "separate but equal" doctrine. It legitimized discriminatory Jim Crow laws and institutionalized segregation in public facilities until overturned by Brown v. Board of Education (1954).
    
    \item \textit{Lynching} - The extrajudicial killing, often by hanging, of individuals (primarily African Americans) by mobs. Lynching was a brutal tool of racial terror used predominantly in the late 19th and early 20th centuries to maintain white supremacy in the U.S.
    
    \item \textit{W.E.B DuBois} - A prominent African American scholar, civil rights activist, and co-founder of the NAACP. Du Bois advocated for the rights of Black Americans through education, activism, and his writings, including The Souls of Black Folk. He challenged ideas of racial inferiority and the accommodationist approach of Booker T. Washington.
\end{itemize}

\subsection*{Week 5}

\begin{itemize}
    \item \textit{Treaty of Versailles} - Signed in 1919, this treaty officially ended World War I. It imposed harsh penalties and reparations on Germany, redrew European borders, and established the League of Nations. Its punitive terms contributed to economic hardship and political instability in Germany, which helped pave the way for World War II.
    
    \item \textit{Red Scare} - Refers to periods of intense fear of communism and radical leftist ideologies in the United States. The first Red Scare (1919–1920) followed World War I and was fueled by labor unrest and the Russian Revolution. The second Red Scare (late 1940s–1950s) was associated with the Cold War and McCarthyism.
    
    \item \textit{Great Gatsby} - A 1925 novel by F. Scott Fitzgerald, considered a classic of American literature. It explores themes of wealth, class, and the American Dream during the Jazz Age, through the tragic story of Jay Gatsby and his pursuit of Daisy Buchanan.
    
    \item \textit{Harlem Rennissance} - A cultural, social, and artistic movement centered in Harlem, New York, during the 1920s and 1930s. It celebrated African American cultural expression through literature, music, art, and performance, featuring figures like Langston Hughes, Zora Neale Hurston, and Duke Ellington. It was pivotal in fostering racial pride and awareness.

\end{itemize}

\centerline{\textbf{READING WEEK}}
\hrulefill

\subsection*{Week 7}

\begin{itemize}
    \item\textit{Lost generation} - A term popularized by Gertrude Stein and often associated with Ernest Hemingway, referring to a group of disillusioned American writers and artists who lived in Europe after World War I. They explored themes of alienation, moral loss, and the futility of war.

    \item\textit{Gertude stein} - An American expatriate writer and art collector, Stein was a central figure in the modernist movement. She supported and mentored writers like Hemingway and Fitzgerald and coined the term "Lost Generation."
    
    \item\textit{Scopes monkey trial} - A 1925 legal case in Tennessee in which teacher John Scopes was tried for teaching evolution, violating a state law that mandated only creationism in public schools. The trial highlighted the conflict between science and religious fundamentalism.
\end{itemize}

\subsection*{Week 8}

\begin{itemize}
    \item\textit{FDR} - Franklin D. Roosevelt, the 32nd President of the United States (1933–1945), led the country through the Great Depression and World War II. His leadership and policies significantly reshaped the U.S. government’s role in society.

    \item\textit{New Deal} - A series of economic and social programs implemented by FDR during the Great Depression to provide relief, recovery, and reform. Key initiatives included Social Security, public works projects, and financial regulations.
    
    \item\textit{Herbert Hoover} - The 31st U.S. President (1929–1933), Hoover’s presidency coincided with the onset of the Great Depression. He faced criticism for his perceived inability to address the economic crisis effectively.
    
    \item\textit{Dorethea Lange} - A celebrated photographer known for documenting the human suffering of the Great Depression, particularly through her work for the Farm Security Administration. Her iconic photograph Migrant Mother remains a symbol of the era.
\end{itemize}

\subsection*{Week 9}

\begin{itemize}
    \item\textit{Adolf Hitler} - The leader of Nazi Germany from 1933 to 1945, Hitler’s fascist regime orchestrated World War II and the Holocaust, causing widespread devastation and the deaths of millions.

    \item\textit{Fascism} - A far-right, authoritarian political ideology that emphasizes nationalism, militarism, and the suppression of dissent. Fascist regimes often focus on dictatorial leadership, as seen in Nazi Germany and Mussolini's Italy.
    
    \item\textit{Spanish Civil War} - A conflict (1936–1939) between fascist forces led by Francisco Franco and a coalition of republicans, socialists, and communists. Franco’s victory established a dictatorship in Spain and served as a prelude to World War II.
    
    \item\textit{Halie Selassie} - The Emperor of Ethiopia (r. 1930–1974) and a symbol of African resistance against imperialism. He appealed to the League of Nations for support when Italy invaded Ethiopia in 1935, highlighting the failure of international intervention.
\end{itemize}

\subsection*{Week 10}

\begin{itemize}
    \item\textit{Pearl Harbor} - A US town in Hawaii with a naval base. A surprise Japanese attack was launched on December 7, 1941, targeting the U.S. naval base in Hawaii. This event led the United States to enter World War II.

    \item\textit{D Day} - The Allied invasion of Normandy, France, on June 6, 1944, marked the turning point in the Western European theater of World War II. It was a critical step in liberating Nazi-occupied Europe.
    
    \item\textit{Tuskegee Airman} - The first African American military aviators in the U.S. Armed Forces. They served with distinction during World War II, breaking barriers in a racially segregated military.
    
    \item\textit{Executive Order 9066} - FDR's executive order which ordered the internment of Japanese Americans.
    
    \item\textit{Hiroshima} - The Japanese city on which the U.S. dropped the first atomic bomb on August 6, 1945, leading to massive destruction and hastening Japan’s surrender, ending World War II.
\end{itemize}

\subsection*{Week 11}

\begin{itemize}
    \item\textit{American Occupation of Japan} - Following Japan's surrender in 1945, the United States, under General Douglas MacArthur, occupied Japan (1945–1952) to oversee its reconstruction. Reforms included democratizing Japan’s government, revamping its economy, and demilitarization.
    
    \item\textit{Shopping Malls} - Large retail complexes that became iconic in suburban America after World War II. Designed as centralized, car-friendly consumer hubs, they symbolized postwar affluence and consumerism.
    
    \item\textit{Affluent Society} - A term coined by economist John Kenneth Galbraith in his 1958 book of the same name, describing postwar America’s unprecedented wealth and consumer culture while criticizing the neglect of public goods and social inequality.
    
    \item\textit{Military-Industrial Complex} - A term popularized by President Dwight D. Eisenhower in his 1961 farewell address, warning of the powerful relationship between the military, defense contractors, and government policy, which could lead to unchecked militarization.
\end{itemize}

\subsection*{Week 12}

\begin{itemize}
    \item\textit{Robert Moses} - An influential urban planner who reshaped New York City and its surrounding areas during the 20th century. He focused on massive infrastructure projects like highways and bridges but faced criticism for prioritizing cars over communities and displacing low-income residents.
    
    \item\textit{Redlining} - A discriminatory practice in which banks and institutions denied loans or services to residents of predominantly Black or minority neighborhoods, reinforcing segregation and economic inequality.
    
    \item\textit{Housing Projects} - Government-funded urban housing developments intended to provide affordable housing for low-income families. Over time, many projects became associated with poverty and neglect due to underfunding and systemic inequality.
    
    \item\textit{Jane Jacobs} - An urban activist and writer best known for The Death and Life of Great American Cities (1961). She advocated for pedestrian-friendly cities, mixed-use neighborhoods, and community-driven urban planning, often opposing large-scale developments like those championed by Robert Moses.
\end{itemize}
    
\end{document}
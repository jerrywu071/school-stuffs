\documentclass[12pt]{book}

\usepackage[]{amsmath}
\usepackage[]{amsthm}
\usepackage[]{amsfonts}
\usepackage[]{amssymb}
\usepackage{blindtext}
\usepackage[a4paper, total={7.5in, 10.5in}]{geometry}
\usepackage{graphicx}
\usepackage{listings}
\usepackage{color}
\usepackage{array}
\usepackage{hyperref}


\definecolor{dkgreen}{rgb}{0,0.6,0}
\definecolor{gray}{rgb}{0.5,0.5,0.5}
\definecolor{mauve}{rgb}{0.58,0,0.82}

\pagenumbering{arabic}
\renewcommand{\chaptername}{Chapter}
\let\cleardoublepage\clearpage

\lstset{frame=tb,
  language=Java,
  aboveskip=3mm,
  belowskip=3mm,
  showstringspaces=false,
  columns=flexible,
  basicstyle={\small\ttfamily},
  numbers=left,
  numberstyle=\small\color{black},
  keywordstyle=\color{blue},
  commentstyle=\color{dkgreen},
  stringstyle=\color{mauve},
  breaklines=true,
  breakatwhitespace=true,
  tabsize=4
}

\title{Notes on Upton Sinclair's "The Jungle"}

\author{Jerry Wu}

\date{Sept 10 2024}

\begin{document}
\maketitle

\tableofcontents

\chapter{}

\textbf{The setting of this chapter is in early 1900s Chicago, home to the largest Lithuanian enclave in the US}

\section*{Key people}
\begin{itemize}
    \item Marija Berczynskas - Based on the Polish spelling conventions of her name and surname, she is of Lithuanian descent hailing from the former territories of the Polish-Lithuanian commonwealth in the early 1900s
    \begin{itemize}
        \item She speaks both Polish and Lithuanian, the coachman only understanding Polish
    \end{itemize}
    \item The coachman - A Polish speaking coach operator in the city
    \item Ona Lukoszaite (Lukošaite) - A young 16 year old girl who is Marija's cousin, married to Jurgis
    \item Jurig Rudkus - a large burly yet bashful man of Lithuanian descent
    \begin{itemize}
        \item his mannerisms don't seem to match his appearance
    \end{itemize}
    \item Kotrina - another family member
    \item Teta Elzbieta - Ona's stepmother
    \item Majauszkiene - Ona's grandmother
    \item Tamoszius Kuszleika - A self taught violinist/fiddler of Polish-Lithuanian descent who works in the "killing beds", a slang term for slaughterhouse
\end{itemize}

\section*{Key terms}
\begin{itemize}
    \item Eik eik - "go go" in Lithuanian
    \item Uzdaryk duris - "close the door" in Lithuanian
    \item Vynas - wine
    \item Sznapsas (šnapsas) - Lithuanian for schnapps, a german alcoholic flavoured beverage flavoured with fruit syrups
    \item Z. Graiczunas (graičiūnas) Pasilinksminimams darzas - Z. Graceful for fun (could be referring to some kind of urban leisure sphere)
    \item Darzas - garden?
    \item Viand - an archaic English word for "food item"
\end{itemize}

\end{document}
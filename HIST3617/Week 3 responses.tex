\documentclass[12pt]{article}

\usepackage[]{amsmath}
\usepackage[]{amsthm}
\usepackage[]{amsfonts}
\usepackage[]{amssymb}
\usepackage{blindtext}
\usepackage[a4paper, total={7.5in, 10.5in}]{geometry}
\usepackage{graphicx}
\usepackage{listings}
\usepackage{color}
\usepackage{array}
\usepackage{hyperref}


\definecolor{dkgreen}{rgb}{0,0.6,0}
\definecolor{gray}{rgb}{0.5,0.5,0.5}
\definecolor{mauve}{rgb}{0.58,0,0.82}

\pagenumbering{arabic}
\let\cleardoublepage\clearpage

\lstset{frame=tb,
  language=Java,
  aboveskip=3mm,
  belowskip=3mm,
  showstringspaces=false,
  columns=flexible,
  basicstyle={\small\ttfamily},
  numbers=left,
  numberstyle=\small\color{black},
  keywordstyle=\color{blue},
  commentstyle=\color{dkgreen},
  stringstyle=\color{mauve},
  breaklines=true,
  breakatwhitespace=true,
  tabsize=4
}

\title{HIST3617 Week 3 Responses: "Uncle Sam's School"}

\author{Jerry Wu (217545898)}

\begin{document}
\maketitle

\begin{itemize}
    \item[1.] \textbf{Who is the intended audience for this cartoon? What message is the audience intended to receive?}
    
    The target audience for this cartoon would likely be American citizens at the time of its creation, or the American public as a whole. The underlying message that the cartoon portrays is that the countries involved in the Spanish-American war had fallen to American expansionism and were thus subject to a re-education of sorts by the great American symbol of patriotism Uncle Sam himself.

    \item[2.] \textbf{Define and discuss the racial politics of the image and their relation to the prevailing imperialist ideology of the day}
    
    In terms of racial politics, the "students" labelled as the countries which the US intended to expand to are illustrated as having a darker complexion with an infant-like appearance. Their appearance is also borderline primitive, disshevelled, and borderline savage in comparison to the so-called regular students at the back of the room, who have the names of US states on their book covers. This would signify that those students have been re-educated and are a part of the US.

    \item[3.] \textbf{Define and discuss the key forms of symbolism present in the cartoon}
    There is quite a lot of symbolism that is used in this cartoon, such as the following:
    \begin{itemize}
        \item \textbf{Uncle Sam being a teacher} - Uncle Sam being the teacher is a good example, since he is seen as the great symbol of American patriotism and spirit, thus he is seen as an omniscient god of sorts that the American people worship. Thus making him out to be a stern and intimidating teacher for the students at the front who require a re-education contrary to the students at the back.
        \item \textbf{The desks for the states} - Contrary to the misbehaved students at the front of the room who do not have a desk, the students who have successfully been re-educated and are granted the privilege of using a desk. This symbolizes that the students who represent US states already are deemed civilized by the American public, and thus should receive more privileges.
        
    \end{itemize} 
\end{itemize}
\end{document}
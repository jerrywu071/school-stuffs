\documentclass[12pt]{book}

\usepackage[]{amsmath}
\usepackage[]{amsthm}
\usepackage[]{amsfonts}
\usepackage[]{amssymb}
\usepackage{blindtext}
\usepackage[a4paper, total={7.5in, 10.5in}]{geometry}
\usepackage{graphicx}
\usepackage{listings}
\usepackage{color}
\usepackage{array}
\usepackage{hyperref}


\definecolor{dkgreen}{rgb}{0,0.6,0}
\definecolor{gray}{rgb}{0.5,0.5,0.5}
\definecolor{mauve}{rgb}{0.58,0,0.82}

\pagenumbering{arabic}
\renewcommand{\chaptername}{Section}
\let\cleardoublepage\clearpage

\lstset{frame=tb,
  language=Java,
  aboveskip=3mm,
  belowskip=3mm,
  showstringspaces=false,
  columns=flexible,
  basicstyle={\small\ttfamily},
  numbers=left,
  numberstyle=\small\color{black},
  keywordstyle=\color{blue},
  commentstyle=\color{dkgreen},
  stringstyle=\color{mauve},
  breaklines=true,
  breakatwhitespace=true,
  tabsize=4
}

\title{HIST3617 Lecture 2}

\author{Jerry Wu}

\date{Sept 17 2024}

\begin{document}
\maketitle

\tableofcontents

\chapter{Lecture 2}

\section{Introduction to immigration: memory and reality}

\subsection{The "Browning" of America}

\begin{itemize}
  \item A term used by demographers as the introduction of non-white people in America
\end{itemize}

\subsection{Factors in growth of industrial America}
\begin{itemize}
  \item Abundant natural resources
  \item Growing supply of labour (immigration)
  \item Expanding market for manufacturing
  \item Availability of capital for investment
  \item Government investment (federal and state, financing for railways)
\end{itemize}

\subsection{Assembly lines in factories}

\begin{itemize}
  \item Meat packing was a common job in early 1900s America, which was done in an assembly line fashion
  \item Segmentation of labour, alienation from the final product
\end{itemize}

\section{Patterns of immigration, old vs new immigration}

\subsection{The primary group}
\begin{itemize}
  \item The primary group of immigrants in 19th century America are Irish people
  \begin{itemize}
    \item They were seen as "non-White"; not Anglo-Saxon, protestant, proper English speaking people, etc. Thus they experienced mass prejudice by "White" people all the way up to the 1960s because of JFK becoming president.
    \item They were often portrayed as sub human savages by "White people", which was also used to portray black people as well.
    \item Race at the turn of the century was talked about in an ethnic context like "italian race", "german race", etc. rather than what we're used to today like black, asian, white, etc.
  \end{itemize}
\end{itemize}

\subsection{The secondary group}

\begin{itemize}
  \item The secondary group of immigrants are southern italians, balkans, central europeans, russian jews, baltics, etc.
  \item Southern italy (south of Naples, Sicily, etc), was the poorest and rural part of italy. \item Italians were not seen as "white" because they didn't speak English, were roman catholic, too many children, etc. So they started "whitening" by doing the opposite of what they were used to and assimilating.
  \item Anarchism and socialism was prominent in italy at the time, so they were persecuted
  \item Most jews in tsarist russia were confined to a part of Russia called the "pale" which encompasses most of Ukraine, Moldova, Crimea, etc. This was a product of anti semitism in the state.
  \item Mass "pogroms" were organized, which were lynchings of jews in russia. The government did nothing to stop it.
  \item Most of them went to the east coast, specifically in NYC (Ellis island, repurposed as an immigrant processing center)
  \item People were checked for various physical and mental illness such as TB, schizophrenia, insanity, etc. TB especially since it was still a new disease so people were dying left and right.
  \item Many of these people were put back onto ships going back to europe. Families were broken up as a result.
  \item They were also asked if they adhered to anarchism or anarchist ideologies (buzzword for terrorism in early 1900s). 
  \item Mannhattan was the final destination for 70\% of immigrants after landing in Ellis island
  \item More remote states like North Dakota, Iowa, Minnesota, Wisconsin, etc were settled in by northern/eastern europeans for the promise of agriculture
  \item East asians mostly went to the west coast (SF, Washington, etc) to work on railways (almost exclusively male for chinese)
  \item Also included are people from the former british India and the Phillipines
  \item The chinese exclusion act of 1882; when chinese men weren't working on railways, they were washing clothes for a living purely by association to feminine work
  \item The chinese exclusion act is the first piece of legislation about immigration that specifically mentions a race of people
  \item They could not marry white women and could not bring their families from china.
\end{itemize}

\section{Push and pull factors; shifts in global and american capitalism}

\subsection{Push factors}
\begin{itemize}
  \item Economic opportunity
  \item Specific jobs
  \item Available land
  \item Familial connection
\end{itemize}

\subsection{Pull factors}

\begin{itemize}
  \item Economic hardship
  \item Lack of employment/land
  \item Persection (political or religious)
  \item Famine
\end{itemize}

\section{Immigration, war, and}

\end{document}
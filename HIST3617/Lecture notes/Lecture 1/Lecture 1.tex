\documentclass[12pt]{book}

\usepackage[]{amsmath}
\usepackage[]{amsthm}
\usepackage[]{amsfonts}
\usepackage[]{amssymb}
\usepackage{blindtext}
\usepackage[a4paper, total={7.5in, 10.5in}]{geometry}
\usepackage{graphicx}
\usepackage{listings}
\usepackage{color}
\usepackage{array}
\usepackage{hyperref}


\definecolor{dkgreen}{rgb}{0,0.6,0}
\definecolor{gray}{rgb}{0.5,0.5,0.5}
\definecolor{mauve}{rgb}{0.58,0,0.82}

\pagenumbering{arabic}
\renewcommand{\chaptername}{Section}
\let\cleardoublepage\clearpage

\lstset{frame=tb,
  language=Java,
  aboveskip=3mm,
  belowskip=3mm,
  showstringspaces=false,
  columns=flexible,
  basicstyle={\small\ttfamily},
  numbers=left,
  numberstyle=\small\color{black},
  keywordstyle=\color{blue},
  commentstyle=\color{dkgreen},
  stringstyle=\color{mauve},
  breaklines=true,
  breakatwhitespace=true,
  tabsize=4
}

\title{HIST3617 Lecture 1}

\author{Jerry Wu}

\date{Sept 10 2024}

\begin{document}
\maketitle

\tableofcontents

\chapter{Lecture 1}

\section{Intro: What is history?}

\begin{itemize}
    \item The study of change over time
    \item \textbf{Presentism} - reading the values of the present back into the past
    \item \textbf{Interpretation} - (how, what, when, why), not the same as condoning an event or an individual's actions
    \item Understanding how the holocaust, trans atlantic slavery, or indigenous depression worked is not the same as excusing or condoning them
    \item \textbf{Historiography} - How historians do history
\end{itemize}

\begin{quote}
    \textit{"The past is a foreign country, they do things differently there" - L.P Hartley}
\end{quote}

\section{Key terms L1}

\begin{itemize}
    \item \textbf{Urban biography}
    \item \textbf{Georg Simmel}
    \item \textbf{How the other half lives}
    \item \textbf{Suburbs}
\end{itemize}

\section{Urbanization}

\subsection{Urbanization in the United States}

\begin{itemize}
    \item The main key to modern history is \textbf{urbanization}; transitioning from a rural country to an urban metropolis
    \item In the early 1900s, most Americans lived in what is called \textbf{island communities}
    \begin{itemize}
        \item Mostly rural communities, homogenous demographic
        \item Most people living here are protestant christian
        \item All English speaking
        \item Conservative
        \item All deaths in population happen within 20 miles
        \item These new cities were clustered in the northeastern corner of the US
    \end{itemize} 
\end{itemize}

\subsection{Mass immigration}
Most immigration to the US came from:
\begin{itemize}
    \item Germany
    \item Ireland (due to the potato famine)
    \item Scandinavia
    \item UK
\end{itemize}

The lower east side of NYC was the most congested area in the early 1900s. It was common for 20 to 30 people to live in one apartment.\\

Other dense cities of the time inclulde:

\begin{itemize}
    \item Chicago (an outpost during the American civil war, known as the \textbf{White City})
    \item Detroit
\end{itemize}

\begin{itemize}
    \item in this context, \textbf{native $\neq$ indigenous}
\end{itemize}

\subsection{The notion of time}

\begin{quote}
    \textit{"The psychological basis of the metropolitan type of individuality consists in the intensification of nervous stimulation which results from the swift and uninterrupted change of outer and inner stimuli"} - G. Simmel 1903
\end{quote}

\begin{itemize}
    \item Capitalism leads to the standardization of time (creation of clocks and time zones rather than primitive measures like the sun or tides)
\end{itemize}

\subsection{Urban Leisure Spheres}
\begin{itemize}
    \item Urbanization brings people together to form groups like sports teams, clubs, businesses, etc.
    \item Professionalization comes to urbanization relative to capital available 
    \item Emersion of gender integrated spaces like dance halls
    \item \textit{Nickelodeon} was a term for short and cheap films watched in a theater (for a nickel)
\end{itemize}

\subsection{How the other half lives}

\begin{itemize}
    \item Immigrants were known as the "other half"
    \item More like the other three quarters
    \item In a capitalist society, the poor live in densely populated slums
    \item Race and gender mixing was common and was not seen as acceptable
\end{itemize}

\subsection{Aspects of the urban crisis}

\begin{itemize}
    \item Emergence of densely populated slums
    \item Lack of socioeconomical and transportation infrastructure
    \item Rampant poverty
\end{itemize}

\subsection{Emergence of public transit}

\begin{itemize}
    \item NYC subway opened in 1904
    \item It operated on a \textbf{flat fare system}. Pay one fare and go as far as you want. Contrary to the European system where you pay by distance (like GO transit)
    \item Lowers transportation and real estate costs.
    \item \textbf{Streetcar suburbs} existed back in the day. These were neighbourhoods with no driveways where a tram would run through.
\end{itemize}



\end{document}
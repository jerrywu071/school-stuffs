\documentclass[12pt]{book}

\usepackage[]{amsmath}
\usepackage[]{amsthm}
\usepackage[]{amsfonts}
\usepackage[]{amssymb}
\usepackage{blindtext}
\usepackage[a4paper, total={7.5in, 10.5in}]{geometry}
\usepackage{graphicx}
\usepackage{listings}
\usepackage{color}
\usepackage{array}
\usepackage{hyperref}


\definecolor{dkgreen}{rgb}{0,0.6,0}
\definecolor{gray}{rgb}{0.5,0.5,0.5}
\definecolor{mauve}{rgb}{0.58,0,0.82}

\pagenumbering{arabic}
\renewcommand{\chaptername}{Section}
\let\cleardoublepage\clearpage

\lstset{frame=tb,
  language=Java,
  aboveskip=3mm,
  belowskip=3mm,
  showstringspaces=false,
  columns=flexible,
  basicstyle={\small\ttfamily},
  numbers=left,
  numberstyle=\small\color{black},
  keywordstyle=\color{blue},
  commentstyle=\color{dkgreen},
  stringstyle=\color{mauve},
  breaklines=true,
  breakatwhitespace=true,
  tabsize=4
}

\title{HIST3617 Lecture 4}

\author{Jerry Wu}

\date{Oct 1 2024}

\begin{document}
\maketitle

\tableofcontents

\chapter{African American life in the age of Jim Crow}

\subsection{Keywords}

\begin{itemize}
    \item 
\end{itemize}

\section{An Elusive Freedom}




\section{The Strange Career of Jim Crow}

\subsection{Who is Jim Crow?}

He's a fictional character from the 1820s-1830s. He was often played by a white person in blackface or by an actual black person.

\begin{itemize}
    \item The character was meant to poke fun at black men
\end{itemize}



\section{African American life: South and North}

\begin{itemize}
    \item The term "urban" in this time period was actually used to describe an area where a lot of black people lived
    \item The blacks living in the south were subject to harsher sentences for small crimes such as stealing, leading to convict labour
    \item Conversely, the ones living in the north were of a higher socioeconomic status and were able to hold down more middle class jobs like tailors, barbers, preachers, etc.
    \item The north didn't have segregation enshrined in law unlike the south
    
\end{itemize}




\section{The Crisis of the 1890s: Disenfranchisement, Legal Segregation, Lynching, etc}

\subsection{Lynching}

\begin{itemize}
    \item Lynching is named after a judge named Lynch and it was a punishment where the convict is hung from a tree
    \item Lynching in the 1900s was extrajudicial, so it wasn't authorities carrying it out.
    \item Between 1890 to 1968, 4742 black americans were lynched
    \item Lynching didn't become a federal crime until 1968
    \item Over 200 anti lynching bills were put forth to congress from 1890 to the 1950s, but failed to pass due to \textbf{southern democrat opposition}
    \item The excuse of lynching was accusations of sexual violence against white women perpetrated by black men. Most of these cases were baseless.
    \item Ida B. Wells was an anti lynching crusader who postulated that victims of lynching were lynched because they became successful at something, rather than committing unspeakable crimes.
\end{itemize}

\end{document}
\documentclass[12pt]{book}

\usepackage[]{amsmath}
\usepackage[]{amsthm}
\usepackage[]{amsfonts}
\usepackage[]{amssymb}
\usepackage{blindtext}
\usepackage[a4paper, total={6in, 8in}]{geometry}

\usepackage{listings}
\usepackage{color}

\definecolor{dkgreen}{rgb}{0,0.6,0}
\definecolor{gray}{rgb}{0.5,0.5,0.5}
\definecolor{mauve}{rgb}{0.58,0,0.82}

\lstset{frame=tb,
  language=Python,
  aboveskip=3mm,
  belowskip=3mm,
  showstringspaces=false,
  columns=flexible,
  basicstyle={\small\ttfamily},
  numbers=left,
  numberstyle=\small\color{black},
  keywordstyle=\color{blue},
  commentstyle=\color{dkgreen},
  stringstyle=\color{mauve},
  breaklines=true,
  breakatwhitespace=true,
  tabsize=4
}

\title{EECS3101 notes:: Intractability}
\author{Jerry Wu}
\date{2023-04-04}

\begin{document}
\maketitle

\chapter*{The million dollar question}
\section*{Abstract}
Say we have a set $P$ which contains all decision based problems that can be solved in a polynomial time on a deterministic Turing machine. For example, if the shortest path between two vertices on a graph is shorter than some value $k$. This implies that we are able to write a program to solve the problem.\\ \\Then we have a set $NP$ which contains all decision based problems that are solvable in polynomial time in a non deterministic Turing machine. The whole point of this is to know when a problem \textbf{cannot} be solved efficiently (anything $\geq O(2^n)$). But is $P=NP$? If we can find a polynomial time solution for an NP complete problem, then $P=NP$. However, if we can find a problem that is in $NP$ but not $P$, then $P\neq NP$.

\begin{quote}
    \textit{"I have been wondering this for 50 years already"-Larry YL Zhang 2023}
\end{quote}

\section*{NP completeness}

There are two classes of NP complete problems::

\begin{itemize}
    \item \textbf{NP-hard}
    \begin{itemize}
        \item If it can be solved efficiently, then it can be used as a subroutine to solve any other problem in NP efficiently.
    \end{itemize} 

    \item \textbf{NP-complete}
    \begin{itemize}
        \item Problems that are both NP and NP-hard
        \item This class is the hardest class of computational problems in NP.
    \end{itemize} 
\end{itemize}

\begin{quote}
    \textit{"There are people who claim to have solved P=NP, but they are all pseudo-scientists"-Larry YL Zhang 2023}
\end{quote}

\subsection*{But what's the point?}

The reason why we want to find NP completeness is so that we can save time from finding a polynomial solution. We do this throigh proving that a polynomial time solution doesn't exist.

\begin{quote}
    \textit{"Keep trying to prove it, and you might win a million dollats"-Larry YL Zhang 2023}
\end{quote}

\section*{NP complete examples}

\subsection*{Satisfiability problem (SAT)}

Given a boolean formula written in a canonical conjunction of disjunctions form, decide if there exists a set of values of $x_1,x_2,\ldots,x_n$ that makes the expression evaluate to TRUE. A 2SAT problem (2 variables) is a P complete problem, and the 3SAT problem (3 variables) is proven to be NP complete.

\subsection*{The travelling salesman problem (TSP)}

\begin{itemize}
    \item A travelling salesperson needs to visit $n$ cities.
    \item Is there a route of at most $d$ length? (decision)
    \item Optomization problem:: find a shortest cycle visiting all vertices once in a weighted graph
\end{itemize}

TSP is proven to be NP complete, as well as the longest path in a graph.

\subsection*{CLIQUE}

A clique is a subgraph of a graph where all the vertices are adjacent to each other.

\begin{itemize}
    \item Given an undirected graph with $n$ vertices, is there a subset of size $s$ such that all pairs of vertices in the subset are adjacent to each other?
\end{itemize}
\newpage
\section*{Identifying NP completeness}

We can do this by way of reduction.

\begin{itemize}
    \item If problem $B$ can be solved by solving an instance of $A$, i.e. $A$ is harder than $B$, then we can say $B$ reduces to $A$. This is counter intuitive since $A$ is harder than $B$.
    \item If we can find an algorithm that reduces $B$ to $A$ in polynomial time, and that $B$ is NP complete, then $A$ is also NP complete.
\end{itemize}

\subsection*{Example}

\begin{itemize}
    \item SAT actually reduces to CLIQUE
    \item Given any input of SAT, we can convert it to an input of a CLIQUE problem.
    \item For an SAT formula with $k$ disjunctions, we construct a CLIQUE input that has a clique of size $K$ iff the original boolean formula is satisfiable.
\end{itemize}

We can represent each vertex of the graph as each literal that appears in any given boolean formula. Each edge will represent two vertices that are connected unless::

\begin{itemize}
    \item They come from the same disjunction
    \item They represent $X$ and $\lnot X$ for some $X$
    \item What happens if we find a clique of size 4? How does it relate to the SAT solution?
\end{itemize}

\section*{Algorithm approximations}

Approximate when finding a polynomial solution isn't realistic. It is possible to prove that the approximate solution tends towards the optimal solution.

\section*{Summary}
Today we only scratched the surface of this topic. Take 4111 or 4115 if you want to know more! (not me)

\end{document}
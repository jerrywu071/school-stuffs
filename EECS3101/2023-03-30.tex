\documentclass[12pt]{book}

\usepackage[]{amsmath}
\usepackage[]{amsthm}
\usepackage[]{amsfonts}
\usepackage[]{amssymb}
\usepackage{blindtext}
\usepackage[a4paper, total={6in, 8in}]{geometry}

\usepackage{listings}
\usepackage{color}

\definecolor{dkgreen}{rgb}{0,0.6,0}
\definecolor{gray}{rgb}{0.5,0.5,0.5}
\definecolor{mauve}{rgb}{0.58,0,0.82}

\lstset{frame=tb,
  language=Python,
  aboveskip=3mm,
  belowskip=3mm,
  showstringspaces=false,
  columns=flexible,
  basicstyle={\small\ttfamily},
  numbers=left,
  numberstyle=\small\color{black},
  keywordstyle=\color{blue},
  commentstyle=\color{dkgreen},
  stringstyle=\color{mauve},
  breaklines=true,
  breakatwhitespace=true,
  tabsize=4
}

\title{EECS3101 notes:: More on graph algorithms}
\author{Jerry Wu}
\date{2023-03-30}

\begin{document}
\maketitle
\chapter*{Shortest path algorithms}

\section*{Abstract}
A shortest path is simply the path of minimum weight on a weighted graph. Applications of this idea include, but are not limited to::
\begin{itemize}
    \item Network routing
    \item Map/route generation in traffic
    \item Robotic motion
\end{itemize}
\section*{Types of shortest path problems}
There are 3 main types of shortest path problems in graph theory.
\subsection*{Single source, all destinations shortest paths}
\begin{itemize}
    \item BFS on an unweighted graph.
    \item Dijkstras(LAN)/Bellman-Ford(WAN) for weighted graphs
\end{itemize}
\subsection*{Single source, single destination}
\begin{itemize}
    \item All known algorithms for this kind of problem have the same worst case running time as single source, all destinations algorithms. In other words, just find all of them!
\end{itemize}
\subsection*{All pair shortest paths}
\begin{itemize}
    \item A naive approach would be to run single source shortest path algorithm $\forall v\in V$
    \item FLoyd-Warshall algorithm
    \item Johnson's algorithm
\end{itemize}

\section*{Properties of shortest path problems}

\begin{quote}
    \textit{"How many times did we use proof by contradiction? I don't know, 3 times a week?"-Larry YL Zhang 2023}
\end{quote}
\subsection*{Optimal substructure}
\begin{itemize}
    \item Sub paths of shortest paths are also shortest paths.
    \item Proof by cut and paste:: Assume $(x,y)$ is not a shortest path, then $\exists$ another $(\psi, \phi)$ that is even shorter $\implies$ contradiction.
\end{itemize}

\subsection*{Relaxation}
\begin{itemize}
    \item $\forall v\in V$, we maintain $d[v]$, the estimate of the shortest path from the source node $s$ initialized to $\infty$ at the start.
    \item Relaxing an edge $(u,v)$ means testing whether we can improve the shortest path to $v$ found so far by going through $u$.
    
    \begin{lstlisting}
Relax(u,v,w):
if d[v]>d[u]+w(u,v) then:
    d[v]=d[u]+w(u,v)
    pi[v]=u
    \end{lstlisting}
\end{itemize}

\begin{quote}
    \textit{"Today's lecture is going to be very relaxing. Today is gonna be a medication session"-Larry YL Zhang}
\end{quote}

\section*{Dijkstra's algorithm}

\begin{itemize}
    \item Non negative edge weights, i.e. $\forall \epsilon \in E, w(\epsilon)>0$
    \item It's like BFS but uses a priority queue (similar to Prim's algorithm). Key for the min priority queue is $d[v]$.
\end{itemize}

\subsection*{The main idea}
\begin{itemize}
    \item Maintain a set $S$ containing the solved vertices
    \item At each step,
    \begin{itemize}
        \item Select the closest vertex $u$.
        \item add it to $S$.
        \item relax all edges from $u$.
    \end{itemize}
    \item Repeat until the queue is empty.
\end{itemize}

\begin{lstlisting}
Dijkstra(G,w,s):
    for v in V:
        d[v]=infinity
    d[s]=0
    S=null
    Q=V
    while Q != null:
        u=ExtractMin(Q)
        S=union(S, [u])
        for(v in Adj[u]):
            if(d[v]>d[u]+w(u,v)):
                d[v]=d[u]+w(u,v)
\end{lstlisting}

\subsection*{Proof of correctness}

\begin{itemize}
    \item Loop invariant
    \begin{itemize}
        \item A the end of each iteration, $\forall v\in S, d[v]$ is the real shortest path distance from the source vertex to $v$.
    \end{itemize}

    \item The proof is trivial and is left as an exercise to the reader (24.3). In summary, it is impossible to find a path shorter than the shortest path.
    
    \begin{align*}
        d[y]=\delta(s,y)\leq \delta(s,u)\leq d[u]\implies d[u]\leq d[y]\\
        \implies d[u]=d[y]=\delta(s,y)=\delta(s,u)\blacksquare
    \end{align*}

    Why is $d[u]\leq d[y]$? This is because $u$ is selected by $ExtractMin(Q)$, so it will always be the lesser value for $d[v]$. So it is the thing we will be adding to $S$.
\end{itemize}

\subsection*{Runtime of Dijkstra's}

\begin{itemize}
    \item We execute $ExtractMin$ exactly $|V|$ times.
    \item We execute $DecreaseKey$ exactly $|E|$ times.
    \item Time complexity depends on the priority queue implementation. With a min heap priority queue, we have that the time complexity is::
    \begin{align*}
        |V|\log(|V|)+|E|\log(|V|)\Theta(|E|\log(|V|)
    \end{align*}
\end{itemize}

\section*{Bellman-Ford algorithm}
Dijkstra's algorithm doesn't allow for \textbf{negative} weighted edges, so we utilize BF instead.

\begin{itemize}
    \item BF uses dynamic programming (they're actually the people who invented dyanmic programming!)
    \item It can detect negative cycles (returns false because the "shortest" path distance will be $-\infty$)
    \item returns shortest path tree otherwise
\end{itemize}

\begin{lstlisting}
BellmanFord(G,w,s):
    for v in V:
        d[v]=infinity
    d[s]=0
    pi[s]=null
    for i in range |V|-1:
        for (u,v) in E:
            Relax(u,v,w)
    for (u,v) in E:
        if d[v]>d[u]+w(u,v):
            return false
    return true
\end{lstlisting}

Runtime is $\Theta(|V|\times|E|)$

\subsection*{Loop invariant of BF}

\begin{itemize}
    \item At the end of the $k$-th iteration, $d[v]$ is the shortest path distance over all paths that contain at most $k$ edges (24.1).
\end{itemize}

\section*{Shortest paths in directed acyclic graphs (DAGs)}

We can perform the following operations in sequence::

\begin{itemize}
    \item Topological sort the graph
    \item Relax all nodes in the topologically sorted order
    \item We only do one round of relax, whereas BF does $|V|-1$ rounds.
\end{itemize}

Running time is $O(|E|)$.

\end{document}
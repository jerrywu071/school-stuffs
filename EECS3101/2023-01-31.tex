\documentclass[12pt]{article}

\usepackage[]{amsmath}
\usepackage[]{amsthm}
\usepackage[]{amsfonts}
\usepackage[]{amssymb}
\usepackage{blindtext}
\usepackage{algpseudocode}
\usepackage[a4paper, total={6in, 8in}]{geometry}

\title{EECS3101 notes}
\author{Jerry Wu}
\date{2023-01-31}

\begin{document}
    \maketitle

    \subsection*{Example 1}
    Cauculate big O and big $\Omega$.
    \begin{align*}
        T(n)=T(\frac{n}{3})+T(\frac{2n}{3})+n
    \end{align*}
    We can combine both $T$ functions into one by taking the bigger one, i.e. $T(\frac{2n}{3})$. So for $O$,
    \begin{align*}
        (T(n)=T(\frac{n}{3})+T(\frac{2n}{3})+n)\leq (2T(\frac{2n}{3})+n)\in \Theta(n^{1.7})=O(n^{1.7})
    \end{align*}
    For $\Omega$, use the smaller one; $T(\frac{n}{3})$

    \begin{align*}
        T(n)=T(\frac{n}{3})+T(\frac{2n}{3})+n\geq 2T(\frac{n}{3})+n=\Theta(n)=\Omega(n)
    \end{align*}

    \subsection*{Example 1 with recursion tree (slide 5)}
    The end result is $O(nlog(n))$ and $\Omega(nlog(n))$. Because it is lower bounded and upper bound by the same class ($nlog(n)$), We can conclude that the runtime of the function is $\Theta(nlog(n))$.

    \subsection*{Exercise 4.3 (important, slide 6)}
    We want to design an algorithm that selects the $k$-th smallest element in a list using a pivot helper function. We can use divide and conquer to design this solution.
    
    \begin{quote}
        \textit{"This is the P O W E R of recursion and mathematical induction!"-Larry YL Zhang 2023}
    \end{quote}

    We always want to partition into evenly sized partitions to split the problem efficiently. $\Theta(n^2)$ is no bueno.

    \subsection*{Exercise 4.4: Search a matrix}
    \begin{quote}
        \textit{"This shows the P O W E R of divide \& conquer and master theorem"-Larry YL Zhang 2023}
    \end{quote}

    We want to search for whether an element $x$ exists in a 2D array of $n$ integers ($\sqrt{n}\times \sqrt{n})$(true/false).\\

    Assume the following are true::
    \begin{itemize}
        \item Each row is sorted in ascending order.
        \item Each column is also sorted in ascending order.
    \end{itemize}

    We want the algorithm to be faster than $\Theta(n)$\\
    Divide into 2 subproblems\\
    Define the runtime as $T(n)=2T(\frac{n}{2})+\sqrt{n}\implies \Theta(n)$ \textbf{NOT GOOD ENOUGH!}\\
    If we use the middle element as a pivot, we can skip one of the quadrants completely.
\end{document}
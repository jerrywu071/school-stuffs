\documentclass[12pt]{article}

\usepackage[]{amsmath}
\usepackage[]{amsthm}
\usepackage[]{amsfonts}
\usepackage[]{amssymb}
\usepackage{blindtext}
\usepackage[a4paper, total={6in, 8in}]{geometry}

\title{EECS3101 notes (Divide and Conquer, recurrence, master theorem, etc.)}
\author{Jerry Wu}
\date{Thursday, Jan 26, 2023}

\begin{document}

\maketitle

\subsection*{Structure of D and C}
\begin{itemize}
    \item Divide - Divide the problem into smaller sub problems
    \item Conquer - If the sub problem is small enough, return the solution directly. Else, use recursion.
    \item Combine - Put it all together into the overall solution.
\end{itemize}

\subsection*{Example :: Segment sum (code on slide 13)}
We can split the array into 2 smaller sub arrays for a start. The problem will be considered small enough once each sub array has one element, i.e. $low==high$, then we return each one individually. If the value is negative, default to 0 since the empty set is also a subset of every set (segment sum of an empty array is always 0).

\subsection*{Recursive analysis of example}

Let $T(n)$ be the worst case runtime of the algorithm. The left and right segments are both $T(\frac{n}{2})$. So it is $T(n)$ overall. When analyzing recursive functions, we always assume the base case is true.

\subsection*{The master theorem (formal definition on slide 19)}
The function of the following form has a runtime determinable by the master theorem.
\begin{align*}
    T(n)=aT(\frac{n}{b})+f(n)
\end{align*}

\begin{itemize}
    \item $a$ is the number of recursive calls.
    \item $b$ is the rate at which subproblem size decreases.
    \item $f(n)$ is the runtime of the non recursive portion of the algorithm.

\end{itemize}

The master theorem has the following cases::

\begin{itemize}
    \item[\textbf{Case 1::}] $f(n)=O(n^{\log_b(a)-\epsilon}), \epsilon > 0: T(n)=\Theta(n^{\log_b(a)})$
    \item[\textbf{Case 2::}] $f(n)=\Theta(n^{\log_b(a)}): T(n)=\Theta(n^{\log_b(a)}\log(n))$
    \item[\textbf{Case 3::}] $f(n)=\Omega(n^{\log_b(a)+\epsilon}), \epsilon > 0: T(n)=\Theta(f(n))$
\end{itemize}

\textbf{According to MT, the segment sum discussed earlier is $\Theta(nlog(n))$!}

\begin{quote}
    \textit{"I don't care about log. I'm not a math prof."-Larry YL Zhang 2023}
\end{quote}
\begin{quote}
    \textit{"Master theorem is P O W E R"-Larry YL Zhang 2023}
\end{quote}
\begin{quote}
    \textit{"log is very important"-Larry YL hang 2023}
\end{quote}

\subsection*{Recursion trees (slide 32)}
When MT isn't appplicable, we can use the general form::
\begin{align*}
    T(n)=\sum_{j=0}^{\log_b(n)-1}a^j f(\frac{n}{b^j})+\Theta(n^{\log_b(a)})
\end{align*}

\begin{quote}
    \textit{"Try doing this proof yourself. It's good practice."-Larry YL Zhang 2023}
\end{quote}

\end{document}
\documentclass[12pt]{book}

\usepackage[]{amsmath}
\usepackage[]{amsthm}
\usepackage[]{amsfonts}
\usepackage[]{amssymb}
\usepackage{blindtext}
\usepackage{pgfplots}
\usepackage[a4paper, total={6in, 8in}]{geometry}

\title{Chapter 6::Chemical equilibrium}

\author{Jerry Wu}

\date{\today}

\begin{document}
\maketitle

\chapter*{Gibbs energy}
\subsection*{Abstract}
We would like to determine the spontaneity of a reaction mixture to approach equilibrium, along with deriving the thermodynamic equlibrium constant $K_p$ and equilibrium concentraations of reactants and products in a mixture of reactive ideal gases. We define Gibbs energy ($\Delta G$), which expresses spontaneity in terms of properties of the system alone.\\

Suppose we have an arbitrary system $\sigma$ in thermal equilibrium with the surroundings. What determines the directions of a spontaneous change? Well, it would be the tendency of a system to minimize energy.

\begin{align*}
    -dq_{\sigma}=dq_{surr}\implies dS_{\Omega}+dS_{surr}\geq 0\implies dS_{\sigma}+\frac{dq_{surr}}{T}\geq 0
\end{align*}

\subsection*{Definition}
\begin{align*}
    \Delta G_{\sigma}=\Delta H_{\sigma}-T\Delta S_{\sigma}\implies dG_{\sigma}=dH_{\sigma}-TdS_{\sigma}
\end{align*}

If we graph this relation, we notice that $\Delta G$ is a function of $T$. $\Delta H$ would be $\Delta G_0$ and the slope would be $\Delta S$.

Gibbs energy holds the following properties::

\begin{itemize}
    \item $\Delta G<0\implies \sigma\equiv spont\implies exergonic$
    \item $\Delta G>0\implies \sigma\equiv \lnot spont\implies endergonic$
    \item Gibbs energy is an extensive property, and is a state function.
\end{itemize}

When a problem concerns Gibbs energy, we only care about \textbf{what happens in the system, not the surroundings}.

\subsection*{Differential forms of $U$, $H$, and $G$}
For an infinitessimal process::
\begin{align*}
    G=H+TS\implies dG=dH-TdS-SdT
\end{align*}

According to the first law::
\begin{align*}
    dH=dU+PdV+VdP\implies dU=dq+dw\implies dU=dq-PdV
\end{align*}

For a reversible process::
\begin{align*}
    dq_{rev}=TdS, dU=TdS-PdV
\end{align*}

So,

\begin{align*}
    dH=TdS+VdP,dG=VdP-SdT, dU=TdS-PdV
\end{align*}

This only applies for \textbf{expansion} work.

\end{document}
\documentclass[12pt]{book}

\usepackage[]{amsmath}
\usepackage[]{amsthm}
\usepackage[]{amsfonts}
\usepackage[]{amssymb}
\usepackage{blindtext}
\usepackage[a4paper, total={7.5in, 10.5in}]{geometry}
\usepackage{graphicx}
\usepackage{listings}
\usepackage{color}
\usepackage{array}
\usepackage{hyperref}


\definecolor{dkgreen}{rgb}{0,0.6,0}
\definecolor{gray}{rgb}{0.5,0.5,0.5}
\definecolor{mauve}{rgb}{0.58,0,0.82}

\pagenumbering{arabic}
\renewcommand{\chaptername}{Section}
\let\cleardoublepage\clearpage

\lstset{frame=tb,
  language=Java,
  aboveskip=3mm,
  belowskip=3mm,
  showstringspaces=false,
  columns=flexible,
  basicstyle={\small\ttfamily},
  numbers=left,
  numberstyle=\small\color{black},
  keywordstyle=\color{blue},
  commentstyle=\color{dkgreen},
  stringstyle=\color{mauve},
  breaklines=true,
  breakatwhitespace=true,
  tabsize=4
}

\title{HIST3801 Lecture 2}

\author{Jerry Wu}

\date{Sept 16 2024}

\begin{document}
\maketitle

\tableofcontents

\chapter{Lecture 2: Video games as a business}

\section{History of video game companies}

\subsection{The big 3}

These companies take up about $\frac{1}{3}$ of the industry. These companies being:

\begin{itemize}
    \item Nintendo
    \item Sony
    \item Microsoft
\end{itemize}

\subsection{Dead companies}

Silicon knights was a Canadian company based in St. Catherines. They have effectively been dead for the past decade or so.

\begin{itemize}
    \item They were reliant on larger companies since they were a smaller studio. They relied on the unreal engine by Epic Games which lead to a lawsuit by silicon knights and subsequently their downfall.
    \item Silicon knights actually took the unreal engine and modified it and made it their own engine, not paying royalties.
\end{itemize}

\subsection{Atari}

\begin{itemize}
    \item Launched in 1977 and was a huge success, specifically the Atari 2600
    \item Killed off by a market crash in the mid 80s
    \item The games were pretty bad and were made cheaply
\end{itemize}

\subsection{Nintendo}

\begin{itemize}
    \item Founded in 1889 as a toy company, specifically playing cards (hanafuda cards)
    \item Something called world war 2 happened in the 1930s and 40s. Japan loses after america bombed them twice and america rebuilds japan in their image by investing in their restoration
    \item Nintendo tried doing everything from catering, more toys, taxi company (killed by labour strike), even love hotels (allegedly)!
    \item Eventually got into video games in the 80s, commanding 83\% market share in NA and 90\% in JP
    \item The NES was relatively dirt cheap at \$100 in 1988 (\$266 in 2024)
    \item Gunpei Yokoi wanted to make a handheld version of the NES, so he concieved the gameboy shortly after in 1989 (it was \$10 cheaper than the NES) but died in 1997 before the full success of his invention was realized
    \item Gameboy games were dirt cheap at \$20 unlike NES games which were around \$60 at the time
\end{itemize}

\subsection{Sega}

\begin{itemize}
    \item Founded in Hawaii as an american company but moved to japan, merging with competitor Rosen Enterprise
    \item Marketed as an entertainment and gambling platform for soldiers in the US army
    \item US states begin cracking down on gambling which is why they moved to japan, where a lot of american soldiers are residing.
    \item Renamed themselves to "Service and Games" aka SEGA
    \item First console launched in 1983 but got destroyed by Nintendo's NES
    \item Sega needed market share elsewhere, so they expanded to EU and SA, becoming successful enough to expand back into the US with the Genesis
\end{itemize}

\subsection{Sony}

\begin{itemize}
    \item Founded by Akio Morita and Masaru Ibuka in 1946 as a music player and TV company
    \item Name derived from the latin word "sonus" which means sonic and sound, and "sonny" which was a slang word for boy in the US
    \item Got into video games in the early to mid 90s with the rise of CD technologies rather than cartridges along with primative 3D graphics
    \item The first playstation was announced in 1991, releasing in 1994 in japan, and september 1995 in NA. It was around \$300 which was competitve at the time and the main gimmick was 3D games
    \item They don't really compete with nintendo because of the N64 (neither does the dreamcast OMEGALUL)
\end{itemize}

\subsection{Microsoft}

Most companies discussed so far are prodominantly japanese. However

\begin{itemize}
    \item Microsoft released the original xbox in november of 2001 which was also \$300. They spent around \$500 million on the launch campaign.
    \item Because Microsoft is a computer company, the xbox was just a computer that looked like a console, so manufacturing it was expensive and was being sold at a market loss
    \item Microsoft mostly focused on sports games, shooters, M-rated titles, etc. and appealed to an older audience
    \item The xbox was very successful in NA
\end{itemize}


\section{Canada's game industry}

\begin{itemize}
    \item Canada ranks 3rd in the number of employees working in the game industry
    \item 75\% of all game companies in canada are canadian owned, but only 17\% of all employment in said industry are in canadian owned companies. This is due to multinational companies like ubisoft.
\end{itemize}

\section{The business model}



\end{document}
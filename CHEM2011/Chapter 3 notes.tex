\documentclass[12pt]{book}

\usepackage[]{amsmath}
\usepackage[]{amsthm}
\usepackage[]{amsfonts}
\usepackage[]{amssymb}
\usepackage{blindtext}
\usepackage{pgfplots}
\usepackage[a4paper, total={6in, 8in}]{geometry}

\title{Chapter 3:: Mathematical properties of state functions}


\author{Jerry Wu (217545898)}

\date{\today}
\begin{document}
\maketitle
\chapter*{Mathematical properties of state functions}
\rule{\textwidth}{0.4pt}

\subsection*{Motivation}
We want to be able to express state functions as variables and partial derivatives. From there we can model stuff like internal energy and enthalpy in terms of $P$, $V$, and $P$ interchangably. We can also do the following::

\begin{itemize}
    \item Quantify the difference between $C_P$ and $C_V$
    \item Understand the Joule-Thomson experiment.
\end{itemize}

Consider $1mol$ of an ideal gas::
\begin{align*}
    P=f(V,T)=\frac{RT}{V}
\end{align*}

When we assume $T\equiv const$,
\begin{align*}
    \frac{\partial P}{\partial V}_T=\lim_{\Delta V\rightarrow 0}\frac{P(V+\Delta V,T)-P(V,T)}{\Delta V}=-\frac{RT}{V^2}
\end{align*}

When we assume $V\equiv const$,

\begin{align*}
    \frac{\partial P}{\partial T}_V=\lim_{\Delta V\rightarrow 0}\frac{P(V+\Delta V,T)-P(V,T)}{\Delta V}=\frac{R}{V}
\end{align*}

So the total change in pressure ($dP$) can be expressed as::

\begin{align*}
    dP=\frac{\partial P}{\partial T}_V dT+\frac{\partial P}{\partial V}_T dV
\end{align*}

In general, 

\begin{align*}
    dz=\frac{\partial z}{\partial x}_y dx + \frac{\partial z}{\partial y}_x dy
\end{align*}

by principle of vector addition.

\subsection*{The cyclic rule}

\begin{align*}
    \frac{\partial P}{\partial V}_T\frac{\partial V}{\partial T}_P\frac{\partial T}{\partial P}_V=-1\implies \frac{\partial P}{\partial T}_v=-\frac{\partial P}{\partial V}_T\frac{\partial V}{\partial T}_P=-\frac{\frac{\partial V}{\partial T}_P}{\frac{\partial V}{\partial P}}_T=\frac{\alpha}{\kappa_T}
\end{align*}

The expansion coefficient for isobaric volumetric thermal is::
\begin{align*}
    \alpha=\frac{1}{V}\frac{\partial V}{\partial T}_P
\end{align*}

Isothermal compressibility::

\begin{align*}
    \kappa_T=-\frac{1}{V}\frac{\partial V}{\partial P}_T
\end{align*}

So we have that

\begin{align*}
    dP=\frac{\alpha}{\kappa_T}dT-\frac{1}{\kappa_T V}dV
\end{align*}

and the total change is expressed as::

\begin{align*}
    \Delta P=\int_{T_i}^{T_f}\frac{\alpha}{\kappa_T}dT-\int_{V_i}^{V_f}\frac{1}{\kappa_T V}dV\approx \frac{\alpha}{\kappa_T}\Delta T-\frac{1}{\kappa_T}\ln(\frac{V_f}{V_i})
\end{align*}

\subsection*{Example 1}

You have accidentally arrived at the end of the range of an ethanol-in-glass
thermometer so that the entire volume of the glass capillary is filled. By how much
will the pressure in the capillary increase if the temperature is increased by another $10.0^{\circ}C$? $\alpha_{glass}=2E-5(^{\circ}C)^{-1}$,$\alpha_{ethanol}=11.2E-4(^{\circ}C)^{-1}$, $\kappa_{T,ethanol}=11.0E-5(bar)^{-1}$. Will the thermometer survive the experiment?

\begin{align*}
    \Delta P=\int_{T_i}^{T_f}\frac{\alpha_{ethanol}}{\kappa_{T,ethanol}}dT-\int_{V_i}^{V_f}\frac{1}{\kappa_{T,ethanol}V}dV\approx \frac{\alpha_{ethanol}}{\kappa_{T,ethanol}}(T_f-T_i)-\frac{1}{\kappa_{T,ethanol}}\ln(\frac{V_f}{V_i})
\end{align*}

So $\Delta P$ becomes::

\begin{align*}
    \Delta P=\frac{\alpha_{ethanol}}{\kappa_{T,ethanol}}\ln(\frac{V_i(1+\alpha_{glass}\Delta T)}{V_i})=100bar
\end{align*}

So no, the glass will not survive the experiment.

\subsection*{Dependence of $U$ on $V$ and $T$}

Internal energy on a fixed amount of substance is a function of $P$, $V$, and $T$.

\begin{align*}
    U=f(P_V)=f(V,T)=f(T,P)
\end{align*}

We are most interested in the one with volume and temperature.

\begin{align*}
    dU=\frac{\partial U}{\partial T}_V dT+\frac{\partial U}{\partial V}_T dV
\end{align*}

So how can numerical values for $\frac{\partial U}{\partial T}_V$ and $\frac{\partial U}{\partial V}_T$ be obtained?

\begin{align*}
    dq-P_{external}dV=\frac{\partial U}{\partial T}_V dT+\frac{\partial U}{\partial V}_T dV
\end{align*}

For an isochoric process, define the following::

\begin{align*}
    dq_V=\frac{\partial U}{\partial T}_V dT
\end{align*}
\begin{align*}
    \frac{dq_V}{dT}=\frac{\partial U}{\partial T}_V=C_V
\end{align*}

So now we can integrate for total change in internal energy.

\begin{align*}
    \Delta U_V=\int_{T_1}^{T_2}C_V dT=n\int_{T_1}^{T_2}C_{V,m}dT
\end{align*}

If $C_{V,m}$ is regarded as constant (small temperature range), we have that::

\begin{align*}
    \Delta U_V=\int_{T_1}^{T_2}C_V dT=C_V\Delta T=nC_{V,m}\Delta T
\end{align*}

\begin{align*}
    \int_{i}^{f}dq_V=\int_{i}^{f}\frac{\partial U}{\partial T}_V dT or q_V=\Delta U
\end{align*}

\subsection*{Internal pressure}

$\frac{\partial U}{\partial V}_T$ has units of $\frac{J}{m^3}$, which is internal pressure.

\begin{align*}
    \frac{\partial U}{\partial V}_T=T\frac{\partial P}{\partial T}_V-P
\end{align*}

Our total infinitessimal change in $U$ can be expressed as the following::

\begin{align*}
    dU=dU_V+dU_T=C_V dT+\left(T\frac{\partial P}{\partial T}_V-P\right) dV
\end{align*}

For $T\equiv const$

\begin{align*}
    dU=dU_T=\left(T\frac{\partial P}{\partial T}_V-P\right)dV
\end{align*}

For $V\equiv const$

\begin{align*}
    dU=dU_V=C_V dT
\end{align*}

So this means that $\Sigma dU=dU_V+dU_T$ by virtue of common sense not so common.

\subsection*{Example 2}

Evaluate $\frac{\partial U}{\partial V}_T$ for an ideal gas.

\begin{align*}
    \frac{\partial U}{\partial V}_T=T\frac{\partial P}{\partial T}_V-P=T\left(\frac{\partial \left( \frac{nRT}{V}\right)}{\partial T}\right)_V-P=\frac{nRT}{V}-P=0
\end{align*}

This is unsurprising, since ideal gases' $\Delta U$ don't depend on $\Delta V$. So we can conclude that $U$ is only a function of $T$ for ideal gases.\\We can observe that

\begin{align*}
    \Delta U_T=\int_{V_i}^{V_f}\frac{\partial U}{\partial V}_T dV\approx 0
\end{align*}

to a good approximation for $g\in \mathbb{R}$ under most conditions.\\ So what is the magnitude of this for processes involving liquids and solids?

\begin{align*}
    \Delta U_T^{solid,liquid}=\int_{V_1}^{V_2}\frac{\partial U}{\partial V}_T dV\approx \frac{\partial U}{\partial V}_T\Delta V\approx 0
\end{align*}

Under most conditions encountered in lab, $U$ can be regarded as a function of $T$ alone $\forall$ substances.

\begin{align*}
    U(T_f,V_f)-U(T_i,V_i)=\Delta U=\int_{T_i}^{T_f}C_V dT=n\int_{T_i}^{T_f}C_{V,m}dT
\end{align*}

\subsection*{Enthalpy as a state function}

$H$ is a function of $P$,$V$, and $T$.

\begin{align*}
    H=f(P,V)=f(V,T)=f(T,P)
\end{align*}

We can make $P\equiv const$.

After some work, we have that

\begin{align*}
    \int_{i}^{f}dU=\int_{i}^{f}dq_P-\int_{i}^{f}PdV
\end{align*}

\begin{align*}
    U_f-U_i=q_P-P(V_f-V_i)
\end{align*}

\subsection*{Phase chages}

For both fusion and vaporization (solid $\rightarrow$ liquid and liquid $\rightarrow$ gas), $q>0\implies C_P\rightarrow \infty$

\subsection*{Derivation}

Following our derivation of $U$ and its differentials, we can do the same for $H$.

\begin{align*}
    dH=\frac{\partial H}{\partial T}_P dT+\frac{\partial H}{\partial P}_T dP
\end{align*}

Because $dP=0$ at $P\equiv const$ and $dH=dq_P$, we have that

\begin{align*}
    dq_P=\frac{\partial H}{\partial T}_P dT
\end{align*}

and the heat capacity at $P\equiv const$ is defined as::

\begin{align*}
    C_P=\frac{dq_P}{dT}=\frac{\partial H}{\partial T}_P
\end{align*}

$C_P$ is determined by measuring the heat flow $dq$ at constant $P$ with the resulting temperature
change $dT$ in the limit in which $dT$ and $dq$ approach zero:

\begin{align*}
    C_P=\lim_{dT\rightarrow 0}\frac{dq}{dT}_P
\end{align*}

For a process where $P\equiv const$ assuming no phase changes or reactions, we have that

\begin{align*}
    \Delta H_P=\int_{T_i}^{T_f}C_P(T)dT=n\int_{T_i}^{T_f}C_{P,m}(T)dT
\end{align*}

\begin{align*}
    \Delta H_P=C_P\Delta T=nC_{P,m}\Delta T
\end{align*}

\subsection*{Example 3 (full solution on slide 30)}

A $143.0g$ sample of $C(s)$ in the form of graphite is heated from $300K \rightarrow 600K$ at a
constant pressure. Over this temperature range, $C_{P,m}$ has been determined to be

\begin{align*}
    \frac{C_{P,m}}{JK^{-1}mol^{-1}}=-12.19+0.1126\frac{T}{K}-(1.947E-4)\frac{T^2}{K^2}+(1.19E-7)\frac{T^3}{K^3}-(7.800E-11)\frac{T^4}{K^4}
\end{align*}

\begin{align*}
    \Delta H=\frac{m}{M}\int_{T_i}^{T_f}C_{P,m}(T)dT
\end{align*}

If $C_{P,m}$ at 300K is used instead, after some work, we have that $\Delta H=30.8kJ$

\subsection*{Variations of enthalpy with pressure at $T\equiv const$}

\begin{align*}
    \Delta H=\int_{T_i}^{T_f}C_P(T)dT=n\int_{T_i}^{T_f}C_{P,m}(T)dT
\end{align*}

Because H is only a function of T, the equation above holds $\forall g\in \mathbb{C}$ even if $\lnot P\equiv const$. For an ideal gas, we know that

\begin{align*}
    \left(\frac{\partial H}{\partial P}\right)_T=0
\end{align*}

\begin{align*}
    \left(\frac{\partial H}{\partial P}\right)_T=V-T\left(\frac{\partial V}{\partial T}\right)_P\approx V(1-T_\alpha)
\end{align*}

For systems that consist of liquids and solids, $T_\alpha<<1 \forall T<1000K$ and $\left(\frac{\partial H}{\partial P}\right)_T\approx V$ and $dH\approx C_P dt+Vdp$.


\subsection*{Joule-Thomson experiment (slide 34)}

\begin{align*}
    dU=dU_V+dU_T=C_V dt+\left[T\left(\frac{\partial P}{\partial T}\right)_V-P\right]dV
\end{align*}

Assume $N_2$ is used in the expansion process ($P_1>P_2$), it is found that $T_2<T_1$; the gas is cooled as it expands. Why does this happen?

\begin{align*}
    w=w_{left}+w_{right}=-\int_{V_1}^{0}P_1dV-\int_{0}^{V_2}P_2dV=P_1V_1-P_2V_2
\end{align*}

Because the pipe is insulating, $q=0$ and

\begin{align*}
    \Delta U=w=P_1V_1-P_2V_2\implies U_2+P_2V_2=U_1+P_1V_1\implies U_2+P_2V_2=U_1+P_1V_1\implies H_2=H_1
\end{align*}

So this means that the system is \textbf{isenthalpic} (no enthalpy change)

\subsection*{The Joule-Thomson coefficient}
The constant can be expressed in the formula
\begin{align*}
    \mu_{J-T}=\lim_{\Delta P\rightarrow 0} \left(\frac{\Delta T}{\Delta P}\right)_H=\left(\frac{\partial T}{\partial P}\right)_H
\end{align*}

\begin{align*}
    dH=C_P dT+\left(\frac{\partial H}{\partial P}\right)_T dP\implies C_P\left(\frac{\partial T}{\partial P}\right)_H+\left(\frac{\partial H}{\partial P}\right)_T=0
\end{align*}

So then,

\begin{align*}
    \left(\frac{\partial H}{\partial P}\right)_T=-C_P\mu_{J-T}
\end{align*}

\subsection*{Example}

Using the equation::

\begin{align*}
    \left(\frac{\partial H}{\partial P}\right)_T=\left[\left(\frac{\partial U}{\partial V}\right)_T+P\right]\left(\frac{\partial V}{\partial P}\right)_T+V
\end{align*}

show that $\mu_{J-T}=0\forall g\in \mathbb{C}$

\textbf{Pf.}

\begin{align*}
    \mu_{J-T}=1\frac{1}{C_P}\left(\frac{\partial H}{\partial P}\right)_T=-\frac{1}{C_P}\left[\left(\frac{\partial V}{\partial P}\right)_T+P\left(\frac{\partial V}{\partial P}\right)_T+V\right]
\end{align*}

\begin{align*}
    =-\frac{1}{C_P}\left(0+{\left(\frac{\partial V}{\partial P}\right)_T}+V\right)
\end{align*}

\begin{align*}
    =-\frac{1}{C_P}\left[P(\frac{\partial\left[\frac{nRT}{P}\right]}{\partial P})+V\right]=-\frac{1}{C_P}\left[-\frac{nRT}{P} +V\right]=0 \blacksquare
\end{align*}
\subsection*{Example 2 (slide 40)}

If the Joule-Thomson coefficient for carbon dioxide, CO2, is $0.6375 \frac{K}{atm}$, estimate
the final temperature of carbon dioxide at $20atm$ and $100^{\circ}C$ that is forced through
a barrier to a final pressure of $1atm$.

\subsection*{How are $C_P$ and $C_V$ related?}

We can start with the differential form of the first law of thermodynamics.

\begin{align*}
    dq=C_VdT+\left(\frac{\partial U}{\partial V}\right)_T dV+P_{ext}dV
\end{align*}

Assume that $P\equiv const$, so

\begin{align*}
    dq_P=C_V dT+\left(\frac{\partial U}{\partial V}\right)_T dV+PdV
\end{align*}

\begin{align*}
    C_P=C_V+\left(\frac{\partial U}{\partial V}\right)_T\left(\frac{\partial V}{\partial T}\right)_P+P\left(\frac{\partial V}{\partial T}\right)_P=C_V+\left[\left(\frac{\partial U}{\partial V}\right)_T+P\right]\left(\frac{\partial V}{\partial T}\right)_P
\end{align*}

\begin{align*}
    =C_V+T\left(\frac{\partial P}{\partial V}\right)_V\left(\frac{\partial V}{\partial T}\right)_P
\end{align*}

Applying the cyclic rule along with the isobaric volumetric thermal expansion coefficient and isothermal compressibility, we have that

\begin{align*}
    C_P=C_V-T\frac{\left(\frac{\partial V}{\partial T}\right)_P^2}{\left(\frac{\partial V}{\partial P}\right)_T}
\end{align*}

Applying our $\alpha$ and $\kappa$ rules, we can conclude

\begin{align*}
    C_P=C_V+TV\frac{\alpha^2}{\kappa_T} or C_{P,m}=C_{V,m}+TV_m\frac{\alpha^2}{\kappa_T}
\end{align*}

What can we conclude from this? Well, $\forall g\in \mathbb{C} \land g\in \mathbb{R}$, we can say

\begin{align*}
    \alpha>0,\kappa_T>0\implies C_P-C_V=nR
\end{align*}

For solids and liquids, $C_P\approx C_V$.

\end{document}
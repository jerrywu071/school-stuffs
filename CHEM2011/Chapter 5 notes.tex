\documentclass[12pt]{book}

\usepackage[]{amsmath}
\usepackage[]{amsthm}
\usepackage[]{amsfonts}
\usepackage[]{amssymb}
\usepackage{blindtext}
\usepackage{pgfplots}
\usepackage[a4paper, total={6in, 8in}]{geometry}

\title{Chapter 5:: The 2nd law of thermodynamics}

\author{Jerry Wu}

\date{\today}

\begin{document}
\maketitle

\chapter*{Entropy}

\subsection*{Abstract}

Entropy is a measure of the degree of disorder/randomness in an arbitrary system. We can use this concept to discuss matters such as spontaneous and non spontaneous processes.

\subsection*{Spontaneous \& non spontaneous processes}

A spontaneous change is one that occurs without a continuous input of energy from an outside system. A good example would be gravitational potential. We can increase the rate of a spontaneous process by introducing a catalyst to the system. \textbf{A spontaneous process is not always fast}.\\Non spontaneous processes aren't impossible, but they require input of external energy or a \textbf{more spontaneous} reaction to proceed.

\subsection*{Example}

A good example of an impossible spontaneous process is a ball starting to bounce on a surface spontaneously. This would require all particles under it (approx $6.02E23$ particles) to start vibrating in the same direction at the same time. The chance of this being possible is basically 0.

\subsection*{Estimating spontaneity}

The first law of thermodynamics accounts for energy in a system, but it does not predict the spontaneity of the energy transfer involved. We will use entropy to do this. A process is spontaneous if and only if $\Delta S>0$.

\subsection*{Entropy as a state function}

Entropy is a measure of how dispersed the energy of a system is (in units of $\frac{J}{K}$). We can start with this property::

\begin{align*}
    \Delta S_{\Omega}=\Delta S_{\sigma}+\Delta S_{\bar{\sigma}}
\end{align*}

Where $\Omega$ is the universe, $\sigma$ is an arbitrary system, and $\bar{\sigma}$ is the surroundings.

To calculate the absolute entropy of a system, the following formula is used::

\begin{align*}
    S=k_b\ln(W)\implies \Delta S=k_b(\ln(W_0)-\ln(W_1))
\end{align*}

Where $k_b=1.38E-23 \frac{J}{K}$ is the boltzman constant, and $W=2^n$ is the number of possible arrangements of position and energy of all molecules in the system (microstates) for $n\in \mathbb{N}$ particles. The macrostate with the \textbf{highest entropy} also has the \textbf{greatest dispersal of energy}.

\subsection*{Multiplicity of energy}

Assume we have a hypothetical solid system with four atoms and a total energy of $E$. How many ways can we distribute that energy among the 4 atoms? Clearly there are 4 ways to distribute 1 unit of energy among 4 atoms because the system is small. However, if there are $N$ atoms, and total energy is $qE$, then the number of microstates is modelled by the following formula::

\begin{align*}
    W(N,q)=\frac{(q+N-1)!}{q!(N-1)!}
\end{align*}

\subsection*{Bringing it all together}

A hot and a cold bar, each made up of 4
atoms, are pushed together. What will
happen? If the cold bar has 1 quantum of energy
and the hot bar has 5 quanta of energy in
the initial state, show that heat transfer is
spontaneous by calculating $\Delta S_{\Omega}$ for
the process.

We can start by calculating the combinations of microstates for each process. %to finish later

\subsection*{Entropy at the molecular level}

In any system, when $T$ increases, $S$ will also increase proportionally with it. This makes sense because the faster the particles move, the more disorder there is in the system overall.

\subsection*{Statistical definition of entropy}

We can model the change in entropy of a system between two states with the following formula::

\begin{align*}
    \Delta S=S_2-S_1=k_b\ln(\frac{W_2}{W_1})
\end{align*}

This is possible since entropy is a state function.

\subsection*{Entropy change due to mixing of ideal gases}

Assume that mixing two ideal gases will sum up their entropies::

\begin{align*}
    \Delta S_{mix}=\Delta S_A + \Delta S_B=n_A R\ln\left(\frac{V_A+V_B}{V_A}\right)+ n_B R\ln\left(\frac{V_A+V_B}{V_B}\right)
\end{align*}

By Avogadro's law, $T\equiv const \land P\equiv const\implies V \propto n$, so,

\begin{align*}
    \Delta S_{mix}=n_A R\ln\left(\frac{n_A+n_B}{n_A}\right)+n_B R\ln\left(\frac{n_A+n_B}{n_B}\right)
\end{align*}

Using the laws of logarithms, we have that::

\begin{align*}
    \Delta S_{mix}=-n_A R\ln\left(\frac{n_A}{n_A+n_B}\right)-n_B R\ln\left(\frac{n_B}{n_A+n_B}\right)
\end{align*}

Further simplifying the expression, we get::

\begin{align*}
    \Delta S_{mix}=-R(n_A\ln(\chi_A)+n_B\ln(\chi_B))
\end{align*}

Recall that $\chi_{\xi}<1\forall \xi\in U$

\subsection*{Isothermal expansion of an ideal gas}

To calculate change in entropy for a process like this, we use::

\begin{align*}
    \Delta S=nR\ln(\frac{V_2}{V_1})
\end{align*}

Heat absorbed by the expansion is given by::

\begin{align*}
    q_{rev}=nRT\ln\left(\frac{V_2}{V_1}\right)
\end{align*}

So we have that::

\begin{align*}
    \frac{q_{rev}}{T}=nR\ln\left(\frac{V_2}{V_1}\right)\implies \Delta S=\frac{q_{rev}}{T}
\end{align*}

\subsection*{Example}
One mole of $N_2$ at $20.5^{\circ}C$ and $6.00 bar$ undergoes a transformation to the state described by $145^{\circ}C$ and $2.75 bar$. Calculate $\Delta S$ if

\begin{align*}
    C_{P,m}(T)=30.81-(11.87E-3)T+(2.3968E-5)T^2-(1.0176E-8)T^3
\end{align*}

We can invoke the formula::

\begin{align*}
    \Delta S=-nR\ln(\frac{P_2}{P_1})+n\int_{T_1}^{T_2}\frac{C_{P,m}(T)}{T}dT
\end{align*}

\begin{align*}
    =\int_{293.65}^{418.15}\frac{30.81}{T}-(11.87E-3)+(2.3968E-5)T-(1.0176E-8)T^2
\end{align*}

After integrating each term and some calculations, we have that::

\begin{align*}
    6.48+\left[30.81\ln(T)-(11.87E-3)T+\frac{2.3969E-5}{2}T^2-\frac{1.0176E-8}{3}T^3\right]_{293.65}^{418.15}\approx 16.8JK^{-1}
\end{align*}

\subsection*{The 2nd law of thermodynamics}

Any system $\sigma$ is significantly smaller than the surroundings. We can consider the surroundings $\sigma ^{-1}$ is an infinitely large reservoir. Thus, we can say::

\begin{align*}
    dq_{surr, rev}=dq_{surr, irrev}=dq_{surr}
\end{align*}

Since entropy is a state function, we can express change in entropy in the surroundings as the ratio::

\begin{align*}
    \Delta S_{surr}=\frac{q_{surr}}{T_{surr}}
\end{align*}

Consider the isothermal expansion of an ideal gas. Heat absorbed by the system is just $q_{sys}=nRT\ln\left(\frac{V_2}{V_1}\right)\implies q_{surr}=-nRT\ln\left(\frac{V_2}{V_1}\right)$. So for a \textbf{reversible} process, 

\begin{align*}
    \Delta S_{\Omega}=nR\ln\left(\frac{V_2}{V_1}\right)+\left[-nR\ln\left(\frac{V_2}{V_1}\right)\right]=0
\end{align*}

For an irreversible process (i.e. a gas expanding into a vacuum),

\begin{align*}
    \Delta S_{sys}=nR\ln\left(\frac{V_2}{V_1}\right)
\end{align*}

\subsection*{Cases (IMPORTANT)}

\begin{itemize}
    \item For a reversible process, $\Delta S_{\Omega}=0$
    \item For an irreversible process, $\Delta S_{\Omega}>0$
    \item Overall, $\Delta S_{\Omega}\geq 0$
\end{itemize}

\subsection*{Example}

One mole of an ideal gas at $25^{\circ}C$ is allowed to expand adiabatically
and irreversibly from $1L$ to $10 L$ with no work done. What is the final
temperature of the gas? Calculate the values of $\Delta S_{sys}, \Delta S_{surr}$ and $\Delta S_{univ}$.

We know that $q=0\land w=0\implies \Delta U=0$ in this system. So we can use the formula to calculate.

\begin{align*}
    \Delta S_{sys}=nR\ln\left(\frac{V_2}{V_1}\right)
\end{align*}

\begin{align*}
    \Delta S_{sys}=(8.3145)\ln\left(\frac{10}{1}\right)=19.1JK^{-1}mol^{-1}
\end{align*}

\begin{align*}
    q=0\implies\Delta S_{surr}=0\implies \Delta S_{\Omega}=19.1JK^{-1}mol^{-1}\implies R\equiv spont.
\end{align*}

\subsection*{Entropy change due to phase transitions}

\begin{itemize}
    \item In general, $\Delta_{vap} S^{\circ}>\Delta_{fus} S^{\circ}$ for the same substance.
    \item Solid to liquid transitions result in relatively small increase in entropy.
    \item Arrangements of molecules in gaseous state is completely random, implying larger increase in entropy from liquid to gas.
\end{itemize}

The molar entropy for oxygen gas ($O_2$) is shown on \textbf{slide 52}. We can see that phase changes have a large vertical jump on the entropy axis.\\

We can model the entropy change of the system during a phase change as the following equation::

\begin{align*}
    \Delta S_{sys}=\frac{\Delta H_{trans}}{T_{trans}}\implies \Delta S_{surr}=-\frac{\Delta H_{trans}}{T_{trans}}
\end{align*}

So at a phase transition, $\Delta S_{\Omega}=0$

For temperature dependence, we can model entropy change as::

\begin{align*}
    \Delta S=n\bar{C_P}\ln\left(\frac{T_2}{T_1}\right)
\end{align*}

To find total entropy change of water over an interval of temperature with 1 phase change (liquid to gas), we can say::

\begin{align*}
    \Delta S_{\Omega}=\sum_{S\in \Omega} \Delta S_i=C_P(H_2O(l))\ln\left(\frac{T_2}{T_1}\right)+\frac{\Delta H_{trans}}{T_2}+C_P(H_2O(g))\ln\left(\frac{T_3}{T_2}\right)
\end{align*}

\subsection*{The 3rd law of thermodynamics}

In essence, the entropy of a perfect crystal at $0K$ is 0. This law gives us the basis in calculating absolute entropies.

\subsection*{Proof}

$\exists$ \textbf{exactly one} microstate at $T=0K$, so $W=1^{6.02E23}=1$. So $S=k_b\ln(W)=k_b\ln(1)=0\blacksquare$

We can also express heat capacities and entropy of a perfect crystal as the following limits::

\begin{align*}
    \lim_{T\rightarrow 0}S=0, \lim_{T\rightarrow 0} C_P=0, \lim_{T\rightarrow 0} C_V=0
\end{align*}

We have never reached $0K$. The world record for lowest temperature is $100pK$.

\subsection*{Absolute entropy}
Assume that the following holds::
\begin{align*}
    q_{rev,P}=C_P dT
\end{align*}

Therefore at $298K$,

\begin{align*}
    S_{gas}(T)=S(0K)+\int_{0}^{T_{fus}}\frac{C_{P,sol}}{T} dT+\frac{\Delta H_{fus}}{T_{fus}}+\int_{T_{fus}}^{T_{boil}}\frac{C_{P,liq}}{T} dT+\frac{\Delta H_{vap}}{T_{boil}}+\int_{T_{boil}}^{T_{f}}\frac{C_{P,gas}}{T}dT
\end{align*}

If not at $298K$,

\begin{align*}
    \Delta S^T=\Delta S_{298.15K}+\int_{298.15}^{T}\frac{n\Delta C_p}{T}dT=\Delta S_{298.15K}+nC_P\ln\left(\frac{T}{298.15}\right)
\end{align*}

\subsection*{Entropy changes for chemical reactions}

Simply invoke Hess' law for entropy.

\begin{align*}
    \Delta S_r=\sum v_p S_{p}-\sum v_r S_r
\end{align*}

\subsection*{Example}

If a spark is applied to a mixture of $H_2(g)$ and $O_2(g)$, an explosion occurs and
water is formed. The gaseous water is cooled to $100^{\circ}C$. Calculate the entropy
change when $2 mol$ of gaseous $H_2O$ is formed at $100^{\circ}C$ and $1 atm$ from $H_2(g)$
and $O_2(g)$ at the same temperature and each at a particular pressure of $1 atm$.

\begin{align*}
    R:2H_2(g)+O_2(g)\rightarrow2H_2O(g)
\end{align*}

\begin{align*}
    \Delta S(25^{\circ}C)=2\bar{S}_{H_2O(g)}-\bar{S}_{O_2(g)}-2\bar{S}_{H_2(g)}
\end{align*}

\begin{align*}
    =2(188.72)-205.04-2(130.57)=-88.74\frac{J}{K}
\end{align*}
In order to find $\Delta S$ at $100^{\circ}C, 1atm$, we need to know the heat capacities of the reactants and products from the table.

\begin{align*}
    \Delta C_P(25^{\circ}C)=2\bar{C_P}(H_2O(g))\bar{C_P}(O_2(g))-2\bar{C_P}(H_2(g))
\end{align*}

\begin{align*}
    =2(33.6)-29.4-2(28.8)=-19.8\frac{J}{K}
\end{align*}

\begin{align*}
    \Delta S(100^{\circ}C)=\Delta S(25^{\circ}C)+\int_{298}^{373}\frac{\Delta C_P}{T} dT
\end{align*}

\begin{align*}
    -88.74\frac{J}{K}-19.8\ln\left(\frac{373}{298}\right)=-93.18\frac{J}{K}
\end{align*}

\subsection*{The Carnot heat engine}

An idealized model of a heat engine which converts heat to mechanical work (car engines, steam engines, etc.). Naturally, it is easier to turn work into heat, rather than heat into work. The reversible Carnot cycle dictates that::

\begin{itemize}
    \item Isothermal expansion
    \item Adiabatic expansion
    \item Isothermal compression
    \item Adiabatic compression
\end{itemize}

What is the cause of this asymmetry? Well, we have to consider the following phenomena occuring in this system::

\begin{itemize}
    \item Entropy increases causing irreversible changes
    \item Some energy is lost as waste heat
    \item Amount of work the system can do is limited as a result
\end{itemize}

We can deduce that, from a PV diagram with 4 points::

\begin{itemize}
    \item $\Delta U=0$ since it is a cyclic path
    \item $q(\psi)=\sum_{q\in \sigma}q_i=q_1+q_2$
    \item $w(\psi)=-RT_2\ln\left(\frac{V_2}{V_1}\right)-RT_1\ln\left(\frac{V_4}{V_3}\right)=-R(T_2-T_1)\ln\left(\frac{V_2}{V_1}\right)$
    \item $q_2=RT_2\ln\left(\frac{V_2}{V_1}\right), q_1=RT_1\ln\left(\frac{V_4}{V_3}\right)=-RT_1\ln\left(\frac{V_2}{V_1}\right)$
    \item $\Delta S_\Sigma=0 \forall$ cyclic processes
    \item 
\end{itemize}

\subsection*{Efficiency of a heat engine}
Efficiency can be expressed by the formula::
\begin{align*}
    \epsilon=\frac{\sum_{w\in \sigma}w_i}{q_2}=\frac{R(T_2-T_1)\ln\left(\frac{V_2}{V_1}\right)}{RT_2\ln\left(\frac{V_2}{V_1}\right)}=\frac{T_2-T_1}{T_2}=1-\frac{T_{cold}}{T_{hot}}
\end{align*}

So it would follow that $w_{\psi,irrev}<w_{\psi,rev}\implies \epsilon_{irrev}<\epsilon_{rev}<1$. This makes sense because actual efficiency is less than ideal carnot efficiency.

\subsection*{Deriving $S$ from efficiency}

\begin{align*}
    \epsilon=\frac{q_1+q_2}{q_2}=\frac{T_2-T_1}{T_2}=\frac{|w|}{q_{in}}
\end{align*}
\begin{align*}
    1+\frac{q_1}{q_2}-1+\frac{T_1}{T_2}=0
\end{align*}
\begin{align*}
    \frac{q_2}{T_2}+\frac{q_1}{T_1}=0
\end{align*}
\begin{align*}
    \oint \frac{dq_{rev}}{T}dT=0\implies \Delta S=\int \frac{dq_{rev}}{T}dT
\end{align*}

Note that $\Delta S$ must be calculated along a \textbf{reversible} path!

\subsection*{Example}
10 moles of water at $60^{\circ}C$ are mixed with an equal amount of water at $20^{\circ}C$. Neglect any heat exchange with the surroundings and calculate the entropy change. The heat capacity of water may be
taken to be $75.3 JK^{-1}mol^{-1}$ and independent of temperature.

\subsection*{Solution}

We know that $q_1+q_2=0$ by way of common sense, so,

\begin{align*}
    n\bar{C_P}(T_f-20)+n\bar{C_P}(T_f-60)=0\implies T_f=40^{\circ}C
\end{align*}

This is a trivial question since there is no phase change, and there are multiple equal quantities. If the quantities are different, we'd need to do more calculations. To calculate $\Delta S$, simply plug into the formula for heating and cooling::

\begin{align*}
    \Delta S_1=n\bar{C_P}\ln\left(\frac{T_f}{T_i}\right)=(10)(75.3)\ln\left(\frac{313}{293}\right)=-46.62JK^{-1}
\end{align*}

\begin{align*}
    \Delta S_2=(10)(75.3)\ln\left(\frac{313}{333}\right)=49.7JK^{-1}
\end{align*}


\begin{align*}
    \implies\Delta S_{\Sigma}=3.08JK^{-1}
\end{align*}
\textbf{MUST} USE KELVIN FOR THIS FORMULA!

\subsection*{Example 2}

1 mol of an ideal gas at $298K$ expands isothermally from $1L$ to $2L$ a) reversibly, b) against a constant external pressure of $12.2atm$ (irrev). Calculate the values of $\Delta S_{\sigma}, \Delta S_{surr}, \Delta S_{\Omega}$

\subsection*{Solution}

\begin{itemize}
    \item[a)]
    \begin{align*}
        \Delta S_{\sigma}=\frac{q_{rev}}{T}=nR\ln\left(\frac{V_2}{V_1}\right)=5.8JK^{-1}
    \end{align*}
    
    We know that since the process is reversible, $\Delta S_{\Omega}$ overall is 0. So,
    
    \begin{align*}
        \Delta S_{\Omega}=0\implies \Delta S_{surr}=-\Delta S_{\sigma}=-5.8JK^{-1}
    \end{align*}

    \item[b)] From the previous section, we know that $\Delta S_{\sigma}=5.8JK^{-1}$. Since the process is irreversible and isothermal, we can calculate the heat exchange of the system.
    \begin{align*}
        q_{\sigma}=-w=P\Delta V=1236J
    \end{align*}

    Taking the negative of the heat exchange of the system, we get the value for the surroundings.

    \begin{align*}
        \Delta S_{surr}=\frac{q_{surr}}{T}=\frac{-1236}{298}=-4.15JK^{-1}\implies \Delta S_{\Sigma}=1.7JK^{-1}
    \end{align*}
\end{itemize}

\subsection*{Question 6}

A sample of an ideal monoatomic gas in a $0.8L$ container is found to be at $25^{\circ}C$ and $1atm$. The gas is heated to $100^{\circ}C$ and it simultaneously expands to $1.5L$. Calculate $\Delta S$.

\subsection*{Solution}

Based on the parameters above, we can reason that our overall change in entropy is positive.

\begin{itemize}
    \item For the isothermal step,
    \begin{align*}
        \Delta S=nR\ln\left(\frac{V_2}{V_1}\right),n=\frac{P_1V_1}{RT_1}=0.0327 mol
    \end{align*}
    \begin{align*}
        \Delta S=(0.0327)(8.314)(\ln\left(\frac{1.5}{0.8}\right)=0.1709 JK^{-1}
    \end{align*}
    \item For the isochoric step,
    \begin{align*}
        \Delta S=n\bar{C_V}\ln\left(\frac{T_2}{T_1}\right)=(0.0327)(\frac{3}{2}R)\ln\left(\frac{373}{298}\right)=0.0915JK^{-1}
    \end{align*}
\end{itemize}

So overall, we have that

\begin{align*}
    \Delta S_\Sigma=0.262JK^{-1}
\end{align*}

\end{document}
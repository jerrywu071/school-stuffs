\documentclass[12pt]{article}

\usepackage[]{amsmath}
\usepackage[]{amsthm}
\usepackage[]{amsfonts}
\usepackage[]{amssymb}
\usepackage{blindtext}
\usepackage[a4paper, total={6in, 8in}]{geometry}

\title{CHEM2011 Chapter 1 problems}
\date{\today}
\author{Jerry Wu}

\begin{document}
\maketitle

\subsection*{Question 1a,b}

One mole of an ideal \textbf{monoatomic} gas initially at $300k$ and pressure of $15.0atm$ expands to a final pressure of $1.00 atm$. The expansion can occur via any of five different paths. For each case, calculate the value of $q$, $w$, $\Delta U$ and $\Delta H$.

\begin{itemize}
    \item[a)] Isothermal and reversible
    \subsection*{Solution::}
    Assume $T_2=T_1\implies \Delta T=0$ by isothermal. Because $\Delta T=0$, we have that $\Delta H=0$ and $\Delta U=0$ as a consequence. So, $\Delta U=q+w=0\implies w=-q$.\\
    We can now find the value of $w$ by using the formula::

    \begin{align*}
        w=-nRT\ln(\frac{P_1}{P_2})=-(1)(8.314)(300)\ln(\frac{15}{1})=-6.75E-3 J\implies q=6.75E-3J
    \end{align*}

    \item[b)] Isothermal and irreversible
    \subsection*{Solution::}
    $\Delta U$ and $\Delta H$ are still 0 since $\Delta T=0$. Because this process is irreversible, $P_2=P_{external}$. So we have that
    \begin{align*}
        w=-P_2\Delta V=-P_2(V_2-V_1)
    \end{align*}

    Apply ideal gas law to find $V_1$ and $V_2$.

    \begin{align*}
        V_1=\frac{nRT}{P_1}=\frac{(1)(0.08206)(300)}{15}=1.641L, V_2=\frac{nRT}{P_2}=24.62L
    \end{align*}

    We can now find our work for the irreversible process.

    \begin{align*}
        w=-1(24.62-1.641)(\frac{101.325J}{1Latm})=-2.33E3J\implies q=2.33E3J
    \end{align*}

    \item[c)] Isothermal and irreversible in a 2 step process. \textbf{Step 1::} $P=7atm$, \textbf{Step 2::} P=$1atm$.
    \subsection*{Solution::}
    The process is still isothermal, so assume $\Delta T, \Delta U, \Delta H=0$ and $q=-w$.\\ Let $P'=7atm$ denote the intermediate pressure between the two steps. So we want to find the work berween $V_1$ and $V'$ along with $V'$ and $V_2$.

    \begin{align*}
        V_1=1.641L, V'=\frac{(1)(0.08206)(300)}{7}=3.52L, V_2=24.62L
    \end{align*}

    Now let us calculate the work done for both steps.

    \begin{align*}
        w=w_1+w_2=-P_2\Delta V=-((7)(3.52-1.641)+(1)(24.62-3.52))\approx -34.251Latm(\frac{101.325}{Latm})=-3470J=-3.47E3J\implies q=3.47E3J
    \end{align*}
\end{itemize}

% QUESION 2 AND 3 ARE ADIABATIC AND WILL BE COVERED ON THE TEST, BUT NOT THE QUIZ

\subsection*{Question 4}

The constant pressure heat capacity of an ideal gas, $A\in \mathbb{C}$ was found to vary with temperature according to the expression

\begin{align*}
    \bar{C}_P=22.17+0.32T in units of JK^{-1}mol^{-1}
\end{align*}

\begin{itemize}
    \item[a)] Calculate $q,w,\Delta U, \Delta H$ when the temperature of $2 mol$ of $A$ is raised from $0^{\circ}C\rightarrow 50^{\circ}C (273K\rightarrow 323K)$. Assume $P\equiv const$.
    
    \subsection*{Solution::}

    First, we can find $\Delta H$.

    \begin{align*}
        \Delta H=\int_{273}^{323} C_P dT=n\int_{273}^{323}\bar{C}_P dT=(2)\int_{273}^{323} 21.17+0.32T=2\left[21.17T+\frac{0.32}{2}T^2\right]_{273}^{323}
    \end{align*}

    \begin{align*}
        =2\left[\left(22.17(323)+\frac{0.32}{2}(323)^2\right)-\left(22.17(273)+\frac{0.32}{2}(273)^2\right)\right]=11753J=11.75kJ
    \end{align*}

    Because we assumed $P\equiv const$, $\Delta H=q=11.75kJ$. Now we find $\Delta U$.

    \begin{align*}
        \Delta U=\Delta H-\Delta (PV)=\Delta H-nR\Delta T=11753-(2)(8.314)(323-273)=10.92kJ
    \end{align*}

    And so, $w=\Delta U-q$.
    \begin{align*}
        w=10.92kJ-11.75kJ=-0.83kJ
    \end{align*}

    Because work is negative, work was done by the system, whereby we have an expansion.

    \item[b)] Now assume $V\equiv const$.
    
    \subsection*{Solution}

    Because $V\equiv const$, we can carry over our $\Delta H$ and $\Delta U$ values since they're only dependent on temperature.\\When volume is constant, $w=0$ since there is no expansion or compression. This implies that $q=\Delta U=10.92kJ$.

\end{itemize}

\end{document}
\documentclass[12pt]{book}

\usepackage[]{amsmath}
\usepackage[]{amsthm}
\usepackage[]{amsfonts}
\usepackage[]{amssymb}
\usepackage{blindtext}
\usepackage{pgfplots}
\usepackage{etoolbox}
\AtBeginEnvironment{align}{\setcounter{equation}{0}}
\usepackage[a4paper, total={6in, 8in}]{geometry}
\setlength\parindent{24pt}

\title{Chapter 18-19:: Kinetics}

\author{Jerry Wu}

\date{\today}

\begin{document}
\maketitle
\chapter*{Chapter 18:: Chemical kinetics}

\section*{Abstract}
Kinetics lets us determine the speed/how fast a process will occur in some arbitrary system $\sigma$. Some examples include carbon dating, digestion times, atmospheric chemistry, etc. When making decisions about processes, we have to consider both thermodynamics and kinetics.

\section*{Rates of reactions}

Reaction rates concern how the concentration of a reactant/product in a system changes as a function of time, i.e. $[A]=f(t)$. Simply put, the rate of reaction is expressed as::

\begin{align*}
    Rate=\frac{[A]}{\Delta t}
\end{align*}
So we can conclude that when $R_{reaction}=0\implies$ reaction has been completed. We will use this to prove completeness later. 

\subsection*{Formal definition of reaction rate}
We need an expression to find a rate expression which works $\forall prod\forall rxt\in \sigma$.
Take the following arbitrary dissociation reaction::

\begin{align*}
    R: aA\rightarrow dD+eE
\end{align*}

Then the reaction rate can be expressed as the following::

\begin{align*}
    Rate=-\left(\frac{1}{a}\right)\left(\frac{\Delta [A]}{\Delta t}\right)=\left(\frac{1}{d}\right)\left(\frac{\Delta D}{\Delta t}\right)=\left(\frac{1}{e}\right)\left(\frac{\Delta [E]}{\Delta t}\right)
\end{align*}

Since we're decreasing the amount of reactant and increasing the amount of product, we put a negative for the rate for the reactants, and positive for the rate for the products. So overall, $Rate\geq 0\forall t\in \mathbb{R}$. Rate is measured in $M/s$.

\subsection*{Average rate of reaction}

Average rate of reaction concerns the rate of reaction in a macroscopic time interval, i.e. $t\in[t_1,t_2]\subset \mathbb{R}$. To do this, we can just use the slope formula for linear functions. Assume $A$ is a reactant in an arbitrary reaction, with $a$ as its stoichiometric coefficient.

\begin{align*}
    E[Rate]=-\left(\frac{1}{a}\right)\left(\frac{[A]_2-[A]_1}{t_2-t_1}\right)
\end{align*}

\subsection*{Instantaneous rate of reaction}

This topic is also trivial, as we use the definition of a derivative. Assume that $A$ is a reactant in an arbitrary reaction, with $a$ as its stoichiometric coefficient.


\begin{align*}
    Rate_i=-\left(\frac{1}{a}\right)\lim_{\Delta t\rightarrow 0} \frac{f(t+\Delta t)-f(t)}{\Delta t}
\end{align*}

\subsection*{Rate laws of reactions}

We want a way to express the rate of a reaction in an elegant manner while keeping the expression independent of time. This is where rate laws come into play. For some arbitrary reaction with $m$ reactants and $n$ products, the rate of reaction can be modelled as the following::

\begin{align*}
    R: a_0A_0+a_1A_1+\ldots+a_mA_m\rightarrow b_0B_0+b_1B_1+\ldots+b_nB_n
\end{align*}

\begin{align*}
    \equiv R: \sum_{i=0}^{m}a_iA_i\rightarrow\sum_{i=0}^{n}b_iB_i
\end{align*}

\begin{align*}
    \implies Rate(R)=k[A_0]^{\omega_0}[A_1]^{\omega_1}\ldots[A_m]^{\omega_m}=k\prod_{i=0}^{m} [A_i]^{\omega_i}
\end{align*}

\begin{itemize}
    \item $[A_i]$ is the concentration of some reactant in the reaction in $M$ or $P$
    \item $k$ is the rate constant, units vary from reaction to reaction
    \item $\omega_i$ is the reaction order for the respective $i$-th reactant in the set of reactants.
\end{itemize}

\subsection*{Some things to keep in mind}

\begin{itemize}
    \item Reaction orders, and consequently rate laws are determined \textbf{experimentally}
    \item Stoichiometric coefficients cannot equal to reaction orders, i.e. $\forall R\forall i(a_i\neq \omega_i)$.
    \item Overall order of a reaction is the sum of all individual orders, i.e. \\$O(R)=\sum_{i=0}^{n} \omega_i$
\end{itemize}

\subsection*{Method of initial rates}

Since the rate constant $k$ is independent of concentrations of products and reactants, we can set the initial rate equal to the final rate to determine the values of $m$ and $n$. For example if we have a reaction,

\begin{align*}
    R: aA\rightarrow products
\end{align*}

and we have that the initial rate of the reaction at $t=0$ is::

\begin{align*}
    v_0=k[A]^n\implies v_0=\lim_{t\rightarrow 0}\left(-\frac{1}{a}\frac{d[A]}{dt}\right)=k[A]_0^n
\end{align*}

We can get a linear function in slope intercept form by taking $\ln$ of both sides.

\begin{align*}
    \ln(v_0)=n\ln([A]_0)+\ln(k)
\end{align*}

\textbf{The slope of this function ($n$) will be the order of the reaction.} To generalize, we can state that for the following reaction,

\begin{align*}
    R: a_0A_0+a_1A_1+\ldots+a_mA_m\rightarrow b_0B_0+b_1B_1+\ldots+b_nB_n
\end{align*}

\begin{align*}
    \equiv R: \sum_{i=0}^{m}a_iA_i\rightarrow\sum_{i=0}^{n}b_iB_i
\end{align*}

The initial rate can be expressed as the following product::

\begin{align*}
    v_0=k[A_0]^{\omega_0}[A_1]^{\omega_1}\ldots[A_m]^{\omega_m}=\lim_{t\rightarrow 0}\left(-\frac{1}{a_0}\frac{d[A_0]}{dt}\right)=k\prod_{i=0}^{m} [A_i]_0^{\omega_i}
\end{align*}

To find each $\omega_i$, we can keep every other one constant and find each $\ln$ relationship individually. 

\begin{align*}
    \forall R\forall i(\ln(v_0)=\omega_i\ln([A_i]_0+\ln(k')))
\end{align*}

\subsection*{Isolation method}

Take a generic reaction

\begin{align*}
    R:aA+bB\rightarrow products
\end{align*}

We assume the following to use this method::

\begin{itemize}
    \item Assume $R$ will run until $B$ is used up
    \item $[A]$ will be $0.99M$ once it is used up
    \item $[A]$ can be assumed to be constant over that time interval, since the reaction has been completed during said interval.
    \begin{align*}
        Rate=k'[B]^{\beta}, k'=k[A]^{\alpha}
    \end{align*}

    \item We can determine the order with respect to $A$ in the same manner.
\end{itemize}

\section*{Reaction mechanisms}
Reactions will typically take place in multiple elementary steps before proceeding to a final product. These sequences of steps are known as reaction mechanisms. In general, a reaction is a sum of its elementary steps, i.e.

\begin{align*}
    R=\sum_{i=0}^{n} R_i
\end{align*}

for a reaction with $n$ elementary steps.

\subsection*{Reaction intermediates}
Simply put, a reaction intermediate is a compound that is formed during a reaction that is consumed during a subsequent elementary step, i.e. it does not appear in the final product. For example,

\begin{align}
    R_1:NO_2(g)+NO_2(g)\rightarrow NO(g)+NO_3(g)\\
    R_2:NO_3(g)+CO(g)\rightarrow NO_2(g)+CO_2(g)\\
    R: NO_2(g)+CO(g)\rightarrow NO(g)+CO_2(g)
\end{align}

We can see that $NO_3$ is a reaction intermediate, as it doesn't appear in the overall reaction $R$.

\subsection*{Proving reaction correctness and completeness}

When finding a reaction mechanism, it is not possible to prove that a mechanism is completely accurate. We can only approximate and prove that a mechanism is \textbf{plausible}. For example, take the following reaction::

\begin{align*}
    R: 2NO_2+F_2\rightarrow 2NO_2F|Rate=k[NO_2][F_2]
\end{align*}

Because this is an overall reaction, the stoichiometric coefficients are not necessarily equal to the orders of each reactant (\textbf{they are however always equal for elementary steps}). Here, both $NO_2$ and $F_2$ are first order, i.e. $NO_2,F_2\in O(n)$. Our job is to find and prove a reaction mechanism that is acceptable. Let's take an arbitrary mechanism with 1 elementary step for the previous overall reaction::

\begin{align}
    R_1:NO_2+NO_2+F_2\rightarrow 2NO_2F|Rate=k[NO_2]^2[F_2]
\end{align}

Here, we use the stoichiometric coefficients as the rate orders since it is an elementary step. We can see that $NO_2$ is 2nd order, i.e. $NO_2\in O(n^2)$. Because the rate law for this elementary step is not equal to the one for the overall reaction, we reject this example as an acceptable reaction mechanism, i.e. it doesnt agree with the experiment.\\So we can say that \textbf{a mechanism is plausible if and only if the elementary steps added together will produce the same rate law as the overall reaction}.\\

Now let us properly divide the overall reaction into elementary steps::
\begin{align}
    R_1:NO_2+F_2\rightarrow^{k_1} NO_2F+F\\
    R_2:F+NO_2\rightarrow^{k_2} NO_2F\\
    R:2NO_2+F_2\rightarrow^k 2NO_2F
\end{align}
After summing up the elementary steps, we arrive at our overall reaction, so this mechanism is indeed correct.

\subsection*{Proof}
Here, depending on the magnitude of each step's rate constants ($k_1,k_2$), it will determine how fast each individual step is. We can also conclude the following properties of this process::

\begin{itemize}
    \item $F$ is the reaction intermediate
    \item The molecularity of both elementary steps is 2, i.e. we have 2 molecules of reactants.
    \item The rate law for both steps are as follows::
    \begin{align}
        Rate_{R_1}=k_1[NO_2][F_2]\\
        Rate_{R_2}=k_2[NO_2][F]
    \end{align}
    Rate orders are all $O(n)$ in this case.
\end{itemize}

Assume $R_1$ is the slow step (rate determining step), and $R_2$ is the fast step. If this is the case, then the overall rate law should be identical to that of $R_1$.\\From this proposition it would follow, because the relative rates postulated for $R_1$ and $R_2$ are consistent with the overall rate law, it would follow that the mechanism is plausible. $\blacksquare$

\subsection*{Reaction mechanism with equilibriums}

Take the following reaction mechanism::

\begin{center}
    $R_1:Cl_2\rightleftarrows_{k_-1}^{k_1} 2Cl$ Fast equilibrium\\
    $R_2:Cl+CO\rightleftarrows_{k_-2}^{k_2} COCl$ \indent Fast equilibrium\\
    $R_3:COCl+Cl_2\rightarrow^k COCl_2+Cl$ \indent Slow step
\end{center}

Our job is to determine the reaction order. When doing this, we want to always use the \textbf{slowest} step in the mechanism to determine the rate law and reaction orders. When $\exists$ equilbirum in our mechanism, we know that those steps are reversible and that their \textbf{forward and reverse rates are equal!}\\ From here, we can express our rate law in terms of the slowest step::

\begin{align}
    Rate=Rate_{R_3}=k[COCl][Cl_2]\\
    Rate_{R_1}=k_1[Cl_2]=k_{-1}[Cl]^2\\
    Rate_{R_2}=k_2[Cl][CO]=k_{-2}[COCl]\\
    \implies Rate=k[COCl][Cl_2]
\end{align}

Because $COCl$ and $Cl$ are intermediates, we want to get rid of it in our final rate expression. How do we do this? We can use the fact that $R_1$ and $R_2$ are equilibrium reactions to rearrange (2) and (3) to express $[COCl]$ independently and substitute the rewritten value into the overall rate law. Here, we'll rearrange the rate laws for $R_1$ and $R_2$.

\begin{align}
    k_2[Cl][CO]=k_{-2}[COCl]\implies [COCl]=\frac{k_2[Cl][CO]}{k_{-2}}\\
    k_1[Cl_2]=k_{-1}[Cl]^2\implies [Cl]=\left(\frac{k_1[Cl_2]}{k_{-1}}\right)^{\frac{1}{2}}
\end{align}

Substituting these values in for the intermediates, we have that::

\begin{align}
    Rate=k[COCl][Cl]=k\left(\frac{k_2[Cl][CO]}{k_{-2}}\right)[Cl_2]\\=\frac{k_2k\left(\frac{k_1[Cl_2]}{k_{-1}}\right)^{\frac{1}{2}}[CO][Cl_2]}{k_{-2}}\\=\frac{k_1^{\frac{1}{2}}k_2k[Cl_2]^{\frac{3}{2}}[CO]}{k_{-1}^{\frac{1}{2}}k_{-2}}
\end{align}

Cleaning this up by grouping rate constants together as ratios, we have that::

\begin{align*}
    Rate=k\left(\frac{k_1}{k_{-1}}\right)^{\frac{1}{2}}\left(\frac{k_2}{k_{-2}}\right)[CO][Cl]^{\frac{3}{2}}
\end{align*}

Therefore, the reaction orders are $Cl_2\in O(n^{1.5})$ and $CO\in O(n)$.

\subsection*{Molecularity}

In simple terms, molecularity is the number of molecules in the reactants side, i.e. $|Rxts|$. Unimolecular and bimolecular reactions are commonplace, but any $|Rxts|>2$ is extremely rare. $\nexists$ any elementary reaction with $|Rxts|>3$ that are known currently. 

\subsection*{Theorem}
For any given overall reaction, it's elementary steps can have a molecularity of at most the overall reaction's molecularity.

\begin{align*}
    \forall R_i\subset R(|Rxts(R_i)|\leq|Rxts(R)|)
\end{align*}

\subsection*{Integrated rate law ($O(1)$)}
Let us start by taking the rate law of an arbitrary $O(1)$ reaction. 
\begin{align}
    A\rightarrow^k products
\end{align}
What does it mean to be $O(1)$? It just means that the reaction will run in constant time for one reactant i.e. the rate will be constant regardless of the concentration.

\begin{align}
    Rate=\frac{-d[A]}{dt}=k[A]^0=k
\end{align}

From here, we can derive the integrated rate law for this reaction via integration, hence "integrated" rate law. This can be done with FTC1.

\begin{align}
    \frac{-d[A]}{dt}=k\implies \int_{[A]_0}^{[A]_t}dA=-\int_{0}^{t}kdt\\
    [A]_{[A_0]}^{[A]_t}=-kt\implies [A]_t=[A]_0-kt
\end{align}

With this, we get $[A]$ at some point in time $t$ as a function of $t$. It is in slope intercept form; the rate constant is the slope, and the y intercept is the initial concentration.

\subsection*{Half life of $O(1)$ processes}

The definition of this property is trivial, as we can take half of the initial concentration as the half life ($[A]_{\frac{\lambda}{2}}=\frac{1}{2}[A]_0$). We can derive a better expression for this via the integrated rate law::

\begin{align}
    [A]_t-[A]_0=-kt\\
    \frac{[A]_0}{2}-[A]_0=-kt_{\frac{1}{2}}\\
    \frac{[A]_0}{2}=kt_{\frac{1}{2}}\implies t_{\frac{1}{2}}=\frac{[A]_0}{2k}
\end{align}

Remember that half life is exponential decay.

\subsection*{Integrated rate law ($O(n)$)}

For first order reactions, deriving the integrated rate law is exactly the same as for $O(1)$ reactions. We will use the same monomolecular reaction from the $O(1)$ example for the derivation.

\begin{align}
    Rate=\frac{-d[A]}{dt}=k[A]\implies \frac{d[A]}{[A]}=-kdt\\
    \int_{[A]_0}^{[A]_t}\frac{d[A]}{[A]}=-\int_{0}^{t}kdt\implies \left[\ln([A])\right]_{[A]_0}^{[A]_t}=\left[-kt\right]_{0}^{t}\\
    \ln([A]_t)-\ln([A]_0)=-kt\implies \ln\left(\frac{[A]_t}{[A_0]}\right)=-kt\\
    \implies [A]_t=[A]_0 e^{kt}
\end{align}
\newpage
\subsection*{Half life of $O(n)$ processes}

We can derive this in the same method as the $O(1)$ derivation.

\begin{align}
    \ln\left(\frac{[A]_t}{[A]_0}\right)=-kt\\
    \ln\left(\frac{\frac{1}{2}[A]_0}{[A]_0}\right)=-kt\\
    \ln\left(\frac{1}{2}\right)=-kt\\
    t_{\frac{1}{2}}=\frac{\ln(2)}{k}
\end{align}

From this, we can conclude that the half life of a $O(n)$ process is a value which is independent of any concentrations $\implies O(n)$ has a constant half life. Thus, $t_{\frac{1}{2}}\equiv const\implies R\in O(n)$.

\subsection*{Example}

A child weighing $20 kg$ has an infection and requires the intake of antibiotics. It is found that
the antibiotic, $A$, during its metabolization obeys the relation $\frac{-d[A]}{dt}=k[A]$. The rate
constant, $k$, depends on temperature and body weight, but for a $20 kg$ child with no fever it
is equal to $2.00E-8$ per milliseconds.

\begin{itemize}
    \item[a)] If the child has taken the first pill, when should it take its second pill (which all contain $100 mg$ of antibiotics) to keep the concentration at a minimum of $1.50 mg$ per $kg$ of body weight assuming instantaneous and uniform distribution of the antibiotic throughout the
    body mass as soon as the pill has been ingested?

    \subsection*{Solution}
    Looking at the rate law, it is evident that this process is $O(n)$, thus the half life will be constant.\\
    First, we determine our rate law equation::
    \begin{align}
        \ln\left(\frac{[A]}{[A]_0}\right)=-kt
    \end{align}

    Next, find initial concentration::
    \begin{align}
        [A]_0=\frac{100mg}{20kg}=5 mg/kg
    \end{align}

    So we can now calculate the time for the next pill.

    \begin{align}
        \ln\left(\frac{1.5}{5}\right)=-2E8t\implies t=6.02E7 ms\approx 16.7 hours
    \end{align}

    $\therefore$ the kid should wait somewhere in the neighbourhood of 16.7 hours before taking the next pill.

    \item[b)]Do you think that the child should be given its third pill after a shorter time interval, after the same time interval, or after a longer time interval than it did for the second
    pill? Give a brief explanation for your answer.

    \subsection*{Answer}
    The child should wait a longer time interval before taking the next pill, since the concentration only drops to $1.5 mg/kg$ after 16.7 hours.

    \item[c)] At what time (value of $t$) would M reach its maximum concentration? Determine the
    expression for the maximum concentration of $[M]$.
    
    \subsection*{Answer}
    Knowing that the process is $O(n)$, we can conclude that we should plot $\ln([A])$ vs $t$ to obtain a straight line plot.

    \item[d)] What is the relationship between the slope of the plot in part (c) and the rate
    constant of the reaction?

    \subsection*{Answer}

    The slope of the plot is exactly $-k$.

\end{itemize}

\subsection*{Example involving differential equations}

A compound $M$ is being synthesized by a $O(1)$ mechanism with a rate constant $k_0$. It is also being degraded by a $O(n)$ mechanism with rate constant $k_1$.

\begin{itemize}
    \item[a)] Write a differential equation consistent with the mechanism.
    
    \subsection*{Solution}
    By inspection of the mechanism, we have that::

    \begin{align*}
        \frac{d[M]}{dt}=k_0-k_1[M]
    \end{align*}

    This equation will model production of products in $O(1)$ and consumption of reactants in $O(n)$.

    \item[b)] If $[M]=0$ for $t=0$, find the solution to the differential equation. 
    
    \subsection*{Solution}

    We do this in the same manner as deriving rate laws, i.e. \textbf{isolate everything with $t$ on one side, and everything with $[M]$ on the other side}.

    \begin{align}
        \frac{d[M]}{dt}=k_0-k_1[M]\\
        \frac{d[M]}{k_0-k_1[M]}=dt\\
        \left[-\frac{1}{k_1}\ln(k_0-k_1[M])\right]_{[M]_0}^{[M]}=t\\
        \ln\left(\frac{k_0-k_1[M]}{k_0}\right)=-k_1t\\
        1-\frac{k_1[M]}{k_0}=e^{-k_1t}\\
        [M]=\frac{k_0}{k_1}(1-e^{-k_1t})
    \end{align}

    \item[c)] At what time (value of $t$) would $M$ reach its maximum concentration? Determine the
    expression for the maximum concentration of $[M]$.

    \subsection*{Solution}

    We can find the derivative of the function and find the solutions when $\frac{d[M]}{dt}=0$.

    \begin{align*}
        \frac{d[M]}{dt}=0=k_0e^{-k_1t}\implies [M]=\frac{k_0}{k_1}
    \end{align*}

    So $[M]$ will reach its maximum when $t\rightarrow \infty$.

\end{itemize}

\subsection*{Integrated rate law for $O(n^2)$}

Again, we will use the same method of derivation for the $O(n^2)$ integrated rate law as the previous two. Use the same monomolecular process as a basis.

\begin{align}
    Rate=\frac{-d[A]}{dt}=k[A]^2\implies \frac{d[A]}{[A]^2}=-kdt\\
    \int_{[A]_0}^{[A]_t}\frac{d[A]}{[A]^2}=-\int_{0}^{t}kdt\implies \left[-\frac{1}{[A]}\right]_{[A]_0}^{[A]_t}=-kt\\
    -\left(\frac{1}{[A]_t}-\frac{1}{[A]_0}\right)=-kt\implies \frac{1}{[A]_t}=kt+\frac{1}{[A]_0}
\end{align}

With this equation, we can plot $\frac{1}{[A]}$ vs $t$ in order to get a straight line plot.

\subsection*{Half life for $O(n^2)$ processes}

Let us derive the half life formula for $O(n^2)$ in the same manner as the $O(n)$ and $O(1)$ derivations.

\begin{align}
    \frac{1}{[A]_t}=kt+\frac{1}{[A]_0}\\
    \frac{1}{[A]_0}=kt_{\frac{1}{2}}+\frac{1}{[A]_0}\\
    kt_{\frac{1}{2}}=\frac{1}{[A]_0}\implies t_{\frac{1}{2}}=\frac{1}{k[A]_0}
\end{align}

When plotting this, we can observe that the half life doubles after each iteration. So we can conclude that if $t_{\frac{1}{2}}=2^n\implies R\in O(n^2)$.

\end{document}
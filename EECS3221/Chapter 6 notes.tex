\documentclass[12pt]{book}

\usepackage[]{amsmath}
\usepackage[]{amsthm}
\usepackage[]{amsfonts}
\usepackage[]{amssymb}
\usepackage{blindtext}
\usepackage{pgfplots}
\usepackage[a4paper, total={6in, 8in}]{geometry}

\title{Chapter 6::Chemical equilibrium}

\author{Jerry Wu}

\date{\today}

\begin{document}
\maketitle

\chapter*{Gibbs energy}
\subsection*{Abstract}
We would like to determine the spontaneity of a reaction mixture to approach equilibrium, along with deriving the thermodynamic equlibrium constant $K_P$ and equilibrium concentraations of reactants and products in a mixture of reactive ideal gases. We define Gibbs energy ($\Delta G$), which expresses spontaneity in terms of properties of the system alone.\\

Suppose we have an arbitrary system $\sigma$ in thermal equilibrium with the surroundings. What determines the directions of a spontaneous change? Well, it would be the tendency of a system to minimize energy.

\begin{align*}
    -dq_{\sigma}=dq_{surr}\implies dS_{\Omega}+dS_{surr}\geq 0\implies dS_{\sigma}+\frac{dq_{surr}}{T}\geq 0
\end{align*}

\subsection*{Definition}
\begin{align*}
    \Delta G_{\sigma}=\Delta H_{\sigma}-T\Delta S_{\sigma}\implies dG_{\sigma}=dH_{\sigma}-TdS_{\sigma}
\end{align*}

If we graph this relation, we notice that $\Delta G$ is a function of $T$. $\Delta H$ would be $\Delta G_0$ and the slope would be $\Delta S$.

Gibbs energy holds the following properties::

\begin{itemize}
    \item $\Delta G<0\implies \sigma\equiv spont\implies exergonic$
    \item $\Delta G>0\implies \sigma\equiv \lnot spont\implies endergonic$
    \item Gibbs energy is an extensive property, and is a state function.
\end{itemize}

When a problem concerns Gibbs energy, we only care about \textbf{what happens in the system, not the surroundings}.

\subsection*{Differential forms of $U$, $H$, and $G$}
For an infinitessimal process::
\begin{align*}
    G=H+TS\implies dG=dH-TdS-SdT
\end{align*}

According to the first law::
\begin{align*}
    dH=dU+PdV+VdP\implies dU=dq+dw\implies dU=dq-PdV
\end{align*}

For a reversible process::
\begin{align*}
    dq_{rev}=TdS, dU=TdS-PdV
\end{align*}

So,

\begin{align*}
    dH=TdS+VdP,dG=VdP-SdT, dU=TdS-PdV
\end{align*}

This only applies for \textbf{expansion} work.

\subsection*{Example}

Calculate the maximum nonexpansion work that can be gained from the combustion of
benzene $(l)$ and of $H_2$ (g) on a per gram and a per mole basis under standard conditions. Is it
apparent from this calculation why fuel cells based on hydrogen oxidation are under
development for mobile applications? Thermodynamic values available in appendix.

\begin{align*}
    R:C_6H_6(l)+O_2(g)\rightarrow 6CO_2(g)+3H_2O(l)
\end{align*}

\textbf{To calculate maximum nonexpansion work, simply calculate} $\Delta G$

\begin{align*}
    \Delta G=3\Delta G(H_2O,l)+6\Delta G(CO_2,g)-\Delta G(G_6H_6,l)\approx 3202.2kJ
\end{align*}

The above value is per mole, so we need to multiply by molar mass ($\frac{1mol}{78.18g}$). In the end, we get $-40.99 \frac{kJ}{g}$

For formation of water, we have that

\begin{align*}
    \Delta G=\Delta G(H_2O,l)=-237.1kJmol^{-1}\implies -237.1kJmol^{-1}\left(\frac{1mol}{2g}\right)=-117.6kJg^{-1}
\end{align*}

\subsection*{Helmholtz energy and spontaneity}

This property is derived for processes occuring at $V\equiv const \land T\equiv const$. We use $A$ to represent the quantity.

\begin{align*}
    A=U-TS
\end{align*}

Gibbs free energy gives \textbf{maximum nonexpansion work}, whereas Helmholtz energy gives \textbf{maximum overall work.}

\subsection*{Exact differentials and Maxwell relations}

Take a function $z=f(x,y)$ where $dz=M(x,y)dx+N(x,y)dy$

Recall that by Euler's theorem, a function is a state function if and only if::

\begin{align*}
    \left(\frac{\partial M}{\partial y}\right)_x=\left(\frac{\partial N}{\partial x}\right)_y
\end{align*}

The total differentials for $U(S,V), H(S,P), G(T,P)$ can be modelled as follows::

\begin{align*}
    dU=TdS-PdV=\frac{\partial U}{\partial S}dS+\frac{\partial U}{\partial V}dV|T=\frac{\partial U}{\partial S}, -P=\frac{\partial U}{\partial V}
\end{align*}

\begin{align*}
    dH=TdS+VdP=\frac{\partial H}{\partial S}dS+\frac{\partial H}{\partial P}dP|T=\frac{\partial H}{\partial S},V=\frac{\partial H}{\partial P}
\end{align*}

\begin{align*}
    dG=-SdT+VdP=\frac{\partial G}{\partial T}dT+\frac{\partial G}{\partial P}dP|-S=\frac{\partial G}{\partial T}, V=\frac{\partial G}{\partial P}
\end{align*}

Because $dU$ is an exact differential, 

\begin{align*}
    \frac{\partial}{\partial V}\left(\frac{\partial U(S,V)}{\partial S}\right)=\frac{\partial}{\partial S}\left(\frac{\partial U(S,V)}{\partial V}\right)
\end{align*}

It would follow that::

\begin{align*}
    \frac{\partial T}{\partial V}=-\frac{\partial V}{\partial S}
\end{align*}

\subsection*{Example}

Derive the maxwell relation $\left(\frac{\partial T}{\partial V}\right)_S=-\left(\frac{\partial P}{\partial S}\right)_V$

Because $S$ and $V$ are constant, we can use the exact differential::

\begin{align*}
    dU=TdS-PdV
\end{align*}

Now we can form the mixed derivative::

\begin{align*}
    \left(\frac{\partial}{\partial V}\left(\frac{\partial U(S,V)}{\partial S}\right)_V\right)_S=\left(\frac{\partial}{\partial S}\left(\frac{\partial U(S,V)}{\partial V}\right)_S\right)_V
\end{align*}

Substituting $dU=TdS-PdV$, we have that

\begin{align*}
    \left(\frac{\partial}{\partial V}\left(\frac{\partial\left[TdS-PdV\right]}{\partial S}\right)_V\right)_S=\left(\frac{\partial}{\partial S}\left(\frac{\partial\left[TdS-PdV\right]}{\partial V}\right)_S\right)_V
\end{align*}

After simplifying, we can conclude that

\begin{align*}
    \left(\frac{\partial T}{\partial V}\right)_S=-\left(\frac{\partial P}{\partial S}\right)_V\blacksquare
\end{align*}

\subsection*{Dependence of $\Delta G$ on $T$}

Recall that $dG=VdP-SdT$. At $P\equiv const,dG=-SdT$. When $P\equiv const$, it also follows that

\begin{align*}
    \frac{\partial G}{\partial T}=-S, S\geq 0
\end{align*}

So it would follow that Gibbs energy \textbf{decreases as energy increases}.

The Gibbs-Helmholtz equation states that for a finite process,

\begin{align*}
    \frac{\partial \left(\frac{\Delta G}{T}\right)}{T}=-\frac{\Delta H}{T^2}
\end{align*}

Assume that $\Delta H\land \Delta S$ are independent of T and that $P=1atm$. We can integrate both sides of the equation to give us that

\begin{align*}
    \int d\left(\frac{\Delta G}{T}\right)=\int\Delta Hd\frac{1}{T}
\end{align*}

\begin{align*}
    \frac{\Delta G(T)}{T}-\frac{\Delta G(298)}{298}\approxeq \left(\frac{1}{T}-\frac{1}{298}\right)\Delta H(298)
\end{align*}

So in the end, we have that the Gibbs-Helmholtz equation can be approximated as::

\begin{align*}
    \frac{\Delta G(T_2)}{T_2}-\frac{\Delta G(T_1)}{T_1}\approxeq \left(\frac{1}{T_2}-\frac{1}{T_1}\right)\Delta H(T_1)
\end{align*}

\subsection*{Dependence of $\Delta G$ on $P$}

Recall that $dG=VdP-SdT$, then at $T\equiv const$, we have that $dG=VdP\implies \frac{\partial G}{\partial P}=V$.\\

We model dependence of $\Delta G$ on $P$ as follows::

\begin{align*}
    \Delta G=\int_{1}^{2}dG=G_2-G-1=\int_{P_1}^{P_2}VdP
\end{align*}

If $V\equiv const$ and the process concerns liquids and solids, we have that::

\begin{align*}
    \Delta G=G(P_2)-G(P_1)=V(P_2-P_2)
\end{align*}

For $g\in \mathbb{C}$, we have that::

\begin{align*}
    \Delta G=G_2-G_1=\int_{P_1}^{P_2}\frac{nRT}{P}dP=nRT\ln\left(\frac{P_2}{P_1}\right)
\end{align*}

Now set $P_1=1bar$ and set $P_2=P$, therefore,

\begin{align*}
    G=G^{\circ}+nRTln(P)\implies \bar{G}=\bar{G^{\circ}}+RT\ln(P)
\end{align*}

If $\forall products,reactants \in \sigma((products\equiv ((s)\lor (l)))\lor (reactants \equiv((s)\lor (l))))$, then

\begin{align*}
    \Delta G=G(P_2)-G(P_1)=V(P_2-P_1)=V\Delta P
\end{align*}

This is not very important, since we can just ignore the quantity altogether.\\

However, if $\exists product\lor reactant\in \sigma(product\lor reactant\equiv (g))$, then the volume of solids and liquids is ignored, and::

\begin{align*}
    \Delta G=G(P_2)-G(P_1)=\Delta nRT\ln\left(\frac{P_2}{P_1}\right)
\end{align*}

\subsection*{Hess's law for Gibbs energy}
It's the same thing as enthalpy.

\subsection*{Example}

For the synthesis of urea::

\begin{align*}
    R:CO_2(g)+2NH_3(g)\rightarrow (NH_2)_2CO(s)+H_2O(l)
\end{align*}

\begin{itemize}
    \item[a)] Calculate $\Delta_r G$ for the reaction at $298K$ and $1 bar$ (Ans: $-6.8 kJ$)
    \item[b)] Assume $g\in \mathbb{C}$, calculate $\Delta_r G$ at $P=10bar$.
\end{itemize}

To calculate $\Delta G(10 bar)$, we can plug in the values into the equation we derived.

\begin{align*}
    \Delta G(10 bar)=\Delta G(1 bar)+\Delta nRT\ln\left(\frac{P_2}{P_1}\right)
\end{align*}

\begin{align*}
    =(6800Jmol^{-1})+(-3)(8.314JK^{-1}mol^{-1})(298K)(\ln\left(\frac{10}{1}\right)=-23.9kJmol^{-1})
\end{align*}

So it would suffice to say that a reaction at higher pressure is \textbf{more spontaneous} than at lower pressure.

\subsection*{Equilibrium}

The following equation provides the relationship between $\Delta G_T$ and $\Delta G_T^{\circ}$ at partial pressures $P=1bar$. When $\Delta G=0$, the system is at equilibrium.

\begin{align*}
    \Delta G_T=\Delta G_T^{\circ}+RT\ln\left(\frac{\left(\frac{P_B}{P^{\circ}}\right)^b}{\left(\frac{P_A}{P^{\circ}}\right)^a}\right)
\end{align*}

All chemical reactions are reversible, i.e. $\forall R, R_1\rightarrow R_2\iff R_2\rightarrow R_1$ since they can go both ways. Reactions that proceed \textbf{nearly} to completion. Often times, the reverse of these reactions are too slow to detect experimentally.

The \textbf{equilibrium point} is the point in time where a concentration of a reactant or product in the system stops changing.

\subsection*{Reaction rates}

When a system is at equilibrium, the forward and reverse reactions occur at the same rate, i.e. for gases, $\frac{\Delta P_1}{\Delta T}=\frac{\Delta P_2}{\Delta T}=0$. For solids and liquids, use concentration instead.

\subsection*{The equilibrium constant}

The equilibrium constant of a reaction is represented as a ratio of concentrations.

\begin{align*}
    K_c=\frac{\prod_{\left[E\right]\in rxt} \left[E\right]_i}{\prod_{\left[E\right]\in prod} \left[E\right]_j}|i,j\in \mathbb{N}
\end{align*}

For $i$ reactants and $j$ products. If we have multiple reactions taking place with distinct equilibrium constants, we can multiply them together to find the overall equilibrium constant. Say we have $R_1,R_2\in \sigma$ each with their own $K_1,K_2$, then $K_3=K_1K_2$.

For gas exclusive systems, we have that::

\begin{align*}
    K_P=K_C (RT)^{\Delta n}
\end{align*}

\subsection*{Thermodynamic equilibrium constant (unitless)}

In thermodynamics we want to relate equilibrium to reaction spontaneity. This would require us to have a unitless description of the equilibrium constant using a concept called activity. Activity is measured as the standard deviation from the sandard state.

\begin{align*}
    a_X=\frac{\left[E\right]}{1M}, a_X=\frac{P_X}{1bar}
\end{align*}

For solutes and gases respectively. For pure liquids and solids, $a_X=1$. For an arbitrary reaction,

\begin{align*}
    bB+cC\leftrightarrows dD+eE|K=\frac{a_D^d a_E^e}{a_B^b a_C^c}\implies K=\frac{\prod a_{prod}}{\prod a_{rxt}}
\end{align*}

Recall that if $K\rightarrow \infty$, the system would contain mostly products, and if $K\rightarrow 0$, the system would contain mostly products.

\subsection*{$\Delta G$ of a reaction mixture}

In a reaction, $G$ would depend on the number of moles $\forall \zeta\in \sigma$, i.e.

\begin{align*}
    G=G(T,P,n_1,n_2\ldots n_i)|i=|\sigma|\in \mathbb{N}
\end{align*}

The chemical potential of the $i$th element in the reaction can be expressed as $\mu_i=\frac{\partial G}{\partial n_j}, i\in \mathbb{N}$

Once concentration equilibrium has been reached with respect to the reactants, we have that::

\begin{align*}
    \mu_{pure}(H_2)=\mu_{mixture}(H_2)
\end{align*}

So,

\begin{align*}
    dG=\frac{\partial G}{\partial T}dT+\frac{\partial G}{\partial P}dP+\sum_{i\in \mathbb{N}} \mu_i dn_i
\end{align*}

With respect to temperature and pressure,

\begin{align*}
    \mu_A^{mixture}(T,P)=\mu_A^{pure}(T,P)+RT\ln(\chi_A)
\end{align*}

For an arbitrary gaseous reaction, we have that

\begin{align*}
    R:aA(g)+ bB(g)\leftrightarrows cC(g)+dD(g)|\Delta G_R=\Delta G_R^{\circ}+RT\ln(Q)
\end{align*}

The $Q$ value is a ratio of pressures::

\begin{align*}
    Q=\frac{\left(\frac{P_C}{P^{\circ}}\right)^c\left(\frac{P_D}{P^{\circ}}\right)^d}{\left(\frac{P_A}{P^{\circ}}\right)^a\left(\frac{P_B}{P^{\circ}}\right)^b}
\end{align*}

\subsection*{What is the point?}

$Q$ can tell us how reaction concentrations must change in order to reach equilibrium (where $Q=K$). In summary,

\begin{itemize}
    \item $Q_C<K_C\implies rxts\rightarrow prods$
    \item $Q_C>K_C\implies prods\rightarrow rxts$
    \item $Q_C=K_C\implies \varnothing$
\end{itemize}

\subsection*{Example}

Take the reaction for the formation of ethylene from carbon and hydrogen at $25^{\circ}C$. $P_{H_2}=100atm, P_{C_2H_4}=0.1atm$

\begin{align*}
    2C(s)+2H_2(g)\leftrightarrows C_2H_4(g)
\end{align*}

So here,

\begin{align*}
    Q_P=\frac{P_{C_2H_4}}{P_{H_2}^2}, \Delta G=68.1+(8.314)(298)(\ln\left(\frac{0.1}{100^2}\right)=39.6 kJmol^{-1}
\end{align*}

\subsection*{Relating $\Delta G$ to equilibrium}

Suppose we have a reaction

\begin{align*}
    R:aA\leftrightarrows bB
\end{align*}

Suppose $G$ changes by a small amount ($\delta G$) and the reaction progresses by a small amount ($\delta \xi$). We have that the change in $G$ is::

\begin{align*}
    dG=\mu_A dn_A+\mu_B dn_B=-\mu_A d\xi +\mu_B d\xi=(\mu_A-\mu_B)d\xi\implies \Delta G=\frac{\delta G}{\delta \xi}=\mu_B-\mu_A
\end{align*}

in $kJmol^{-1}$.

So let us summarize,

\begin{itemize}
    \item $\mu_A>\mu_B\implies R\equiv spont(\rightarrow)$
    \item $\mu_A=\mu_B\implies equilibrium$
    \item $\mu_B>\mu_A\implies R\equiv spont(\leftarrow)$
\end{itemize}

We can rewrite the equation as::

\begin{align*}
    \Delta G_T=-RT\ln(K)+RT\ln(Q)=RT\ln\left(\frac{Q}{K}\right)
\end{align*}

\subsection*{Chemical equilibria of ideal gases (ice tables)}

Suppose $1.00 mol$ of $H_2$ and $2.00 mol$ of $I_2$ are placed in a $1.00L$ vessel. How many
moles of substances are in the gaseous mixture when it comes to equilibrium at
$458^{\circ}C$? The equilibrium constant $K_c$ at this temperature is $49.7^{\circ} C$.

\subsection*{Solution}

We have that $[H_2]=1M$ and $[I_2]=2M$ since volume is $1L$.

\begin{align*}
    R:H_2(g)+I_2(g)\leftrightarrows 2HI(g)
\end{align*}

\begin{align*}
    K_c=\frac{2x^2}{(1-x)(2-x)}=49.7\implies 0.92x^2-3x+2=0
\end{align*}

\begin{align*}
    x=\frac{-b\pm\sqrt{b^2-4ac}}{2a}=\frac{3\pm\sqrt{9-4(0.92)(2)}}{2(0.92)}\approx 1.63\pm 0.7
\end{align*}

Since $x_1=2.33$ would give us a negative concentration, we reject it. So we can plug in $x_2=0.93$ to check if our values work::

\begin{align*}
    K_c=\frac{1.86^2}{(0.07)(1.07)}=49.7
\end{align*}

It is indeed the same number, so our solution is correct.

\subsection*{The Van't Hoff equation}

We want to find a relationship between $K$ and $T_2$ if $K$ is known for $T_1$

\begin{align*}
    \Delta_r G^{\circ}=\Delta_r H^{\circ}-T\Delta_r S\\
    \Delta_r G^{\circ}=-RT\ln(K)\\
    -RT\ln(K)=\Delta_r H^{\circ}-T\Delta_r S\\
    \ln(K)=\frac{-\Delta H}{RT}+\frac{\Delta S}{R}
\end{align*}

So we have found a relationship for $K_1, T_1$ and $K_2, T_2$

\begin{align}
    \ln(K)=-\frac{\Delta H}{RT}+\frac{\Delta S}{R}
\end{align}

\begin{align}
    \ln\left(\frac{K_2}{K_1}\right)=-\frac{\Delta H}{R}\left(\frac{1}{T_2}-\frac{1}{T_1}\right)
\end{align}

\subsection*{Example}

A polypeptide can exist in either the helical or random coil forms. The
equilibrium constant for equilibrium reaction of the helix to random coil
transition is $0.86$ at $40 ^{\circ}C$ and $0.35$ at $60^{\circ}C$. Calculate values of $\Delta H$ and $\Delta S$ for
the reaction.

\subsection*{Solution}

Assuming $\Delta H$ and $\Delta S$ are independent of $T$, then we can use the equation

\begin{align*}
    \ln\left(\frac{0.35}{0.86}\right)=-\frac{\Delta H}{333K}\left(\frac{1}{333K}-\frac{1}{313K}\right)\implies \Delta H=-3.9E4 kJmol^{-1}
\end{align*}

Use the values at $40^{\circ} C$ to calculate $\Delta G$ and $\Delta S$

\begin{align*}
    \Delta G=-RT\ln(K)=-(8.314)(313)\ln(0.86)=391Jmol^{-1}\\
    \Delta S=\frac{\Delta H-\Delta G}{T}=\frac{-3.9E4-392}{313K}=-1.3E2JK^{-1}mol^{-1}
\end{align*}

\subsection*{Example 2}

If the reaction $R:FeN_2(s)+\frac{3}{2}H_2(g)\rightarrow 2FE(s)+NH_3(g)$ comes to equilibrium at a total pressure of $1 bar$, analysis of the gas shows that at
$700K$ and $800K$, $\frac{P(NH_3)}{P(H_2)} = 2.165$ and $1.083$, respectively, if only $H_2(g)$ was initially
present in the gas phase and $Fe_2N(s)$ was in excess.

\begin{itemize}
    \item[a)] Calculate $K_P$ at $700K$ and $800K$.
    We know that if $P_{\Sigma}=1bar$, then $P_{NH_3}+P_{H_2}=1bar$. So at 700K,

    \begin{align*}
        1atm=2.165P_{H_2}+P_{H_2}=3.165P_{H_2}\\
        P_{H_2}=0.316atm, P_{NH_3}=0.684atm\\
        K_P(700K)=\frac{0.684}{(0.316)^{\frac{3}{2}}}=3.85
    \end{align*}

    At 800K,

    \begin{align*}
        1atm=1.083P_{H_2}+P_{H_2}=2.083P_{H_2}\\
        P_{H_2}=0.480atm, P_{NH_3}=0.520atm\\
        K_P(800K)=\frac{0.520}{(0.480)^{\frac{3}{2}}}=1.56
    \end{align*}
    \item[b)] Calculate $\Delta_r S^{\circ}$ at $700K$ and $800K$ and $\Delta_R H^{\circ}$ assuming it is independent of temperature.
    
    \begin{align*}
        \ln\left(\frac{K_P(800)}{K_P(700)}\right)=-\frac{\Delta H}{R}\left(\frac{1}{800}-\frac{1}{700}\right)
    \end{align*}

    After some plugging in and calculations, we have that::

    \begin{align*}
        \Delta H=-42.1kJmol^{-1}
    \end{align*}

    To find $\Delta S$, we need $\Delta G$ first.

    \begin{align*}
        \Delta G(700K)=-RT\ln(K_P(700K))=-7.81kJmol^{-1}
    \end{align*}
    \begin{align*}
        \Delta G(800K)=-RT\ln(K_P(800K))=-2.91kJmol^{-1}
    \end{align*}

    Now we can calculate $\Delta S$.

    \begin{align*}
        \Delta S(700K)=\frac{\Delta H(700)-\Delta G(700)}{700K}=48.9jmol^{-1}K^{-1}=\Delta S(800K)
    \end{align*}

    Note that the values of $\Delta S$ at $700K$ and $800K$ are nearly equal because $|\Delta G_r|\llless|\Delta H_r|$

    \item[c)] Calculate $\Delta G_r$ at $298K$

    \begin{align*}
        ln(K_P(298.15K))=ln(K_P(700K))-\frac{\Delta H}{R}\left(\frac{1}{298}-\frac{1}{700}\right)=11.1
    \end{align*}

    So now we can calculate our $\Delta G$.
    
    \begin{align*}
        \Delta G=-RT\ln(K_P(298K))=-(8.314)(298.15)(11.1)=-27500Jmol^{-1}=-27.5kJmol^{-1}
    \end{align*}
\end{itemize}

\subsection*{Le Chatelier's principle}

We want to know what happens when we perturb a system that is at equilibrium. Changing the amount of reactants and products in addition to changing volume, pressure, and temperature are all different ways to perturb the system. In general, 

\begin{itemize}
    \item The concentration stress of an \textbf{added reactant} or product is relieved by \textbf{consuming} the added substance.
    \item The concentration stress of a \textbf{removed} reactant or product is relieved by \textbf{replenishing} the removed substance
\end{itemize}

A special case is if we add \textbf{inert} gases to the system. Then no changes will occur because no reactions will be taking place to increase/decrease pressure/volume.

\end{document}
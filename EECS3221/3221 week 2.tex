\documentclass[12pt]{book}

\usepackage[]{amsmath}
\usepackage[]{amsthm}
\usepackage[]{amsfonts}
\usepackage[]{amssymb}
\usepackage{blindtext}
\usepackage[a4paper, total={6in, 8in}]{geometry}

\usepackage{listings}
\usepackage{color}

\definecolor{dkgreen}{rgb}{0,0.6,0}
\definecolor{gray}{rgb}{0.5,0.5,0.5}
\definecolor{mauve}{rgb}{0.58,0,0.82}

\lstset{frame=tb,
  language=Python,
  aboveskip=3mm,
  belowskip=3mm,
  showstringspaces=false,
  columns=flexible,
  basicstyle={\small\ttfamily},
  numbers=left,
  numberstyle=\small\color{black},
  keywordstyle=\color{blue},
  commentstyle=\color{dkgreen},
  stringstyle=\color{mauve},
  breaklines=true,
  breakatwhitespace=true,
  tabsize=4
}

\title{EECS3221 notes}
\author{Jerry Wu}
\date{Week II 2023/05/18}

\begin{document}
\maketitle
\chapter*{Operating system structures}

\section*{Services}

\begin{itemize}
    \item Operating systems provide an environment for execution of programs and services to programs and users.
    \item Some other OS services include but are not limited to::
    \begin{itemize}
        \item \textbf{UI} - Almost all operating systems have a user interface (cmd, GUI, touch screen, etc)
        \item \textbf{Program execution} - The system must be able to load a program into memory and run the program, end execution either normally or abnormally (indicating an error)
        \item \textbf{IO operations} - A running program may require IO which may involve a file or an IO device
        \item \textbf{File system manipulation} - Programs need to read and write files and directories, create and delete them, search, list file info, and permission management
        \item \textbf{Communications} - Processes may exchange information on the same computer or between computers over a network
        \begin{itemize}
            \item Communications may be via shared memory or through message passing (packets moved by the OS)
        \end{itemize} 
        \item \textbf{Error detection} - OS needs to be constantly aware of possible errors
        \begin{itemize}
            \item May occur in the CPU and memory, IO devices, or in software
            \item For each type of error, OS should take the appropriate action to ensure correct and consistent computing
            \item Debugging facilities can greatly enhance the user's and programmer's abilities to efficiently use the system
        \end{itemize}
        \item \textbf{Resource allocation}
        \item \textbf{Security}
    \end{itemize}
    \item Our goal is to develop an understanding of OS in order to build scalable software
\end{itemize}

\subsection*{CLI (command line interpreter)}

CLI allows direct command entry closer to the hardware which is much faster than a GUI

\begin{itemize}
    \item Sometimes implemented in the kernel, sometimes by system programs
    \item Sometimes multiple flavours implemented (shells)
    \item Primarily fetches a command from the user and executes
    \item There are multiple build in commands, and some are just names of system programs
    \begin{itemize}
        \item If the latter, adding new features doesn't require shell modification
    \end{itemize}
\end{itemize}

\subsection*{GUI}

First invented by Xerox. Made for ease of use for the user so they won't have to use commands.

\begin{itemize}
    \item Controlled by keyboard and mouse with a monitor for viewing, or touch screens
    \item Icons represent files, programs, actions, etc
    \item Various mouse buttons over objects in the interface cause actions (providing information, options, function execution, opening directories (folders))
\end{itemize}

\section*{System calls}

\begin{itemize}
    \item Programming interface to the services provided by the OS
    \item Typically written in a high level language (C or C++)
    \item Mostly accessed by programs via a high level API rather than direct system call use
    \item Three most common APIs are Win32 for windows, POSIX for UNIX, mac, Linux, etc. and the Java API for the JVM.
\end{itemize}

\subsection*{Implementation}

\begin{itemize}
    \item Each system call is associated with a number
    \begin{itemize}
        \item \textbf{System call interface} maintains a table indexed according to said numbers
    \end{itemize}
    \item The system all interface invokes the intended system call in the OS kernel and returns the status of the system call and any return values
    \item To make system calls, we need to pass parameters into the OS
    \begin{itemize}
        \item Simplest: pass parameters in registers (sometimes there are more parameters than there are registers)
        \item Pass in blocks, or table in memory and address of the block passed as a parameter in a register (Linux/solaris)
        \item Parameters are pushed and popped from the program stack
    \end{itemize} 
\end{itemize}

\subsection*{Types of system calls}
System call API cheat sheet on \textbf{slide 2.24}
\begin{itemize}
    \item Create/terminate processes
    \item end, abort, load, execute
    \item get process attributes, set process attributes
    \item wait for time/event/signal event
    \item allocate and free memory
    \item dump memory if error
    \item \textbf{debuggers} for determining bugs: \textbf{single step} execution
    \item \textbf{locks} for managing access to shared date between processes
\end{itemize}

\section*{Example:: Arduino}

\begin{itemize}
    \item Single task (one program at a time)
    \item No on board OS
    \item Programs (sketch) loaded via USB stick

\end{itemize}

\section*{System services}
System programs provide a convenient environment for program development and execution. They can be divided into::

\begin{itemize}
    \item File manipulation/management (text editors, file manager, etc)
    \item Status information sometimes stored in a file(s) (task manager)
    \item Programming language support (GCC, JDK, etc)
    \item Program loading and execution
    \item Communications
    \item Background services (daemons)
    \item Application programs (everything else)
\end{itemize}

Most users' view of the OS is deined by system programs and not the actual system calls.

\section*{Linkers and loaders}

\begin{itemize}
    \item Source code compiled into object files designed to be loaded into any physical memory location - \textbf{relocatable object file}
    \item \textbf{Linker} combines these into a single binary executable file.
    \item 
\end{itemize}

\end{document}
\documentclass[12pt]{book}

\usepackage[]{amsmath}
\usepackage[]{amsthm}
\usepackage[]{amsfonts}
\usepackage[]{amssymb}
\usepackage{blindtext}
\usepackage[a4paper, total={6in, 8in}]{geometry}
\usepackage{graphicx}
\usepackage{listings}
\usepackage{color}
\usepackage{array}

\definecolor{dkgreen}{rgb}{0,0.6,0}
\definecolor{gray}{rgb}{0.5,0.5,0.5}
\definecolor{mauve}{rgb}{0.58,0,0.82}

\pagenumbering{arabic}
\renewcommand{\chaptername}{Lecture}
\let\cleardoublepage\clearpage

\lstset{frame=tb,
  language=Java,
  aboveskip=3mm,
  belowskip=3mm,
  showstringspaces=false,
  columns=flexible,
  basicstyle={\small\ttfamily},
  numbers=left,
  numberstyle=\small\color{black},
  keywordstyle=\color{blue},
  commentstyle=\color{dkgreen},
  stringstyle=\color{mauve},
  breaklines=true,
  breakatwhitespace=true,
  tabsize=4
}

\title{EECS4313 week 2}
\author{Jerry Wu}
\date{2024-01-15}

\begin{document}
\maketitle
\tableofcontents

\chapter{Reporting and analyzing bugs}

\section*{Bug reporting}
\subsection*{Writing an effective bug report}
\begin{itemize}
    \item Explain the problem and how to reproduce the problem
    \item Analyze the problem so that it can be described with a \textbf{minimum} number of steps
    \item Write a report that is complete, easy to understand, and non-antagonistic
\end{itemize}

\begin{quote}
    \textit{"Be nice" - H.V. Pham 2024}
\end{quote}

\subsection*{What kind of errors to report?}

\begin{itemize}
    \item \textbf{Coding error} - The program does not do what the programmer would expect it to do; general bugs (pull request)
    
    \item \textbf{Design issue} - The program does what the programmer intended, but a reasonable customer would be confused or unhappy with it
    
    \item \textbf{Requirements issue} - The program is well designed and well implemented, but it won't meet one of the customer's requirements
    
    \item \textbf{Documentation/code mismatch} - Report this to the programmer (via bug report/github issues) and to the writer (via memo or a comment on the manuscript)
    
    \item \textbf{Specification/code mismatch} - Sometimes, the specs are right, sometimes the code is right and the specs should be changed
\end{itemize}

\section*{Making effective bug reports}

\begin{quote}
    \textit{"you are not the enemy of the programmer" - H.V. Pham 2024}
\end{quote}

\begin{itemize}
    \item Bug reports store all information needed to document, report and fix problems occuring in a software system
    \item Bug reports are like a pitch to the devs. Good bug reports will "sell" the developer the idea of spending their time and energy to fix said bug (incentive)
    \item Bug reports are the \textbf{primary work product} of a tester. This is what people outside of the testing group will notice and remember most about your work. In other words, include as much detail as possible
    \item The best tester is not the one who finds the most bugs or who embarrasses the most programmers, but the one who \textbf{gets the most bugs fixed}
    \item The primary goal at the end of the day is to work together with the developer to get the system running properly
\end{itemize}

\subsection*{Selling a bug in a bug report}

\begin{itemize}
    \item Time is in short supply, so if you want to convince the dev to spend their time fixing your bug, you have to sell it to them
    \item Selling revolves around two fundamental objectives:
    \begin{itemize}
        \item Motivate the buyer (make them \textbf{want} to fix the bug)
        \item Overcome objections (get past their \textbf{excuses} and reasons for not fixing the bug)
    \end{itemize} 
\end{itemize}

\subsection*{Motivating the bug fix}

\begin{itemize}
    \item When you run a test and find a failure, you are looking at a sympton and not the underlying fault. You may or may not have foun the best example of a failure that can be caused by the underlying fault.
    
    \item $\therefore$ you should do some follow up work to try to prove that a fault:
    \begin{itemize}
        \item is more serious than it first appears
        \item is more general than it first appears
        \item affects more versions of the software
    \end{itemize} 

    \item These things will often motivate a developer to fix a bug:
    \begin{itemize}
        \item The bug looks really bad
        \item It looks like an interesting puzzle and piques the programmer's curiosity
        \item It will affect lots of people
        \item The problem is trivial
        \item It has embarrassed the company, or a bug like it embarrassed a competitor
        \item Management (that is, someone with influence) has said that they really want it fixed
    \end{itemize} 

    \item As soon as you run into a problem in the software, fill out a \textbf{problem report form}. In a well written report, you:
    \begin{itemize}
        \item \textbf{Explain the problem} and how to reproduce the problem
        \item \textbf{Analyze the error} so you can describe it in a minimum number of steps
        \item Include all the steps
        \item Make the report easy to understand
        \item Keep your tone neutral and non antagonistic
        \item \textbf{Keep it simple}, report one bug per report
        \item If a sample test file is essential to reproducing a problem, reference it and attach the test file.
    \end{itemize} 
\end{itemize}

\subsection*{Problem report form outline}

A typical report form includes some of the following fields

\begin{itemize}
    \item \textbf{Problem report number} - unique number assigned to the report
    \item \textbf{Reported by} - author of the report
    \item \textbf{Date reported} - self explanatory
    \item \textbf{Program/component name} - the visible item/class under test
    \item \textbf{Release number} - self explanatory
    \item \textbf{Version/build identifier} - like version C or 20000802a
    \item \textbf{Configuration(s)} - h/w and s/w configurations under which the bug was found and replicated
    \item \textbf{Report type} - coding error, design issue, documentation mismatch, etc.
    \item \textbf{Can reproduce} - yes/no
    \item \textbf{Severity} - assigned by the tester. Some variation of small/medium/large
    \item \textbf{Priority} - assigned by the programmer/project manager
    \item \textbf{Problem summary} - self explanatory
    \item \textbf{Keywords} - used for searching for open reports in the project, anyone can add keywords at any time
    \item \textbf{Problem desc and reproduction steps}
    \item \textbf{Suggested fix} - don't do this unless you know for sure
    \item \textbf{Status} - Tester fills this in (open/closed/resolved/re-opened)
    \item \textbf{Resolution} - The project manager owns this field, common resolutions include:
    \begin{itemize}
        \item \textbf{Pending} - the bug is currently being worked on
        \item \textbf{Fixed} - the dev says it is fixed, so the tester should check it again
        \item \textbf{Cannot reproduce} - the dev was unable to reproduce the issue. Add more details and notify the dev
        \item \textbf{Deferred} - we'll fix this later (not)
        \item \textbf{As designed} - the program works as intended
        \item \textbf{Need info} - not enough information provided
        \item \textbf{Duplicate} - the same issue has been brought up already
        \item \textbf{Withdrawn} - tester has withdrew the report
    \end{itemize}

    \item \textbf{Resolution version} - build identifier
    \item \textbf{Resolved by} - name of programmer, project manager, tester, etc
    \item \textbf{Resolution tested by} - original tester, or a tester if the originator was not a tester
    \item \textbf{Change history} - date stamped list of all changes to the record including name and fields changed
    %%finish this later

\end{itemize}

\chapter{Open research questions on bug management}



\end{document}
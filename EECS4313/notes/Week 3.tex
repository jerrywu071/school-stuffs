\documentclass[12pt]{book}

\usepackage[]{amsmath}
\usepackage[]{amsthm}
\usepackage[]{amsfonts}
\usepackage[]{amssymb}
\usepackage{blindtext}
\usepackage[a4paper, total={6in, 8in}]{geometry}
\usepackage{graphicx}
\usepackage{listings}
\usepackage{color}
\usepackage{array}

\definecolor{dkgreen}{rgb}{0,0.6,0}
\definecolor{gray}{rgb}{0.5,0.5,0.5}
\definecolor{mauve}{rgb}{0.58,0,0.82}

\pagenumbering{arabic}
\renewcommand{\chaptername}{Lecture}
\let\cleardoublepage\clearpage

\lstset{frame=tb,
  language=Java,
  aboveskip=3mm,
  belowskip=3mm,
  showstringspaces=false,
  columns=flexible,
  basicstyle={\small\ttfamily},
  numbers=left,
  numberstyle=\small\color{black},
  keywordstyle=\color{blue},
  commentstyle=\color{dkgreen},
  stringstyle=\color{mauve},
  breaklines=true,
  breakatwhitespace=true,
  tabsize=4
}

\title{EECS4313 week 3}
\author{Jerry Wu}
\date{2024-01-22}

\begin{document}
\maketitle
\tableofcontents

\chapter{Test automation}

\section{Why automated testing?}

\begin{itemize}
    \item Manual testing is time consuming and not economical
    \item Automation is \textbf{unattended} and \textbf{can happen overnight}; thus not requiring human intervention
    \item Automation increases the speed of test execution
    \item Manual testing is error prone due to its recurring nature
\end{itemize}

\subsection{Executing automated tests}

\begin{itemize}
    \item Test automation is the use of software separate from the software being tested to control the execution of tests
    \begin{itemize}
        \item Generating test inputs and expected results
        \item Running test suites without manual intervention
        \item Evaluating pass/no pass
    \end{itemize}

    \item Testing must be automated to be \textbf{effective and repeatable}
\end{itemize}


\subsection{What can be automated?}

Each one of these steps in software development can be automated, and each step down is easier to automate than the last. Of course, there has to be some human intervention to formulate the requirements for automation at the start.

\begin{itemize}
    \item \textbf{Analyze} - intellectual (performed once)
    \item \textbf{Design} - mostly intellectual
    \item \textbf{Construct} - clerical and intellectual
    \item \textbf{Testing} - mostly clerical
    \item \textbf{Deploy} - clerical (repeated many times)
\end{itemize}

The more clerical something is, the more worth automating it is.

\section{Record \& playback}

\begin{itemize}
    \item Usually for websites, mobile, UIs, etc.
    \item Test scripts are recorded on the initial version of the application
    \item The same scripts are executed on the next version
    \item The scripts need \textbf{some modification} (quite costly at high number of modification) for the changes happening on the application during every version of the application
    \item The test scripts repository keeps growing as the application goes through changes. This makes this kind of test suite very hard to maintain
\end{itemize}

\subsection{Procedure}
\begin{itemize}
    \item Automation tool generates scripts by recording user actions
    \item The generated scripts can be played back to reproduce the exact user actions
\end{itemize}

\subsection{Advantages}
\begin{itemize}
    \item Less effort for automation
    \item Quick returns
    \item Does not require expertise on tools
\end{itemize}

\subsection{Limitations}
\begin{itemize}
    \item High dependency on the GUI of AUT (application under test)
    \item Difficult to maintain the scripts
\end{itemize}

\subsection{Problems}

\begin{itemize}
    \item Very fragile
    \begin{itemize}
        \item A single change in UI can cause the whole system to break
    \end{itemize} 
    \item Hard to maintain
    \begin{itemize}
        \item An abundance of test scripts causes the test suite to be hard to maintain
    \end{itemize} 
    \item No modularity or reuse
    \begin{itemize}
        \item System must be ready before automation can start
    \end{itemize} 
\end{itemize}

\section{Testing terminology contd'}

\begin{quote}
    \textit{"You don't want to make assumptions. That is not unit testing" - H.V. Pham 2024}
\end{quote}

\begin{itemize}
    \item A \textbf{unit test} is a test of a single class/method
    \item A \textbf{test case} tests the respose of a single method to a particular set of inputs
    \item A \textbf{test suite} is a collection of test cases
    \item A \textbf{test runner} is software that runs tests and reports results
    \item A \textbf{flaky test} is a test case where it is sometimes triggered and sometimes not depending on specific conditions, seemingly randomly.
\end{itemize}

\newpage
\subsection{Std. structure of a JUnit test class}

\begin{itemize}
    \item Test a class called \texttt{Fraction}
    \item Create a test class called \texttt{FractionTest}
\end{itemize}

\begin{lstlisting}
import org.junit.;
import static org.junit.Assert.;

public class FractionTest{
    @Test
    public void addTest(){...}

    @Test
    public void testToString(){...}

    //useful if we have a counter during the tests to set in SetUp() and reset in TearDown()
    @Before
    public void SetUp(){...}

    @After
    public void TearDown(){...}
}
\end{lstlisting}


\subsection{More fixtures for test classes}
\begin{itemize}
    \item There is also a \texttt{@BeforeClass} annotation that will execute once before \textbf{all} test cases
    \item Similarly, there is also an \texttt{@AfterClass} annotation that executes a method after every test case
    \item Usually done for expensive allocations for resources like connecting/disconnecting a database.
\end{itemize}

\subsection{Assert method: Equal, Null and fail}

\begin{itemize}
    \item \texttt{assertEquals(Object expected, Object actual)} - Compare two objects to see if they're identical. Use \texttt{.equals(Object other)} for compare
    \item \texttt{assertArrayEquals(int[] expecteds, int[] actuals)} or \texttt{assertArrayEquals(long[] expecteds, long[] actuals)} - Compares two arrays
    \item \texttt{assertSame(Object expected, Object actual)} - asserts that two rerferences are the same object (using \texttt{==}). Useful for testing a find method for data structures
    \item \texttt{assertNull(Object object)} - asserts that a reference is not null
    \item \texttt{fail()} - causes the test to fail and throw an exception (\texttt{AssertionError}). Useful as a result of a complex test, or when testing for exceptions.
\end{itemize}

\subsection{Exception testing}

\begin{itemize}
    \item If a test case is expected to raise an exception, it can be written as follows:
\begin{lstlisting}
@Test(expected = Exception.class)
public void testException(){
    //some code that should raise an exception

    fail("Should raise an exception");
}
\end{lstlisting}

    \item if code doesn't throw an exception in this case, test fails. Else test passes.
\end{itemize}

\subsection{Assert statement}

\begin{itemize}
    \item \texttt{assert boolean\_condition;} - throws an \texttt{AssertionError} if \texttt{boolean\_condition == false}. Can be used in place of \texttt{assertTrue()} in JUnit.
\end{itemize}

\subsection{Ignoring test cases}
\begin{itemize}
    \item Test cases that are not finished being written yet can be ignored using the \texttt{@Ignore} annotation.
    \item JUnit will not execute the test cases marked with this annotation, and will report the number of test cases that were ignored
\end{itemize}

\section{Types of test generation}

\begin{itemize}
    \item \textbf{Random testing}
    \begin{itemize}
        \item Pure random
        \begin{itemize}
            \item Easiest way to do automatic test case generation
            \item Doesn't require any preparation and is easy to implement
            \item However, there may be semantically redundant inputs. E.g., for a program that computes $\displaystyle\frac{10}{x}$, providing any onput except 0 means the same thing.
            \item For example:
            
            \begin{lstlisting}
//Simple useful test
Set s = new HashSet();
s.add("hi");
assertTrue(s.equals(s));
            
            
            
            \end{lstlisting}

            \item Redundant test:
            \begin{lstlisting}
//Redundant test
Set s = new HashSet();
s.add("hi");
s.add("hi");
assertTrue(s.equals(s));        
            \end{lstlisting}

            \item Another example
            \begin{lstlisting}
//Simple useful test
Date d = new Date(2006,2,14);
assertTrue(d.equals(d));
            \end{lstlisting}

            \item Illegal test:
            \begin{lstlisting}
//Simple useful test
Date d = new Date(2006,2,14);
d.setMonth(-1); // not allowed!
assertTrue(d.equals(d));
            \end{lstlisting}
                
            \item Illegal test:
            \begin{lstlisting}
//Simple useful test
Date d = new Date(2006,2,14);
d.setMonth(-5); // ILLEGAL!
assertTrue(d.equals(d));
            \end{lstlisting}

        \end{itemize}
        \item Rule-based - uses information like past crashes, constraints, etc. to make test cases
        \begin{itemize}
            \item Use rules like boundary cases: $x\in [-2^{32}, 2^{32}]$
            \item Use distributions like normal, bimodal, $\chi^2$, etc. Ask Jason.
        \end{itemize} 
        \item Feedback guided (Radoop)
        \begin{itemize}
            \item Question: It is easy to generate random values for primitive types like \texttt{int}. How do we generate random objects like linked lists, trees, etc.?
            
            \item Use \texttt{id = -1, content = "", child = null}
            
            \begin{lstlisting}
class TreeNode{
    int id; String content; Child child;
    public TreeNode(String str, Child child){
    this.id = Global.id ++;
    this.content = "node:" + str;
    this.child = child;
    }
    public do(){
    String content = this.content.substring(4); // NullPointerException! (at random)
    this.child.do();
    ...
    }
}
            \end{lstlisting}
        \end{itemize} 

        \item Build test inputs incrementally
        \begin{itemize}
            \item New test inputs extend previous ones
            \item In this context, a test input would be a method sequence
        \end{itemize}

        \item As soon as new test input is created, execute it
        \item Use execution result to guide future test case generations
        \begin{itemize}
            \item away from redundant or illegal method sequences
            \item towards sequences that create new object states
            \item do not use duplicate and null objects
            \item do not use objects generated from exceptions
        \end{itemize}

        %%finish taking some notes for slides 42 and 43
    \end{itemize}

    \item \textbf{Search based testing}
    \begin{itemize}
        \item EvoSuite
        \item Deem test case generation as an optimization problem
        \item Based on random testing, and focuses on the input domains
        \item Uses code coverage as guidance
        \item Find input values that can achieve best coverage such as statement coverage, logic coverage, input, etc.
        \item Try to find maximum or minimum value of a certain function
        \item Numerous practical problems can be viewed as optimization problems
        \begin{itemize}
            \item Least cost to travel to a number of cities
            \item Least camera to cover an area
            \item Distribution of stores to attract most customers
            \item Design pipe of systems with least material
            \item Put items into a backpack (with limited volume) of the highest value
        \end{itemize}
    \end{itemize}

    \subsection{Solutions of optimization}
    \begin{itemize}
        \item Hill climbing
        \begin{itemize}
            \item Start from random point
            \item Try all neighbouring points and choose neighbour with highest value until all neighboring points have a lower value than the current point
            \item Easy to find \textbf{local} maxima
        \end{itemize} 

        \item Random restart hill climbing
        \begin{itemize}
            \item Restart hill climbing multiple times to avoid local maxima and get to global maxima
        \end{itemize}

        \item Annealing simulation
        \begin{itemize}
            \item Improved hill climbing
            \item Has probability to move (i.e. restart) after reaching local peak
            \item The probability drops as time goes by
        \end{itemize} 

        \item Genetic algorithm (search based SE)
        \begin{itemize}
            \item Simulate the process of evolution
            \item Start with random points
            \item Select a number of best points
            \item Combine and mutate these points until no more improvements can be made
        \end{itemize}
    \end{itemize}
\end{itemize}

\subsection{Symbolic execution based test generation use cases}
\begin{itemize}
    \item Making a target at a certain code coverage
    \item Understanding how the code works to use this method
    \item Analyze the code structure to find out a path to go to a certain statement \& the constraints of certain inputs to let the program follow the path
    \item Analyzing a program to determine what inputs cause a part of a program to execute
\end{itemize} 

\subsection{The main idea}
\begin{itemize}
    \item If a statement hasn't been covered yet, try and provide an input to go over that statement
    \item A statement is only covered once a path that goes to said statement is covered
    \item So what input will cause the program to go through certain paths? Only if the input satisfies all if conditions along the path!
\end{itemize}

\subsection{Static symbolic execution}

\begin{itemize}
    \item Do not execute the software at all
    \item Construct a constraint for each statement, the statement will be executed when the constraint is satisfied
    \item Solve the constraints
    \item The parameters (variables) in te constraint are inputs of the software to cover a specific statement
    \newpage
    \item \textbf{Basic example}:\\
    Here, \texttt{T} is the condition for the statement to reach this block. \texttt{y=s} is the relationship of all variables to the inputs after the statement is executed. \texttt{s} is a symbolic variable for input
    \begin{lstlisting}
y=read();              //T                                          (y=s)
y=2*y;                 //T                                          (y=2*s)
if(y <= 12)            //T                                          (y=2*s)
    fails();           //T && y <= 12                               (y=2*s)
else
    success();         //T && !(y <= 12)                            (y=2*s)
print("OK");           //T && y <= 12 (y=2*s) || T && !(y<=12)      (y=2*s)
    \end{lstlisting}
\end{itemize}

\subsection{Problems in static symbolic execution}

\begin{itemize}
    \item Path explosion
    \begin{itemize}
        \item Remember $n$ branches will cause $2n$ paths
        \item Infinite paths for unbounded (infinite) loops
        \item Calculating constraints on all paths is infeasible for real software
    \end{itemize} 

    \item Overcomplicated constraints
    \begin{itemize}
        \item Constraints get very complex for large programs (ESPECIALLY when loops are involved)
        \item Not to mention resolving the constraint is an NPC (nondeterministic polynomial complete, or NP complete) problem.
    \end{itemize}

    \item \textbf{Moral of the story}: we only want to usse SSE for functional/method level testing. Anything beyond that will be impractical.
\end{itemize}

\subsection{Dynamic symbolic execution}
\begin{itemize}
    \item One of the revolutionary progress milestones of SWE in the 21st century
\end{itemize}

\subsection{Main idea}
\begin{itemize}
    \item Actually run the program
    \item Generate the constraints as the program runs.
    \item \textbf{Flip the constraints to reach other statements}
\end{itemize} 

\subsection{Symbolic execution challenges}

\begin{itemize}
    \item Scalability
    \begin{itemize}
        \item Key challenge!
        \item Path space of a large program is massive
    \end{itemize} 

    \item Complex non linear constraints
    \item Testing web apps and security problems
    \begin{itemize}
        \item String constraints
        \item Mixed numeric and string constraints
    \end{itemize}

    \item Java Pathfinder for java
    \item PEX
    \begin{itemize}
        \item Visual studio 2010 power tool
    \end{itemize} 

    \item KLEE
    \begin{itemize}
        \item Most popular symbolic execution tool for test generation in C/C++
    \end{itemize} 
\end{itemize}

\end{document}
\documentclass[12pt]{book}

\usepackage[]{amsmath}
\usepackage[]{amsthm}
\usepackage[]{amsfonts}
\usepackage[]{amssymb}
\usepackage{blindtext}
\usepackage[a4paper, total={6in, 8in}]{geometry}
\usepackage{graphicx}
\usepackage{listings}
\usepackage{color}
\usepackage{array}

\definecolor{dkgreen}{rgb}{0,0.6,0}
\definecolor{gray}{rgb}{0.5,0.5,0.5}
\definecolor{mauve}{rgb}{0.58,0,0.82}

\pagenumbering{arabic}
\renewcommand{\chaptername}{Section}
\let\cleardoublepage\clearpage

\lstset{frame=tb,
  language=Java,
  aboveskip=3mm,
  belowskip=3mm,
  showstringspaces=false,
  columns=flexible,
  basicstyle={\small\ttfamily},
  numbers=left,
  numberstyle=\small\color{black},
  keywordstyle=\color{blue},
  commentstyle=\color{dkgreen},
  stringstyle=\color{mauve},
  breaklines=true,
  breakatwhitespace=true,
  tabsize=4
}

\title{EECS4313 week 4}
\author{Jerry Wu}
\date{2024-01-29}

\begin{document}
\maketitle
\tableofcontents

\chapter{Structural coverage}

\begin{quote}
  \textit{One of the biggest testing goals is maximum code coverage, and is still worked towards to this day.}
\end{quote}
\section{Abstract}
If we can have a test suite where every test case covers every statement, then we have 100\% coverage, however this does not mean we have the best test suite, but the likelihood of finding bugs increases as we cover more lines of code.

\begin{quote}
    \textit{"If you don't try, how can you win?" - H.V. Pham 2024}
\end{quote}

\section{Basic example}

\begin{itemize}
  \item $floor(x)$ is some $max(n\in \mathbb{Z}\implies n\leq x)$, i.e. $floor(2.3)=2$
  \begin{lstlisting}
public class Floor{
    public static int floor(int x, int y){
        if(x > y)
            return x/y;
        else
          return y/x;
    }
}
  \end{lstlisting}

  \item What would we test here?
\end{itemize}

\subsection{Test requirement \& test criterion}

\begin{quote}
  \textit{"We need to try all the flavours" - H.V. Pham 2024}
\end{quote}

\begin{itemize}
  \item \textbf{Test requirement}: A specific element of a software artifact that a test must satisfy or cover
  \begin{itemize}
    \item Ice cream cone flavours: vanilla, chocolate, mint
    \item Example: Test one chocolate cone
    \item TR denotes a set of test requirements
  \end{itemize} 

  \item A test criterion is a rule or collection of rules that define the test requirements
  \begin{itemize}
    \item Coverae is one recipe for generating TR in a systematic way
    \item Flavour criterion [cover all flavours] (MORE GENERAL)
    \item So, \texttt{TR=\{f=chocolate, f=vanilla, f=mint\}}
  \end{itemize} 
\end{itemize}

\subsection{Criteria based on structures}
We have four ways to model software in testing:
\begin{itemize}
  \item Graph (control flow graph)
  \item Logical expression (or, and, not, etc.)
  \item Input domain characterization:
  \begin{itemize}
    \item $A:\{0,1,n>1\}$
    \item $B:\{600,700,800\}$
    \item $C:\{swe, cs, isa, infs\}$
  \end{itemize}
  \item Syntactic structure
  \begin{lstlisting}
if x > y z = x - y
else z = 2 * x
  \end{lstlisting} 
\end{itemize}

\section{Measuring test sets}

\begin{itemize}
  \item To test an ice cream shop, how do we know how good a test set is? We can choose the best test set based on widest coverage.
  \begin{itemize}
    \item Test set 1: 1 chocolate cone
    \item Test set 2: 2 chocolate, 1 vanilla
    \item \textbf{Test set 3: 1 chocolate, 1 vanilla, 1 mint} - still not the best one even if we try one of each. We can still try more of each one, just don't get diabetes.
  \end{itemize}
\end{itemize}

\section{Coverage}

Given a set of test requirements $TR$ for a coverage criterion $C$, $T$ satisfies $C$ iff $\forall tr\in TR, \exists t\in T$ that satisfies $tr$ 

\subsection{Coverage level}
\begin{itemize}
  \item Given TR and a test set T, the coverage level is the \textbf{ratio of number of test requirements satisfied by T to the size of TR}
  \begin{itemize}
    \item $TR={f=chocolate, f=vanilla, f=mint}$
    \item $T_1={chocolate(3), vanilla(1)}$
    \item Coverage level = $\frac{2}{3}\approx 66.7\%$
  \end{itemize} 

  \item Coverage levels help us evaluate the goodness of a test set, especially in the presence of infeasible test requirements
\end{itemize}

\subsection{Subsumption}
\begin{itemize}
  \item Criteria subsumption: A test criterion $C_1$ subsumes $C_2$ iff \textbf{every} set of test cases that satisfies criterion $C_1$ also satisfies $C_2$
  \item Must be true $\forall T$, i.e. $\forall t\in C_1$, $t$ satisfies $C_2$.
  \item A common example is that edge coverage subsumes node coverage, since if we visit every edge between two nodes, we also visit every node.
  \item Subsumption is a rough guide for comparing criteria, although it is hard to use in practice
\end{itemize}

\subsection{Moral of the story}
Maximum coverage can only go so far for detecting bugs. But just because we have high code coverage doesn't mean we can find every bug. At the end of the day, finding as many bugs as possible is the true goal.

\section{Control flow graph definitions and terminology}

\begin{itemize}
  \item $N$: \textbf{Set of all nodes} $\{A,B,C,D,E\}$
  \item $N_0$: \textbf{Set of initial nodes} $\{A\}$ - doesn't usually happen in code
  \item $N_f$: \textbf{Set of final nodes} $\{E\}$ - possible in code since there can be multiple terminating conditions, return statements, etc.
  \item $E\subseteq N\times N$: \textbf{Edges}, e.g. $(A,B)$ and $(C,E)$, etc. $C$ is the predecessor and $E$ is the successor in $(C,E)$.
\end{itemize}

\subsection{Subgraphs}

Leg $G'$ be a subgraph if $G$; then
\begin{itemize}
  \item $\forall v\in G'$, $v$ must be a subset $N_{\subseteq}$ of $N$
  \item $\forall v_0\in G'$ are $N_0 \cap N$
  \item The edges of $G'$ are $E\cap (N_{\subseteq}\times N_{\subseteq})$
\end{itemize}

\subsection{Path}

Recall that a path in a graph is a sequence of nodes from a graph $G$ whose adjacent pairs all belong to the set of edges $E$ of $G$.

\begin{itemize}
  \item Path A: $\{2,3,5\}$, length is 2
  \item Path B: $\{1,2,3,4,5,6\}$ length is 4
\end{itemize}

Paths have to be connected, and if the graph is directed, needs to go in the correct direction.

\subsection{Subpath}
It's just a smaller path within a bigger path. Self explanatory.

\subsection{Test path}

A test path is a path $p$ (possibly of length 0) that starts at some $v_0\in N_0$ and ends at some $v_f \in N_f$. In other words, it's just a simulation of how the program runs.

\subsection{Paths \& semantics}
\begin{itemize}
  \item Some paths in a control flow graph might not correspond to a program's semantics. For example:

  \begin{lstlisting}
if(false)
    gamma();
    return;
beta();
  \end{lstlisting}
  In this case, $\beta$ is never executed since it is a concrete false statement.

  \item In this course, we generally only talk about the syntax of the graph and not the semantics

\end{itemize}

\section{Reachability}
define $reach_G(x)$ as the subgraph that is syntactically reachable from $x$, where $x$ is a single node, an edge, or a set of nodes and edges.

\subsection{Syntactical and semantic reachability}

\begin{itemize}
  \item A node $n$ is \textbf{syntactically} reachable from $n_i$ if $\exists$ path $(n_i,n)$, e.g. $\gamma$ from $\alpha$ or $\beta$ from $\alpha$
  \item A node is \textbf{semantically} reachable if one of the paths from $n_i$ to $n$ can be reached on the same input, e.g. $\gamma$ from $\alpha$ but \textbf{not} $\beta$ from $\alpha$
\end{itemize}

\subsection{SESEs (single entry, single exit graphs)}
SESE graphs definition: All test paths start at a single node and end at some other node.
\begin{itemize}
  \item $N_0$ and $N_f$ have exactly one node
  \item A path is said to \textbf{tour} all of its subpaths
  \item Any path \textbf{tours} itself
\end{itemize}

Take the following set of paths:

\begin{align*}
  p_A=\left[0,1,3,4,6\right]\\
  p_B=\left[0,1,3,5,6\right]\\
  p_C=\left[0,2,3,4,6\right]\\
  p_D=\left[0,2,3,5,6\right]
\end{align*}

\begin{itemize}
  \item $p_0=\left[1,3,4\right]$ is a subpath of $p_A\therefore p_A$ tours $p_0$ 
\end{itemize}

\section{Connect test cases and test paths}

\begin{itemize}
  \item Connect test cases and test paths with a mapping \texttt{path()} from test cases to test paths
  \begin{itemize}
    \item E.g. $path(t)$ is the set of test paths corresponding to test case $t$
    \item Given a test suite $T$, $path(T)=\{path(t)|t\in T\}$
  \end{itemize}

  \item Each test case gives at least one test path
  \begin{itemize}
    \item If the software is \textbf{deterministic}, then each test case gives exactly one test path
    \item Otherwise, a test case is connected to multiple test paths and vice versa.
  \end{itemize}
\end{itemize}

\subsection{Deterministic vs non deterministic CFGs}

\begin{itemize}
  \item Generally, we want to avoid NFA control flow graphs at all costs, but things like java \texttt{hashCode()} implementations can cause nondeterminism
  \item We want to avoid it because it makes it hard to check test case outputs since more than one output can be a valid result of the test case
  \item Other causes can stem from thread scheduler ad memory address
\end{itemize}

\subsection{Node coverage}

\begin{quote}
  \textit{"You need to choose wisely" - H.V. Pham 2024}
\end{quote}
\begin{itemize}
  \item \textbf{Node coverage is the same as statement coverage}. For each node $n\in reach_G\left[N_0\right]$, $TR$ contains a requirement to visit ndoe $n$
  \begin{itemize}
    \item $TR$ visits each reachable $n\in G$
    \item $TR=\{0,1,2,3,4,5,6\}$
  \end{itemize} 
\end{itemize}

\subsection{Edge \& edge pair coverage}

\begin{itemize}
  \item Edge coverage: $TR$ contains each reachable path of length up to 1, inclusive in $G$
  \item $TR=\{[1,2],[2,4],\ldots\}$
  \item Edge pair coverage: $TR$ contains each reachable path of up to length 2 inclusive in $G$
  \item $TR=\{[1,2,3],[1,2,4],\ldots\}$
  \item Edge coverage \textbf{subsumes} node coverage, so it is a stronger form of coverage than node coverage
  \item However, if the CFG is a diamond, then both are equally good
\end{itemize}

\subsection{Simple path}

\begin{itemize}
  \item A path is \textbf{simple} if no node appears more than once in a path, except that the first and last nodes may be the same
  \item Some properties of simple paths:
  \begin{itemize}
    \item No internal loops
    \item Can bound their length
    \item Can create any path by composing simple paths
    \item Many simple paths exist (too many!)
  \end{itemize} 

  \item Examples
  \begin{itemize}
    \item $p=[1,2,3,5,6,7]$
    \item $p=[1,2,4]$
    \item $p=[2,3,4,5,2]$
  \end{itemize} 
\end{itemize}

\subsection{Prime path}

\begin{itemize}
  \item Because there are too many simple paths, let's consider prime paths, which are \textbf{simple paths of maximum length}
  \item A path is \textbf{prime} if it is \textbf{simple} and \textbf{does not appear as a proper subpath} of any other simple path
\end{itemize}

\subsection{Prime path examples}



\end{document}